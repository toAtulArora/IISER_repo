%% LyX 2.1.3 created this file.  For more info, see http://www.lyx.org/.
%% Do not edit unless you really know what you are doing.
\documentclass[english]{article}
\usepackage{ccfonts}
\usepackage[T1]{fontenc}
\usepackage[latin9]{inputenc}
\usepackage{geometry}
\geometry{verbose,tmargin=3cm,bmargin=3cm,lmargin=3cm,rmargin=3cm,headheight=3cm,headsep=3cm,footskip=3cm}
\setcounter{secnumdepth}{3}
\setcounter{tocdepth}{3}
\usepackage{url}
\usepackage{amsmath}
\usepackage{cancel}

\makeatletter
%%%%%%%%%%%%%%%%%%%%%%%%%%%%%% Textclass specific LaTeX commands.
\usepackage{beamerarticle,pgf}
% this default might be overridden by plain title style
\newcommand\makebeamertitle{\frame{\maketitle}}%
\AtBeginDocument{
\let\origtableofcontents=\tableofcontents
\def\tableofcontents{\@ifnextchar[{\origtableofcontents}{\gobbletableofcontents}}
\def\gobbletableofcontents#1{\origtableofcontents}
}

%%%%%%%%%%%%%%%%%%%%%%%%%%%%%% User specified LaTeX commands.
\usepackage{listings}

\makeatother

\usepackage{babel}
\begin{document}

\title{Measuring Geometric Phase using 3 Pin Holes :.}

\makebeamertitle
We describe a method a of measuring the geometric phase optically,
using 3 pin holes. This method is in stark contrast with conventional
methods where states are evolved and the phase


\section{Introduction}


\subsection{Geometric Phase}

Concept of geometric phase was discovered as a relative phase that
is introduced in an adiabatic and cyclic evolution of a state, between
the final and initial state. This relative phase has both the geometric
phase and what is known as the dynamic phase, which is the time integral
of the Hamiltonian. The concept since then, has been generalized to
work in cases of non-adiabatic evolution and then even to non-cyclic
evolution. 


\subsection{Conventional Measurement Scheme}

Experimentally, conventional measurement of the geometric phase is
associated with directly measuring the total phase difference between
the evolved and unevolved reference states. This requires phase calibration,
which is an experimental complication. Further, one must eliminate
the dynamic phase to obtain the geometric phase. 


\subsection{Kinematic Geometric Phase and the Proposed Scheme}

In their paper {[}quote{]}, a method of measurement of geometric phase
has been proposed, which is built on the kinematic approach to geometric
phase. In this approach, the geometric phase is attributed to the
structure of the Hilbert space itself and doesn't require dynamics
(evolution of states). One can show that Geometric Phase is a ray
space object and more explicitly, it has been shown that for n points
in the ray space, there exists an entity known as the Bargmann invariant,
from which the geometric phase can be recovered. This geometric phase
is that for a closed curve obtained by connecting the n points with
geodesics.

The proposed scheme involves use of 3 quantum states and letting them
interfere directly, viz. without evolving the system. From the intereferogram
thus obtained, using the 3 vertex Bargmann invariant and some more
details specific to the setup, it has been shown that the geometric
phase can be recovered. Further, since the initial 3 states are known,
one can theoretically evaluate the Geometric phase. Theory and experimental
observations have been shown to agree. Since this method doesn't require
evolution, the need for phase calibration and removal of dynamic phase
has been eliminated.


\section{Geometric Phase and Bargmann Invariant}

\[
I\propto1+\left|\left\langle \psi_{1}|\psi_{2}\right\rangle \right|\cos\left(\phi+\arg\left\langle \psi_{1}|\psi_{2}\right\rangle \right)
\]


which can be derived easily as 
\[
\begin{aligned}\left|\psi\right\rangle  & =e^{i\phi}\left|\psi_{1}\right\rangle +\left|\psi_{2}\right\rangle \\
\left\langle \psi|\psi\right\rangle  & =\left(\left\langle \psi_{1}\right|e^{-i\phi}+\left\langle \psi_{2}\right|\right)\left(e^{i\phi}\left|\psi_{1}\right\rangle +\left|\psi_{2}\right\rangle \right)\\
 & =\left\langle \psi_{1}|\psi_{1}\right\rangle +\left\langle \psi_{2}|\psi_{2}\right\rangle +\left(e^{-i\phi}e^{i\text{arg}\left\langle \psi_{1}|\psi_{2}\right\rangle }\left|\left\langle \psi_{1}|\psi_{2}\right\rangle \right|+\text{hc}\right)\\
 & \propto1+\left|\left\langle \psi_{1}|\psi_{2}\right\rangle \right|\cos\left(\phi+\arg\left\langle \psi_{1}|\psi_{2}\right\rangle \right)
\end{aligned}
\]


TODO: Check this

\[
\begin{aligned}\Delta_{3}\left(\psi_{1},\psi_{2},\psi_{3}\right) & \equiv\left\langle \psi_{1}|\psi_{2}\right\rangle \left\langle \psi_{2}|\psi_{3}\right\rangle \left\langle \psi_{3}|\psi_{1}\right\rangle \\
\text{arg}\left(\Delta_{3}\left(\psi_{1},\psi_{2},\psi_{3}\right)\right) & =\arg\left(\left\langle \psi_{1}|\psi_{2}\right\rangle \left\langle \psi_{2}|\psi_{3}\right\rangle \left\langle \psi_{3}|\psi_{1}\right\rangle \right)\\
 & =\sum_{(i,j)\text{cycle}}\text{arg}\left\langle \psi_{j}|\psi_{j}\right\rangle 
\end{aligned}
\]
if $\text{arg}\left\langle \psi_{1}|\psi_{2}\right\rangle =0$ and
$\text{arg}\left\langle \psi_{2}|\psi_{3}\right\rangle =0$ then $\text{arg}\left(\Delta_{3}\left(\psi_{1},\psi_{2},\psi_{3}\right)\right)=\text{arg}\left\langle \psi_{1}|\psi_{3}\right\rangle $


\section{Geometric Phase and Ridge Lines}

\[
\left|\psi(\mathbf{r})\right\rangle =C\sum_{j=1}^{3}\frac{\text{exp}\left[i\left(k\left|\mathbf{R}-\mathbf{a}_{j}\right|+\phi_{j}\right)\right]}{\left|\mathbf{R}-\mathbf{a}_{j}\right|}\left|\psi_{j}\right\rangle 
\]
where $\mathbf{R}$ is the position vector on the observation plane
($z=L$); $\mathbf{r}\equiv\mathbf{R}-\left(\mathbf{R}.\mathbf{z}\right)\mathbf{z}$
(component of $\mathbf{R}$ perpendicular to $\mathbf{z}$, the unit
vector along z-axis); $C$ is dimensionless normalization; $\phi_{j}$
is the phase of $\left|\psi_{j}\right\rangle $ (the polarization
state of the $j^{th}$ source). If we make the para-axial approximation,
we get

\[
\begin{aligned}\mathbf{R}-\mathbf{a}_{j} & =\left(L\mathbf{z}+\mathbf{r}\right)-\mathbf{a}_{j}\\
\left|\mathbf{R}-\mathbf{a}_{j}\right| & =\left[\left(\left(L\mathbf{z}+\mathbf{r}\right)-\mathbf{a}_{j}\right)\left(\left(L\mathbf{z}+\mathbf{r}\right)-\mathbf{a}_{j}\right)\right]^{1/2}\\
 & =\left[\left(L\mathbf{z}+\mathbf{r}\right)\left(L\mathbf{z}+\mathbf{r}\right)-2\left(L\mathbf{z}+\mathbf{r}\right)\mathbf{a}_{j}+\mathbf{a}_{j}^{2}\right]^{1/2}\\
 & =\left[L^{2}+2L\cancelto{0}{\mathbf{z}.\mathbf{r}}+\mathbf{r}^{2}-2(L\cancelto{0}{\mathbf{z}.\mathbf{r}}+\mathbf{r}.\mathbf{a}_{j})+\mathbf{a}_{j}^{2}\right]^{1/2}\\
 & =\left[L^{2}+2\frac{r^{2}+a^{2}}{2}-2(\mathbf{r}.\mathbf{a}_{j})\right]^{1/2}\\
 & =L\left[1+2\left(\frac{r^{2}+a^{2}}{2L^{2}}-\frac{(\mathbf{r}.\mathbf{a}_{j})}{L^{2}}\right)\right]^{1/2}\\
 & \approx L\left[1+\frac{r^{2}+a^{2}}{2L^{2}}-\frac{(\mathbf{r}.\mathbf{a}_{j})}{L^{2}}\right]\\
 & =L+\frac{r^{2}+a^{2}}{2L}-\frac{(\mathbf{r}.\mathbf{a}_{j})}{L}
\end{aligned}
\]
\[
\left|\psi(\mathbf{r})\right\rangle \approx C\sum_{j=1}^{3}\frac{\text{exp}\left[i\left(k\left(L+\frac{r^{2}+a^{2}}{2L}-\frac{(\mathbf{r}.\mathbf{a}_{j})}{L}\right)+\phi_{j}\right)\right]}{L}\left|\psi_{j}\right\rangle 
\]
where we've used the fact that $\mathbf{a}_{j}^{2}=a^{2}$. TODO:
Figure why in the denominator we can get away with writing L only

The intensity distribution can then be written as 
\[
\begin{aligned}p(x,y) & =\left\langle \psi(\mathbf{r})|\psi(\mathbf{r})\right\rangle \\
 & =\frac{C^{2}}{L^{2}}\sum_{j,k=1}^{3}\text{exp}\left[-i\left(k\left(\cancel{L+\frac{r^{2}+a^{2}}{2L}}-\frac{(\mathbf{r}.\mathbf{a}_{j})}{L}\right)+\phi_{j}\right)\right]\text{exp}\left[i\left(k\left(\cancel{L+\frac{r^{2}+a^{2}}{2L}}-\frac{(\mathbf{r}.\mathbf{a}_{k})}{L}\right)+\phi_{k}\right)\right]\left\langle \psi_{j}|\psi_{k}\right\rangle \\
 & =\frac{C^{2}}{L^{2}}\left\{ 3+\sum_{j\neq k=1}^{3}\text{exp}\left[i\left(k\left(\frac{(\mathbf{r}.\mathbf{a}_{j})}{L}-\frac{(\mathbf{r}.\mathbf{a}_{k})}{L}\right)+\phi_{k}-\phi_{j}+\text{arg}\left\langle \psi_{j}|\psi_{k}\right\rangle \right)\right]\left|\left\langle \psi_{j}|\psi_{k}\right\rangle \right|\right\} \\
 & =\frac{C^{2}}{L^{2}}\left\{ 3+\sum_{\left(j,k\right)\text{cycle}}\cos\left[k\left(\frac{(\mathbf{r}.\mathbf{a}_{j})}{L}-\frac{(\mathbf{r}.\mathbf{a}_{k})}{L}\right)+\phi_{k}-\phi_{j}+\text{arg}\left\langle \psi_{j}|\psi_{k}\right\rangle \right]\left|\left\langle \psi_{j}|\psi_{k}\right\rangle \right|\right\} \\
 & =\frac{C^{2}}{L^{2}}\left\{ -3+\sum_{(i,j)\text{cycle}}P_{ij}(x,y)\right\} 
\end{aligned}
\]


\[
p(x,y)=\frac{C^{2}}{L^{2}}\left\{ -3+\sum_{(i,j)\text{cycle}}P_{ij}(x,y)\right\} 
\]
where 
\[
P_{ij}(x,y)\equiv2\left(1+\cos\left[(\mathbf{k}_{ij}.\mathbf{r})-\phi_{ij}+\text{arg}\left\langle \psi_{i}|\psi_{j}\right\rangle \right]\left|\left\langle \psi_{i}|\psi_{j}\right\rangle \right|\right)
\]
with $k_{ij}\equiv k(\mathbf{a}_{i}-\mathbf{a}_{j})/L$ and $\phi_{ij}\equiv\phi_{i}-\phi_{j}$

Condition for maxima then is 

\[
(\mathbf{k}_{ij}.\mathbf{r})-\phi_{ij}+\text{arg}\left\langle \psi_{i}|\psi_{j}\right\rangle =2n_{ij}\pi
\]
which is of the from $\mathbf{r}.\mathbf{k}=c$ (in 3d, this is a
plane, in 2d, its a line obviously) 
\[
\begin{aligned}(x\mathbf{x}+y\mathbf{y}).(k_{x}\mathbf{x}+k_{y}\mathbf{y}+k_{z}\mathbf{z}) & =c\\
 & =xk_{x}+yk_{y}
\end{aligned}
\]
Using the formula for finding the are of a triangle, given equations
of the lines making the edges as 
\[
\begin{aligned}a_{1}x+b_{1}y+c_{1} & =0\\
a_{2}x+b_{2}y+c_{2} & =0\\
a_{3}x+b_{3}y+c_{3} & =0
\end{aligned}
\]
is 
\[
\frac{\text{det}\left[\begin{array}{ccc}
a_{1} & b_{1} & c_{1}\\
a_{2} & b_{2} & c_{2}\\
a_{3} & b_{3} & c_{3}
\end{array}\right]^{2}}{2C_{1}C_{2}C_{3}}
\]
where $C_{i}=\text{cofactor of }c_{i}$ to finally obtain

\[
S=\frac{L^{2}}{4k^{2}S_{0}}\left\{ \text{arg}\left[\Delta_{3}\left(\psi_{1},\psi_{2},\psi_{3}\right)\right]-2n\pi\right\} ^{2}
\]
where $n=n_{12}+n_{23}+n_{31}$


\section{Extraction of Ridge Lines}

For arbitrary $\mathbf{b},\,\mathbf{k}$ we have

\[
\begin{aligned}\mathbf{b}.\mathbf{\nabla}\left(\mathbf{k}.\mathbf{r}\right) & =b_{i}\frac{\partial}{\partial x_{i}}k_{j}r_{j}\\
 & =b_{i}k_{j}\left(\frac{\partial}{\partial x_{i}}r_{j}\right)\\
 & =b_{i}k_{j}\delta_{ij}\\
 & =\mathbf{b}.\mathbf{k}
\end{aligned}
\]
Now we define 
\[
\mathbf{b}_{i}\equiv\mathbf{e}_{z}\times(\mathbf{a}_{j}-\mathbf{a}_{k})=\frac{L}{k}\mathbf{e}_{z}\times\mathbf{k}_{jk}
\]
where $(i,j,k)=(1,2,3),(2,3,1),(3,1,2)$ NB: $\mathbf{b}_{i}\mathbf{k}_{jk}=0$

Recall: $P_{ij}(x,y)\equiv2\left(1+\cos\left[(\mathbf{k}_{ij}.\mathbf{r})-\phi_{ij}+\text{arg}\left\langle \psi_{i}|\psi_{j}\right\rangle \right]\left|\left\langle \psi_{i}|\psi_{j}\right\rangle \right|\right)$

\[
\begin{aligned}\left(\mathbf{b}_{1}.\nabla\right)p(x,y) & =\left(\mathbf{b}_{1}.\mathbf{\nabla}\right)\frac{C^{2}}{L^{2}}\left\{ -3+\sum_{(i,j)\text{cycle}}P_{ij}(x,y)\right\} \\
 & \propto\sum_{(i,k)\text{cycle}}\left(\mathbf{b}_{1}.\mathbf{\nabla}\right)P_{ik}(x,y)\\
 & \propto\sin\left[(\mathbf{k}_{12}.\mathbf{r})-\phi_{12}+\text{arg}\left\langle \psi_{1}|\psi_{2}\right\rangle \right]\left|\left\langle \psi_{1}|\psi_{2}\right\rangle \right|\cancelto{\mathbf{b}_{1}.\mathbf{k}_{12}}{\left(\mathbf{b}_{1}.\mathbf{\nabla}\right)(\mathbf{k}_{12}.\mathbf{r})}\\
 & +\sin\left[(\mathbf{k}_{23}.\mathbf{r})-\phi_{23}+\text{arg}\left\langle \psi_{2}|\psi_{3}\right\rangle \right]\left|\left\langle \psi_{2}|\psi_{3}\right\rangle \right|\cancelto{0}{\mathbf{b}_{1}.\mathbf{k}_{23}}\\
 & +\sin\left[(\mathbf{k}_{31}.\mathbf{r})-\phi_{31}+\text{arg}\left\langle \psi_{3}|\psi_{1}\right\rangle \right]\left|\left\langle \psi_{3}|\psi_{1}\right\rangle \right|\mathbf{b}_{1}.\mathbf{k}_{31}
\end{aligned}
\]


\[
\begin{aligned}\left(\mathbf{b}_{2}.\nabla\right)\left(\mathbf{b}_{1}.\nabla\right)p(x,y) & \propto-\cos\left[(\mathbf{k}_{12}.\mathbf{r})-\phi_{12}+\text{arg}\left\langle \psi_{1}|\psi_{2}\right\rangle \right]\left|\left\langle \psi_{1}|\psi_{2}\right\rangle \right|\mathbf{b}_{1}.\mathbf{k}_{12}\mathbf{b}_{2}.\mathbf{k}_{12}\\
 & -\cos\left[(\mathbf{k}_{31}.\mathbf{r})-\phi_{31}+\text{arg}\left\langle \psi_{3}|\psi_{1}\right\rangle \right]\left|\left\langle \psi_{3}|\psi_{1}\right\rangle \right|\mathbf{b}_{1}.\mathbf{k}_{31}\cancelto{0}{\mathbf{b}_{2}.\mathbf{k}_{31}}\\
 & \propto\cos\left[(\mathbf{k}_{12}.\mathbf{r})-\phi_{12}+\text{arg}\left\langle \psi_{1}|\psi_{2}\right\rangle \right]\left|\left\langle \psi_{1}|\psi_{2}\right\rangle \right|
\end{aligned}
\]


NB: $\left(\mathbf{b}_{2}.\nabla\right)\left(\mathbf{b}_{1}.\nabla\right)p(x,y)=\left(\mathbf{b}_{1}.\nabla\right)\left(\mathbf{b}_{2}.\nabla\right)p(x,y)$

So in general then, I have

\[
\left(\mathbf{b}_{i}.\mathbf{\nabla}\right)\left(\mathbf{b}_{j}.\mathbf{\nabla}\right)p(x,y)\propto\cos\left[(\mathbf{k}_{ij}.\mathbf{r})-\phi_{ij}+\text{arg}\left\langle \psi_{i}|\psi_{j}\right\rangle \right]\left|\left\langle \psi_{i}|\psi_{j}\right\rangle \right|
\]



\section{References}

\url{http://math.stackexchange.com/questions/901819/direct-formula-for-area-of-a-triangle-formed-by-three-lines-given-their-equatio}

\url{http://en.wikipedia.org/wiki/Bloch_sphere}
\end{document}
