%% LyX 2.1.3 created this file.  For more info, see http://www.lyx.org/.
%% Do not edit unless you really know what you are doing.
\documentclass[english]{article}
\usepackage{ccfonts}
\usepackage[T1]{fontenc}
\usepackage[latin9]{inputenc}
\usepackage{geometry}
\geometry{verbose,tmargin=3cm,bmargin=3cm,lmargin=3cm,rmargin=3cm,headheight=3cm,headsep=3cm,footskip=3cm}
\usepackage{amsmath}
\usepackage{amssymb}
\usepackage{cancel}
\usepackage{stmaryrd}
\usepackage{setspace}
\usepackage{esint}
\onehalfspacing

\makeatletter
%%%%%%%%%%%%%%%%%%%%%%%%%%%%%% User specified LaTeX commands.
\usepackage{listings}

\makeatother

\usepackage{babel}
\begin{document}

\title{Dirac and Majorana Mass}


\author{Atul Singh Aurora}


\date{April 15, 2015}
\maketitle
\begin{abstract}
The concept of mass is intuitively linked with inertia. It is first
encountered in a quantitatively precise formulation, in the work of
Newton. With Einstein's special relativity, it acquires a new meaning
($p^{2}=m^{2}$) (and ofcourse again, with identification of gravitational
mass and inertial mass; general relativity). In particle physics (the
study of `fundamental' particles) historically, it has been hard and
consequently interesting to formulate a consistent theory of particles
interacting in the way we know they do from experiments, and implement
their masses, taking inspiration from one of the most successful theory
of physics, Quantum Electrodynamics (quantum generalization of classical
electrodynamics). Assuming the knowledge of STR and QM, I'll motivate
the concept of fields and particles and briefly describe how it all
fits to describe fundamental particles. Thereafter, I'll take the
example of a Klein Gordan Field (obtained by replacing $E$ with $-i\hbar\frac{\partial}{\partial t}$
and $\vec{p}$ with $i\hbar\vec{\nabla}$ in $p^{2}=m^{2}$) to explain
the origin and definition of mass. Then I'll move to the main topic,
that of Dirac and Majorana Mass, starting with the Dirac Equation.
I'll describe the physical significance of this investigation, viz.
why it is a candidate in describing the mass of neutrinos (a particle
which was suspected to exist, assuming conservation of energy holds,
by observing the spectrum of beta decay) and hint on alternatives.
\end{abstract}

\section{Prerequisites}

From CM,
\begin{itemize}
\item Definition of $L$, the Lagrangian
\item Euler Lagrange equations and their derivation
\item Noether's Theorem that relates conserved quantities and continuous
symmetries of the Lagrangian
\end{itemize}
From STR, I'll use the following conventions
\begin{itemize}
\item $\eta^{\mu\nu}=\text{diag}(1,-\vec{1})$ (NB: $\eta^{T}=\eta^{-1}=\eta$)
\item $c=1$, $\hbar=1$
\item indices 

\begin{itemize}
\item $i,j,k,l$ etc. run from 1 to 3
\item $\alpha,\beta,\gamma$ etc. run from 0 to 3
\end{itemize}
\end{itemize}
and need the four vector notation. Recall 
\begin{itemize}
\item the Einstein summing convention, $A^{\alpha}B_{\alpha}=\sum_{\alpha=0}^{4}A^{\alpha}B_{\alpha}$
(contractions)
\item if $A^{\alpha}=(A^{0},\vec{A})$ then $A_{\alpha}\equiv\eta_{\alpha\beta}A^{\beta}=(A^{0},-\vec{A})$
\item Under a Lorentz transformation $\lambda_{\beta}^{\alpha}$, $A^{\alpha}\to A'^{\alpha}=\lambda_{\beta}^{\alpha}A^{\beta}$;
objects with all indices contracted are invariant under lorentz transformations
(NB: $\lambda^{T}\eta\lambda=\eta$)
\end{itemize}
From QM, I'll need the postulates. Recall
\begin{itemize}
\item System state: State of a system is described by a ket (or a wavefunction)
$\left|\psi\right\rangle $ (or $\psi(x)=\left\langle x|\psi\right\rangle $)
\item Time evolution: Given a Hermitian operator $H$ (viz. $H^{\dagger}=H$;
where $H^{\dagger}\equiv H^{*T}$), 
\[
H\left|\psi\right\rangle =-i\hbar\frac{\partial}{\partial t}\left|\psi\right\rangle 
\]
When $H$ has no time dependence, then $\left|\psi(t)\right\rangle =e^{(-i\hbar)^{-1}Ht}\left|\psi(0)\right\rangle $.
NB: $U\equiv e^{(-i\hbar)^{-1}Ht}$, is unitary, viz. $U^{\dagger}=U^{-1}$
(OR, TODO: more generally, a unitary operator)
\item Measurement/Observables: 

\begin{itemize}
\item Upon being measured, the system collapses into one of the eigenstates
of the operator corresponding to the measurement
\item The probability that the system will collapse to the state $\left|n\right\rangle $,
is given by $\left|\left\langle n|\psi\right\rangle \right|^{2}$,
where $\left|n\right\rangle $ is the $n^{th}$ eigenstate of the
system.
\end{itemize}
\end{itemize}
and I'll need the basics of the quantum harmonic oscillator developed
using the $a$ and $a^{\dagger}$ notation.

I'll further use the usual conventions for the pauli matrices
\[
\begin{aligned}\sigma^{1} & = & \left(\begin{array}{cc}
0 & 1\\
1 & 0
\end{array}\right)\\
\sigma^{2} & = & \left(\begin{array}{cc}
0 & -i\\
i & 0
\end{array}\right)\\
\sigma^{3} & = & \left(\begin{array}{cc}
1 & 0\\
0 & -1
\end{array}\right)
\end{aligned}
\]


Terminology from particle physics maybe helpful
\begin{itemize}
\item Leptons {[}TODO: Complete this part{]}
\item Quarks
\end{itemize}

\section{Towards a quantum theory of fields}


\subsection{Motivation}

One can ask for a theory that links QM and STR, solely on grounds
that QM is built on top of classical mechanics which we know is bound
to fail when we goto higher speeds (thus energies). Practically though
that's not such a big issue. However, when one looks at collision
of particles, but can't make any predictions about the products, except
being able to veto products on whose mass is greater than the centre
of mass energy (solely on the energy momentum relation based on STR),
there's a real crisis. The idea is to build a theory that can automatically
handle the creation and destruction of particles in a way that's fully
consistent with STR and at the same time, predict the results of colliding
high energy particles.


\subsection{Known Issues}

One may guess that instead of writing $H_{\text{classical}}\psi=-i\hbar\frac{\partial}{\partial t}\psi$,
if we write the relativistic version as $\left(E^{2}-\vec{p}^{2}\right)\psi=m^{2}\psi$
and replace $E\to-i\frac{\partial}{\partial t}$, $\vec{p}\rightarrow i\vec{\nabla}$,
we get 
\[
\boxempty\psi=m^{2}\psi
\]
where $\boxempty=\frac{\partial}{\partial t}^{2}-\vec{\nabla}^{2}$,
it might just solve the problem. As it turns out, this attempt fails
miserably, leading to issues in causality (events in the past are
affected by the future), negative energies (this means there's no
energy minima, which implies there's no stable ground state) among
others. 

One can actually predict that things will get inconsistent simply
by observing that in QM, time and position are treated very differently;
one is a parameter, the other's an operator. STR seamlessly mixes
the two. This meant that we need to either make time an operator (which
again is known to not work) or demote position from an operator to
a parameter. All these hint to interpreting the object $\psi$ differently.


\subsection{Concept of field}

This idea almost sounds like it's been taken from a fairy tale, even
though we're familiar with a similar notion. It is that of a field.
Fields were first encountered in the context of electromagnetism.
In the beginning, we use it as a device for calculating forces, and
once we reach the Maxwell equations, we find that these fields can
exist without sources that we thought they were an artifact of, to
start with.

In this context, there's a field for every fundamental\footnote{I know I haven't defined the word `fundamental' yet, but its intent
and meaning will become obvious as we proceed. } particle. This field can create particles out of thin air and destroy
them too. These fields can talk to each other and consequently the
particles can interact. If we can construct a framework in which if
we write a relativistically covariant (or invariant?) Lagrangian describing
the fields and their interaction, it produces for us probabilities
of all possible processes that involve these particles, then, then
that does it in principle. It'll only remain to write the right Lagrangian
(which has taken years to pin down exactly and is known as the Standard
Model\footnote{Except ofcourse neutrino masses and oscillations})


\subsection{Framework: QFT}

Let us start with a classical field. Consider the simplest case, a
real scalar field. Imagine a number being assigned to every location
in space and time. If I demand that the field must satisfy the Klien
Gordan equation\footnote{where note now that $m$ is only a parameter to start with},
I can write a Lagrangian\footnote{density to be precise} 
\[
\mathcal{L}=\frac{1}{2}\left(\partial^{\mu}\phi\partial_{\mu}\phi+m^{2}\phi\right)
\]
by noting that the equation of motion produced by this $\mathcal{L}$
is the Klien Gordan equation itself. Note that in general, $\phi=\phi(t,\vec{x})$.
If I further assume that it is sufficiently well behaved, I can write
\[
\phi=\int\frac{d^{3}p}{\left(2\pi\right)^{3}\sqrt{2\omega_{p}}}\,\left(ae^{i\mathbf{px}}+a^{\dagger}e^{-i\mathbf{px}}\right)
\]
with $a=a(\vec{p})$ and the second term added explicitly to ensure
$\phi$ is real\footnote{$\frac{d^{3}p}{\left(2\pi\right)^{3}\sqrt{2\omega_{p}}}$ is a Lorentz
invariant measure which I won't discuss here}. I can also evaluate $\pi$, the conjugate momentum from $\mathcal{L}$.
If I impose now, the commutation (equal time) $[\phi(t,\mathbf{x}),\pi(t,\mathbf{x}')]=i\delta(\mathbf{x}-\mathbf{x}')$,
(note, we just converted our classical field to a quantum field) it
can be shown that $[a(\mathbf{p}),a^{\dagger}(\mathbf{p}')]\sim\delta(\mathbf{p}-\mathbf{p}')$.
Further, it can be shown that the Hamiltonian for the given Lagrangian,
turns out to be 
\[
H\sim a^{\dagger}a+\frac{1}{2}[a(\mathbf{p}),a^{\dagger}(\mathbf{p})]
\]
This now has striking similarities to the quantum harmonic oscillator.
This motivates existence of states such as $a^{\dagger}(\vec{p})\left|\text{vacuum}\right\rangle $
which can be rigerously shown to be particles indeed. From the space-time
translation invariance, one can arrive\footnote{using Noether's theorem},
at the physical momentum and energy operators. If the aforesaid state
indeed represents a particle, it must satisfy $E^{2}-\vec{p}^{2}=m^{2}$.
And infact, it so turns out that applying the momentum operator and
the hamiltonian operator and finding the eigenvalues of this state
produces exactly that. Thus we conclude that this state indeed represents
a particle that has a definite energy and momentum. Further, we now
know that the parameter $m$ that appeared in the Lagrangian was indeed
the mass.

Obviously, the example we took had only one field which interacts
with nothing. Any observable field must interact to be observable.
QFT is developed then as a perturbation theory, expanded around the
non-interacting part, such as the one just described. This results
in what are known as Feynman rules. These rules can be used to evaluate
quantities such as decay rates and scattering cross sections, which
can be measured experimentally.

In the discussion of mass, it must be mentioned that when interactions
are introduced, the term $m$ in the Lagrangian, no longer represents
mass. In that case, it is defined as the pole of the `full propogator'.
I'll leave it at that.


\section{Dirac and Majorana Mass}


\subsection{The Dirac Equation}

Let us now consider the famous Dirac equation and the particles its
field produces. I must begin with motivating the Dirac equation, because
it can be overwhelming at first. The idea was to factor the Klien
Gordan equation. So I start with writing an equation of the form 
\[
(i\partial_{\mu}\gamma^{\mu}-m)\psi=0
\]
where $\gamma^{\mu}$ are to start with, some parameters (we think
of them as numbers to start with). Now if I multiply this equation
with itself (excluding the $\psi$ part), I'll have
\[
\begin{aligned} & (i\partial_{\mu}\gamma^{\mu}-m)(i\partial_{\nu}\gamma^{\nu}-m)\\
= & (-\partial_{\mu}\partial_{\nu}\gamma^{\mu}\gamma^{\nu}-im\partial_{\mu}\gamma^{\mu})-(im\partial_{\nu}\gamma^{\nu}-m^{2})\\
= & (-\frac{1}{2}\partial_{\mu}\partial_{\nu}\{\gamma^{\mu},\gamma^{\nu}\}-2m\left[i(\partial_{\mu}\gamma^{\mu})\right]+m^{2})\\
= & (-\frac{1}{2}\partial_{\mu}\partial_{\nu}\{\gamma^{\mu},\gamma^{\nu}\}-2m^{2}+m^{2})\\
= & (-\frac{1}{2}\partial_{\mu}\partial_{\nu}\{\gamma^{\mu},\gamma^{\nu}\}-m^{2})
\end{aligned}
\]
Now if I want this to become Klien Gordan, it's trivial to conclude
that $\{\gamma^{\mu},\gamma^{\nu}\}=-2\delta^{\mu\nu}$. Now without
getting into more detail, I'll just claim that the parameters $\gamma^{\mu}$
are chosen to be $4\times4$ matrices and $\psi$ then is a $4-$component
object, called a Dirac spinor. It is useful to work in a specific
basis (choice of $\gamma$) which for this discussion I'll take out
of a hat to be
\[
\begin{aligned}\sigma^{\mu}\equiv(1,\mathbf{\vec{\sigma})} &  & \overline{\sigma}^{\mu}\equiv(1,-\vec{\sigma})\\
\gamma^{\mu} & \equiv & \left(\begin{array}{cc}
0 & \sigma^{\mu}\\
\overline{\sigma}^{\mu} & 0
\end{array}\right)
\end{aligned}
\]
One can check that the anti-commutation is indeed what we wanted. 


\subsection{The Dirac Lagrangian}

$\psi$ has interesting transformation properties which I won't derive
but state as and when I need them. I claim that if I write my four
component $\psi$ as 
\[
\psi=\left(\begin{array}{c}
\psi_{L}\\
\psi_{R}
\end{array}\right)
\]
then under a lorentz boost $\mathbf{\beta}$ and a rotation $\mathbf{\theta}$,
\[
\begin{aligned}\psi_{L}\rightarrow & e^{-i\mathbf{\theta}.\frac{\mathbf{\sigma}}{2}-\mathbf{\beta}.\frac{\mathbf{\sigma}}{2}}\psi_{L}=S_{L}\psi_{L}\\
\psi_{R}\rightarrow & e^{-i\mathbf{\theta}.\frac{\mathbf{\sigma}}{2}+\mathbf{\beta}.\frac{\mathbf{\sigma}}{2}}\psi_{R}=S_{R}\psi_{R}
\end{aligned}
\]
which in a compact form, I'll write as
\[
\psi\rightarrow\Lambda_{1/2}\psi
\]


We need this to understand how must we write a Lagrangian for the
Dirac equation. We must first realize that the Lagrangian is a scalar.
We must understand therefore how to create scalars out of these objects
we call Dirac spinors. A first guess at writing a real Lagrangian
could be to write $\psi^{\dagger}\psi$. However, this object is not
Lorentz invariant. This is easy to see by simply noting that the transformation
of $\psi_{L}$ and $\psi_{R}$ is not unitary. I claim that 
\[
\psi^{\dagger}\gamma^{0}\rightarrow\psi^{\dagger}\gamma^{0}\Lambda_{1/2}^{-1}
\]
So now I simply define 
\[
\overline{\psi}\equiv\psi^{\dagger}\gamma^{0}
\]
so that
\[
\overline{\psi}\rightarrow\overline{\psi}\Lambda_{1/2}^{-1}
\]


Obviously now, $\overline{\psi}\psi$ is Lorentz invariant. I claim
further that $\overline{\psi}\gamma^{\mu}\psi$ transforms as a four-vector.
Armed with these, I can write the following Lorentz invariant quantity
\[
\mathcal{L}_{\text{Dirac}}=\overline{\psi}(i\gamma^{\mu}\partial_{\mu}-m)\psi
\]
I am allowed to call it the Dirac Lagrangian, if it's equation of
motion is the Dirac equation itself. This is indeed so; recall 
\[
\frac{\partial\mathcal{L}}{\partial(\partial_{\mu}\phi)}-\frac{\partial\mathcal{L}}{\partial\phi}=0
\]
which in this case, (treating $\psi$ and $\overline{\psi}$ as independent)
for $\overline{\psi}$ yeilds the Dirac equation in $\psi$.

Since this discussion is focussed on the mass term, let me expand
it for you. 
\[
m\overline{\psi}\psi=m\left(\begin{array}{cc}
\psi_{L}^{\dagger} & \psi_{R}^{\dagger}\end{array}\right)\left(\begin{array}{cc}
0 & 1\\
1 & 0
\end{array}\right)\left(\begin{array}{c}
\psi_{L}\\
\psi_{R}
\end{array}\right)=m(\psi_{R}^{\dagger}\psi_{L}+\psi_{L}^{\dagger}\psi_{R})
\]
The objective of this exercise was to show that in the Dirac equation,
the mass term mixes the left and right spinors. We develop some more
formalism and then see whether we can write a different Lorentz invariant
mass term that doesn't mix the left and right parts.


\subsection{Projectors, mixing of mass terms}

If from the full spinor, I wanted to obtain only the left handed part,
I could do the following 
\[
\left(\begin{array}{cc}
1 & 0\\
0 & 0
\end{array}\right)\left(\begin{array}{c}
\psi_{L}\\
\psi_{R}
\end{array}\right)=\left(\begin{array}{c}
\psi_{L}\\
0
\end{array}\right)
\]
As it turns out, if I define $\gamma^{5}\equiv i\gamma^{0}\gamma^{1}\gamma^{2}\gamma^{3}$,
then in our basis it becomes $\gamma^{5}=\left(\begin{array}{cc}
-1 & 0\\
0 & 1
\end{array}\right)$. Note then that the left hand projector can be written simply as
\[
P_{L}=\frac{1-\gamma^{5}}{2}
\]
Similarly, its almost immediate that 
\[
P_{R}=\frac{1+\gamma^{5}}{2}
\]
Further, in our basis, it's obvious from explicit matrices corresponding
to $P_{L}$ and $P_{R}$ that 
\[
P_{L}+P_{R}=1_{4\times4}
\]
and that 
\[
P_{L}P_{R}=0;\,P_{R}P_{L}=0
\]
I claim that $\{\gamma^{5},\gamma^{\mu}\}=0$ which allows me to show
that 
\[
\gamma^{\mu}P_{L}=\gamma^{\mu}\frac{\left(1-\gamma^{5}\right)}{2}=\frac{\left(1+\gamma^{5}\right)}{2}\gamma^{\mu}=P_{R}\gamma^{\mu}
\]


Finally I define
\[
\Psi_{L}\equiv P_{L}\psi=\left(\begin{array}{c}
\psi_{L}\\
0
\end{array}\right)
\]
and 
\[
\Psi_{R}\equiv P_{R}\psi=\left(\begin{array}{c}
0\\
\psi_{R}
\end{array}\right)
\]
This allows me to write 
\[
\psi=1_{4\times4}\psi=\left(P_{L}+P_{R}\right)\psi=\Psi_{L}+\Psi_{R}
\]
along with the hermitian conjugate 
\[
\psi^{\dagger}=\psi^{\dagger}1_{4\times4}=\psi^{\dagger}\left(P_{L}+P_{R}\right)=\Psi_{L}^{\dagger}+\Psi_{R}^{\dagger}
\]
plus for clarity 
\[
\overline{\psi}=\psi^{\dagger}\gamma^{0}=\psi^{\dagger}1_{4\times4}\gamma^{0}=\psi^{\dagger}\left(P_{L}+P_{R}\right)\gamma^{0}=\Psi_{L}^{\dagger}\gamma^{0}+\Psi_{R}^{\dagger}\gamma^{0}=\overline{\Psi}_{L}+\overline{\Psi}_{R}
\]
and that in turn allows me to expand the mass term 
\[
\begin{aligned}m\overline{\psi}\psi=m\left(\overline{\Psi}_{L}+\overline{\Psi}_{R}\right)\left(\Psi_{L}+\Psi_{R}\right) & = & m\left(\overline{\Psi}_{L}+\overline{\Psi}_{R}\right)\left(\Psi_{L}+\Psi_{R}\right)\\
 & = & m\left(\overline{\Psi}_{L}+\overline{\Psi}_{R}\right)\left(\Psi_{L}+\Psi_{R}\right)\\
 & = & m\left(\overline{\Psi}_{L}\Psi_{R}+\overline{\Psi}_{R}\Psi_{L}\right)
\end{aligned}
\]
It is easy to get confused between the equation we derived in the
previous subsection 
\[
m\overline{\psi}\psi=m(\psi_{R}^{\dagger}\psi_{L}+\psi_{L}^{\dagger}\psi_{R})
\]
but both are correct and equivalent. The latter is written in the
2-component form, while the former is in the 4-component notation.
And while I'm still discussing notation, let me define for a four
vector $B^{\mu}$ 
\[
\cancel{B}\equiv\gamma^{\mu}B_{\mu}
\]


So in this notation, I can write the Lagrangian as 
\[
\begin{aligned}\mathcal{L}_{\text{Dirac}} & = & \overline{\psi}(i\gamma^{\mu}\partial_{\mu}-m)\psi\\
 & = & \left(\overline{\Psi}_{L}+\overline{\Psi}_{R}\right)\left(i\cancel{\partial}-m\right)\left(\Psi_{L}+\Psi_{R}\right)\\
 & = & \overline{\Psi}_{L}i\cancel{\partial}\Psi_{L}+\overline{\Psi}_{R}i\cancel{\partial}\Psi_{R}-m\left(\overline{\Psi}_{L}\Psi_{R}+\overline{\Psi}_{R}\Psi_{L}\right)
\end{aligned}
\]
Note how the kinetic term has the left and right handed spinor separated,
but the mass term is mixing the spinor components. We now have enough
machinery to explore mass terms that don't mix the spinor components.


\subsection{Majorana Mass}

I start with defining the following matrix 
\[
C=i\left(\begin{array}{cc}
\sigma^{2} & 0\\
0 & -\sigma^{2}
\end{array}\right)
\]
Now I go back to the transformations of the Dirac spinor. If I write
$\psi^{T}$ as 
\[
\psi^{T}=\left(\begin{array}{cc}
\psi_{L}^{T} & \psi_{R}^{T}\end{array}\right)
\]
then under a lorentz boost $\mathbf{\beta}$ and a rotation $\mathbf{\theta}$,
\[
\begin{aligned}\psi_{L}^{T}\rightarrow & \psi_{L}^{T}e^{-i\mathbf{\theta}.\frac{\mathbf{\sigma^{T}}}{2}-\mathbf{\beta}.\frac{\mathbf{\sigma^{T}}}{2}}=\psi_{L}^{T}S_{L}^{T}\\
\psi_{R}^{T}\rightarrow & \psi_{R}^{T}e^{-i\mathbf{\theta}.\frac{\mathbf{\sigma}^{T}}{2}+\mathbf{\beta}.\frac{\mathbf{\sigma}^{T}}{2}}=\psi_{R}^{T}S_{R}^{T}
\end{aligned}
\]
which in a compact form, according to prior definitions is
\[
\psi^{T}\rightarrow\psi^{T}\Lambda_{1/2}^{T}
\]
Noting from the explicit Pauli matrices, 
\[
\begin{aligned}\sigma^{1} & = & \left(\begin{array}{cc}
0 & 1\\
1 & 0
\end{array}\right)\\
\sigma^{2} & = & \left(\begin{array}{cc}
0 & -i\\
i & 0
\end{array}\right)\\
\sigma^{3} & = & \left(\begin{array}{cc}
1 & 0\\
0 & -1
\end{array}\right)
\end{aligned}
\]
we see that 
\[
\begin{alignedat}{2}\left(\sigma^{1}\right)^{T} & = & \sigma^{1}\\
\left(\sigma^{2}\right)^{T} & = & -\sigma^{2}\\
\left(\sigma^{3}\right)^{T} & = & \sigma^{3}
\end{alignedat}
\]
Further, we can use the fact that $\{\sigma^{i},\sigma^{k}\}=2\delta^{ik}$
to see that 
\[
\begin{alignedat}{2}\left(\sigma^{1}\right)^{T}\sigma^{2} & = & -\sigma^{2}\sigma^{1}\\
\left(\sigma^{2}\right)^{T}\sigma^{2} & = & -\sigma^{2}\sigma^{2}\\
\left(\sigma^{3}\right)^{T}\sigma^{2} & = & -\sigma^{2}\sigma^{3}
\end{alignedat}
\]
and this is fantastic, why? Well, consider the object $\psi_{L}^{T}\sigma^{2}$.
We'll have 
\[
\psi_{L}^{T}\sigma^{2}\rightarrow\psi_{L}^{T}e^{-i\mathbf{\theta}.\frac{\mathbf{\sigma^{T}}}{2}-\mathbf{\beta}.\frac{\mathbf{\sigma^{T}}}{2}}\sigma^{2}=\psi_{L}^{T}\sigma^{2}e^{-\left(-i\mathbf{\theta}.\frac{\mathbf{\sigma}}{2}-\mathbf{\beta}.\frac{\mathbf{\sigma}}{2}\right)}=\psi_{L}^{T}\sigma^{2}S_{L}^{-1}
\]
and by the same reasoning, we have 
\[
\psi_{R}^{T}\sigma^{2}\rightarrow\psi_{R}^{T}e^{-i\mathbf{\theta}.\frac{\mathbf{\sigma}^{T}}{2}+\mathbf{\beta}.\frac{\mathbf{\sigma}^{T}}{2}}\sigma^{2}=\psi_{R}^{T}\sigma^{2}e^{-\left(-i\mathbf{\theta}.\frac{\mathbf{\sigma}}{2}+\mathbf{\beta}.\frac{\mathbf{\sigma}}{2}\right)}=\psi_{R}^{T}\sigma^{2}S_{R}^{-1}
\]
We now want to write this in the compact notation. For this, we note
that 
\[
\psi^{T}C=i\left(\begin{array}{cc}
\psi_{L}^{T} & \psi_{R}^{T}\end{array}\right)\left(\begin{array}{cc}
\sigma^{2} & 0\\
0 & -\sigma^{2}
\end{array}\right)=i\left(\begin{array}{cc}
\psi_{L}^{T}\sigma^{2} & -\psi_{R}^{T}\sigma^{2}\end{array}\right)
\]
so that the transformation is 
\[
\psi^{T}C\rightarrow i\left(\begin{array}{cc}
\psi_{L}^{T}S_{L}^{T}\sigma^{2} & -\psi_{R}^{T}S_{R}^{T}\sigma^{2}\end{array}\right)=i\left(\begin{array}{cc}
\psi_{L}^{T}\sigma^{2}S_{L}^{-1} & -\psi_{R}^{T}\sigma^{2}S_{R}^{-1}\end{array}\right)=\psi^{T}C\lambda_{1/2}^{-1}
\]
And what will all of this do? Well, it means that 
\[
\psi^{T}C\psi\rightarrow\psi^{T}C\lambda_{1/2}^{-1}\lambda_{1/2}\psi=\psi^{T}C\psi
\]
so that means we have arrived at a Lorentz scalar! Before proceeding,
I'll make two comments. 
\begin{enumerate}
\item We could've arrived at the same result by simply noting that $\lambda_{1/2}^{T}C\lambda_{1/2}=C$
\item I can write $C$ as a product of $\gamma$ matrices as 
\[
C=i\left(\begin{array}{cc}
\sigma^{2} & 0\\
0 & -\sigma^{2}
\end{array}\right)=-i\gamma^{2}\gamma^{0}
\]
which is easy to verify.
\end{enumerate}
So let us use this term finally to define a different mass term. I
write 
\[
\begin{alignedat}{2}\mathcal{L}_{\text{Majorana Mass}} & \sim & m\psi^{T}C\psi\\
 & = & -im\left(\Psi_{L}^{T}+\Psi_{R}^{T}\right)\gamma^{2}\gamma^{0}\left(\Psi_{L}+\Psi_{R}\right)\\
 & = & -im\left(\Psi_{L}^{T}\gamma^{2}\gamma^{0}\Psi_{L}+\Psi_{R}^{T}\gamma^{2}\gamma^{0}\Psi_{R}\right)\\
 & = & m\left(\Psi_{L}^{T}C\Psi_{L}+\Psi_{R}^{T}C\Psi_{R}\right)
\end{alignedat}
\]
so here I have arrived at a mass term which does \emph{mix }the left
and right spinors! I'd point out that this object ofcourse is not
guaranteed to be real. To ensure that, we must add its hermitian conjugate
($-m\psi^{\dagger}C\psi^{*}$). To conclude this subsection, I'd mention
that 
\begin{itemize}
\item there are what's called a majorana fermion, which is such that $\psi$
equals it's own `conjugate'. This results in reduction of degrees
of freedom from 4 to 2
\item $C$ is closely related to the charge conjugation operator
\end{itemize}
and state some results
\begin{itemize}
\item The Lagrangian with the `kinetic part' is
\[
\begin{alignedat}{2}\mathcal{L}_{\text{Majorana}} & = & \psi_{L}^{\dagger}i\overline{\sigma}.\partial\psi_{L}+\frac{im}{2}\left(\psi_{L}^{T}\sigma^{2}\psi_{L}-\psi_{L}^{\dagger}\sigma^{2}\psi_{L}^{*}\right)\\
 & = & i\Psi_{L}^{\dagger}\cancel{\partial}\Psi_{L}-\frac{m}{2}\left(\Psi_{L}^{T}C\Psi_{L}+\Psi_{L}^{\dagger}C\Psi_{L}^{*}\right)
\end{alignedat}
\]

\item Corresponding Euler Lagrange
\[
\begin{aligned}i\overline{\sigma}.\partial\psi_{L}-im\sigma^{2}\psi_{L}^{*} & =0\end{aligned}
\]
which implies that the Klien Gordan is satisfied and since it is derived
from a Lorentz invariant lagrangian, it too is Lorentz invariant.
\end{itemize}

\section{Physical Relevance}

In the physical world, we don't observe `right handed'\footnote{in the sense we have just discussed}
neutrinos. Even the `left handed' neutrino was thought to be massless
for a long period of time. The standard model consequently has been
created to have the following term
\[
\mathcal{L}=\overline{\nu}_{L}i\cancel{\partial}\nu_{L}
\]
However, lately experiments have shown that the `left handed' neutrinos
also have mass. To implement them, we first of all need a way of handling
the fact that there's just one part of the full spinor, viz. only
2 degrees of freedom. Plus the Dirac mass as we know it, necessarily
mixes the left and right handed part. In this regard, atleast in some
sense, the Majorana mass and Majorana spinor are possible candidates
to describe neutrinos.

For completeness, I must mention that there are other alternatives
to describe the mass of neutrinos, such as the 'see-saw' model.
\end{document}
