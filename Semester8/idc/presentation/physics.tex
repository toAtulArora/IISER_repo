%%%%%%%%%%%%%%%%%%%%%%%%%%%%%%%%%%%%%%%%%
% Beamer Presentation
% LaTeX Template
% Version 1.0 (10/11/12)
%
% This template has been downloaded from:
% http://www.LaTeXTemplates.com
%
% License:
% CC BY-NC-SA 3.0 (http://creativecommons.org/licenses/by-nc-sa/3.0/)
%
%%%%%%%%%%%%%%%%%%%%%%%%%%%%%%%%%%%%%%%%%

%----------------------------------------------------------------------------------------
%	PACKAGES AND THEMES
%----------------------------------------------------------------------------------------

\documentclass{beamer}

\mode<presentation> {

% The Beamer class comes with a number of default slide themes
% which change the colors and layouts of slides. Below this is a list
% of all the themes, uncomment each in turn to see what they look like.

%\usetheme{default}
%\usetheme{AnnArbor}
%\usetheme{Antibes} %useless type of index (not really an index) | wasteful
%\usetheme{Bergen}
%\usetheme{Berkeley}
%\usetheme{Berlin}
%\usetheme{Boadilla} %elegant, loads of space, no index
%\usetheme{CambridgeUS} %standard with no index
%\usetheme{Copenhagen} %index but wasteful
%\usetheme{Darmstadt}
%\usetheme{Dresden}
%\usetheme{Frankfurt} %index on top
%\usetheme{Goettingen}
%\usetheme{Hannover}
%\usetheme{Ilmenau} %index, not too wasteful
%\usetheme{JuanLesPins} like antibes
%\usetheme{Luebeck}
%\usetheme{Madrid} %elegent, no index
%\usetheme{Malmoe} %index, wasetful
%\usetheme{Marburg} %index side
%\usetheme{Montpellier} %elegant, useless type index
%\usetheme{PaloAlto} %index on the left
%\usetheme{Pittsburgh} %no index
%\usetheme{Rochester} %simple but nice
\usetheme{Singapore}
%\usetheme{Szeged}
%\usetheme{Warsaw}

% As well as themes, the Beamer class has a number of color themes
% for any slide theme. Uncomment each of these in turn to see how it
% changes the colors of your current slide theme.

%\usecolortheme{albatross}
%\usecolortheme{beaver}
%\usecolortheme{beetle}
%\usecolortheme{crane}
%\usecolortheme{dolphin}
%\usecolortheme{dove}
%\usecolortheme{fly}
%\usecolortheme{lily}
%\usecolortheme{orchid}
%\usecolortheme{rose}
%\usecolortheme{seagull}
%\usecolortheme{seahorse}
%\usecolortheme{whale}
%\usecolortheme{wolverine}

%\setbeamertemplate{footline} % To remove the footer line in all slides uncomment this line
%\setbeamertemplate{footline}[page number] % To replace the footer line in all slides with a simple slide count uncomment this line

%\setbeamertemplate{navigation symbols}{} % To remove the navigation symbols from the bottom of all slides uncomment this line
}

\usepackage{graphicx} % Allows including images
\usepackage{booktabs} % Allows the use of \toprule, \midrule and \bottomrule in tables
\usepackage{braket}
\usepackage{amsmath}
\usepackage{amssymb}
\usepackage{cancel}
\usepackage{ccfonts}
\usepackage[T1]{fontenc}
\usepackage[latin9]{inputenc}
% \usepackage{geometry}
% \onehalfspacing
%----------------------------------------------------------------------------------------
%	TITLE PAGE
%----------------------------------------------------------------------------------------

\title[Dirac and Majorana Mass]{Dirac and Majorana Mass} % The short title appears at the bottom of every slide, the full title is only on the title page

\author{Atul Singh Arora} % Your name
\institute[IISER M] % Your institution as it will appear on the bottom of every slide, may be shorthand to save space
{
Indian Institute of Science Education and Research Mohali \\ % Your institution for the title page
\medskip
% \textit{ms10024@iisermohali.ac.in} \\
% \textit{ms11003@iisermohali.ac.in} % Your email address
}
\date{\today} % Date, can be changed to a custom date

\begin{document}

\begin{frame}
\titlepage % Print the title page as the first slide
\end{frame}

\section{Outline}
\begin{frame}
\frametitle{Overview of the Talk} % Table of contents slide, comment this block out to remove it
\tableofcontents % Throughout your presentation, if you choose to use \section{} and \subsection{} commands, these will automatically be printed on this slide as an overview of your presentation
\end{frame}

\section{Introduction}
\begin{frame}	
	\Huge{\centerline{Introduction}}
	% \tiny{\centerline{first demonstrated to us by Prof. Arvind}}
\end{frame}


\begin{frame}
	\frametitle{Motivation | Mass}
		\begin{itemize}
			\item Inertia | Newton
			\pause
			\item Special Relativity | $p^{2}=m^{2}$
			\pause
			\item Particle Physics
			\begin{itemize}
				\item QED (Quantum Electrodynamics)
				\pause
				\begin{itemize}
					\item Classical Electrodynamics gauge field: $A^{\mu}$
					\pause
					\item Quantize | Massless Gauge Field
					\pause
				\end{itemize}
				\item Standard Model
				\begin{itemize}
					\item Massive Gauge fields | Spontaneous Symmetry Breaking
					\pause
					\item Scalar field | Higgs
					\pause
				\end{itemize}
				\item Beyond Standard Model
				\begin{itemize}
					\item Neutrino Oscillations
					\pause
					\item Neutrino Mass
					\pause
				\end{itemize}				
			\end{itemize}
		\end{itemize}
\end{frame}

\section{Prerequisites}
\begin{frame}	
	\Huge{\centerline{Prerequisites}}
	% \tiny{\centerline{first demonstrated to us by Prof. Arvind}}
\end{frame}

\begin{frame}
	\frametitle{Prerequisites}
		\begin{itemize}
			\item From CM, recall
			\begin{itemize}
				\item Lagrangian Formalism
				\pause
				\item Euler Lagrange equations
				\pause
				\item Noether's Theorem relating conserved quantities and continuous symmetries
				\pause
			\end{itemize}
			\item From STR, I'll use
			\begin{itemize}
				\item $\eta^{\mu\nu}=\text{diag}(1,-\vec{1})$ (NB: $\eta^{T}=\eta^{-1}=\eta$)
				\pause
				\item $c=1, \hbar=1$
				\pause
				\item indices
				\begin{itemize}
					\item $i,j,k,l$ etc. run from 1 to 3
					\pause
					\item $\alpha,\beta,\gamma$ etc. run from 0 to 3
					\pause
				\end{itemize}
			\end{itemize}

		\end{itemize}
\end{frame}


\begin{frame}
	\frametitle{Prerequisites}
		\begin{itemize}
			\item I'll need the 4 vector notation. Recall
			\begin{itemize}
				\item Summation Convention $A^{\alpha}B_{\alpha}=\sum_{\alpha=0}^{4}A^{\alpha}B_{\alpha}$
				\pause
				\item if $A^{\alpha}=(A^{0},\vec{A})$, then $A_{\alpha}\equiv\eta_{\alpha\beta}A^{\beta}=(A^{0},-\vec{A})$
				\pause
				\item $\lambda_{\beta}^{\alpha}$, $A^{\alpha}\to A'^{\alpha}=\lambda_{\beta}^{\alpha}A^{\beta}$
				\pause
				\item contracted indices don't transform (NB: $\lambda^{T}\eta\lambda=\eta$)
				\pause
			\end{itemize}
			\item From QM, I'll need the following. Recall
			\begin{itemize}
				\item State: $\ket{\psi}$ (or $\psi(x)=\left\langle x|\psi\right\rangle$)
				\pause
				\item Time Evolution: For $H$ (st. $H^{\dagger}=H$; where $H^{\dagger}\equiv H^{*T}$) we have 
				\[H\left|\psi\right\rangle =-i\hbar\frac{\partial}{\partial t}\left|\psi\right\rangle\] 
				\pause
				and
				\[\left|\psi(t)\right\rangle =e^{(-i\hbar)^{-1}Ht}\left|\psi(0)\right\rangle \]
				NB: $U\equiv e^{(-i\hbar)^{-1}Ht}$ is unitary, viz. $U^{\dagger}=U^{-1}$
				\pause
				\item Measurement/Observables
				\begin{itemize}
					\item Collapse into eigenstate of operator corresponding to the measurement
					\pause
					\item Collapse to state $\ket{n}$ with probability $\left|\left\langle n|\psi\right\rangle \right|^{2}$
					\pause
				\end{itemize}
				\item Basics of quantum harmonic oscillator using $a$ $a^{\dagger}$
			\end{itemize}
		\end{itemize}
\end{frame}

\begin{frame}
	\frametitle{Prerequisites}
		\begin{itemize}
			\item I'll use the following pauli matrices
			\pause
			$\begin{aligned}\sigma^{1} & = & \left(\begin{array}{cc}
			0 & 1\\
			1 & 0
			\end{array}\right)\\
			\sigma^{2} & = & \left(\begin{array}{cc}
			0 & -i\\
			i & 0
			\end{array}\right)\\
			\sigma^{3} & = & \left(\begin{array}{cc}
			1 & 0\\
			0 & -1
			\end{array}\right)
			\end{aligned}
			 $
			 \pause
 			\item Terminology from particle physics
 			\pause
 			\begin{itemize}
 				\item Leptons: Eg. Electron, Electron Neutrino \pause (beta decay, massless, rather inert)
 				\pause
 				\item Quarks
 			\end{itemize}

		\end{itemize}
\end{frame}

\section{Towards a quantum theory of fields}
\begin{frame}	
	\Huge{\centerline{Towards a}}
	\Huge{\centerline{quantum theory of fields}}
	% \tiny{\centerline{first demonstrated to us by Prof. Arvind}}
\end{frame}

\begin{frame}
	\frametitle{Motivation}
		\begin{itemize}	
			\item All electrons are identical
			\pause
			\item Unification of QM and STR
			\pause
			\item Crisis: Can't predict the result of collision of particles
		\end{itemize}
		\pause
		Targets of the new theory
		\begin{itemize}
			\item Creation and destruction
			\pause
			\item Consistent with STR (high energy)
			\pause
			\item Predict probablities			
		\end{itemize}
\end{frame}

\begin{frame}
	\frametitle{Known Issues}
		\begin{itemize}
			\item $\left(E^{2}-\vec{p}^{2}\right)\psi=m^{2}\psi$ \pause 

			and put $E\to-i\frac{\partial}{\partial t}, \vec{p}\rightarrow i\vec{\nabla}$ to get

			\pause
\[
( \frac{\partial}{\partial t}^{2}-\vec{\nabla}^{2} ) \psi=m^{2}\psi
\]
			\pause
			\item Causality
			\pause
			\item Negative Energies (no stable ground state)
			\pause
			\item Expected: t parameter, $\vec{x}$ operator
		\end{itemize}
\end{frame}

\begin{frame}
	\frametitle{Concept of field}
		\begin{itemize}
			\item One field for each type of particle \pause (Wheeler's idea)
			\pause
			\item creates and destroys particles
			\pause
			\item Interacting fields, interacting particles
			\pause
		\end{itemize}
\end{frame}

\begin{frame}
	\frametitle{Framework: QFT}
		\begin{itemize}
			\item Classical field | real scalar (number at every space time point)
			\pause
			\item Demand Klien Gordan, then 
			\[
				\mathcal{L}=\frac{1}{2}\left(\partial^{\mu}\phi\partial_{\mu}\phi+m^{2}\phi\right)
			\]
			\pause
			\item $\phi=\phi(t,\vec{x})$ which I assume I can write as
			\pause 
			\[
			\phi=\int\frac{d^{3}p}{\left(2\pi\right)^{3}\sqrt{2\omega_{p}}}\,\left(ae^{i\mathbf{px}}+a^{\dagger}e^{-i\mathbf{px}}\right)
			\]
			where $a=a(\vec{p})$
			\pause
			\item $\pi$ from $\mathcal{L}$.
			\pause
			\item Quantum Field | $[\phi(t,\mathbf{x}),\pi(t,\mathbf{x}')]=i\delta(\mathbf{x}-\mathbf{x}')$			
		\end{itemize}
\end{frame}

\begin{frame}
	\frametitle{Framework: QFT}
		\begin{itemize}
			\item $[a(\mathbf{p}),a^{\dagger}(\mathbf{p}')]\sim\delta(\mathbf{p}-\mathbf{p}')$
			\pause
			\item \[
				H\sim a^{\dagger}a+\frac{1}{2}[a(\mathbf{p}),a^{\dagger}(\mathbf{p})]
				\]
			\pause
			\item Similarity with Quantum Harmonic Oscillator
			\pause
			\item $a^{\dagger}(\vec{p})\left|\text{vacuum}\right\rangle $
			\pause
			\item Noether's theorem + Space-time invariance of $\mathcal{L} \rightarrow$ physical momentum and energy operators
			\pause
			\item To be a particle, it must satisfy $E^{2}-\vec{p}^{2}=m^{2}$
			\pause
			and it does
			\pause
			\item Conclusion: Parameter $m$ is mass
		\end{itemize}
\end{frame}


\begin{frame}
	\frametitle{Framework: QFT}
		\begin{itemize}
			\item Non-interacting field
			\pause
			\item Observable fields must interact
			\pause
			\item QFT | perturbation theory, expanded around the non-interacting part
			\pause
			\item results in Feynman Rules \pause (Prof. Mukunda story)
			\pause
			\begin{itemize}
				\item Decay Rates
				\pause
				\item Scattering Cross sections
				\pause
			\end{itemize}
			\item Interacting case, $m$ no longer the mass
			\pause
			\begin{itemize}
				\item defined as pole of the `full propogator' \pause (I'll leave it at that)
			\end{itemize}
		\end{itemize}
\end{frame}

\section{Dirac and Majorana Mass}
\begin{frame}	
	\Huge{\centerline{Dirac and Majorana Mass}}
	% \tiny{\centerline{first demonstrated to us by Prof. Arvind}}
\end{frame}

\begin{frame}
	\frametitle{The Dirac Equation}
		\begin{itemize}
		\item 
		\[
		(i\partial_{\mu}\gamma^{\mu}-m)\psi=0
		\]

		\pause
		\item
		\[
		\begin{aligned} & (i\partial_{\mu}\gamma^{\mu}-m)(i\partial_{\nu}\gamma^{\nu}-m)\\
		= & (-\partial_{\mu}\partial_{\nu}\gamma^{\mu}\gamma^{\nu}-im\partial_{\mu}\gamma^{\mu})-(im\partial_{\nu}\gamma^{\nu}-m^{2})\\
		= & (-\frac{1}{2}\partial_{\mu}\partial_{\nu}\{\gamma^{\mu},\gamma^{\nu}\}-2m\left[i(\partial_{\mu}\gamma^{\mu})\right]+m^{2})\\
		= & (-\frac{1}{2}\partial_{\mu}\partial_{\nu}\{\gamma^{\mu},\gamma^{\nu}\}-2m^{2}+m^{2})\\
		= & (-\frac{1}{2}\partial_{\mu}\partial_{\nu}\{\gamma^{\mu},\gamma^{\nu}\}-m^{2})
		\end{aligned}
		\]


		\pause 
		\item 

		To be Klein Gordan, $\{\gamma^{\mu},\gamma^{\nu}\}=-2\delta^{\mu\nu}$

		\pause
		\item Claim

		$\gamma^{\mu}$ are $4\times4$ matrices and $\psi$ then is a $4-$component object, called a Dirac spinor.

	\end{itemize}
\end{frame}

\begin{frame}
	\frametitle{The Dirac Equation}
\begin{itemize}
	\item 
	Hat:
	\[
	\begin{aligned}\sigma^{\mu}\equiv(1,\vec{\sigma}) &  & \overline{\sigma}^{\mu}\equiv(1,-\vec{\sigma})\\
	\gamma^{\mu} & \equiv & \left(\begin{array}{cc}
	0 & \sigma^{\mu}\\
	\overline{\sigma}^{\mu} & 0
	\end{array}\right)
	\end{aligned}
	\]
	\pause
	\item Claim: commutation holds
\end{itemize}

\end{frame}

\begin{frame}
	\frametitle{The Dirac Lagrangian}
	\begin{itemize}
		\item Claim: 

\[
\psi=\left(\begin{array}{c}
\psi_{L}\\
\psi_{R}
\end{array}\right)
\]
under a lorentz boost $\mathbf{\beta}$ and a rotation $\mathbf{\theta}$
\pause
\[
\begin{aligned}\psi_{L}\rightarrow & e^{-i\mathbf{\theta}.\frac{\mathbf{\sigma}}{2}-\mathbf{\beta}.\frac{\mathbf{\sigma}}{2}}\psi_{L}=S_{L}\psi_{L}\\
\psi_{R}\rightarrow & e^{-i\mathbf{\theta}.\frac{\mathbf{\sigma}}{2}+\mathbf{\beta}.\frac{\mathbf{\sigma}}{2}}\psi_{R}=S_{R}\psi_{R}
\end{aligned}
\]
\pause
which in a compact form, I'll write as
\[
\psi\rightarrow\Lambda_{1/2}\psi
\]

		\item $\psi^{\dagger}\psi$ \pause | not Lorentz invariant
		\pause
		\item $\psi_{L}$ and $\psi_{R}$ is not unitary
	\end{itemize}
\end{frame}


\begin{frame}
	\frametitle{The Dirac Lagrangian}
	\begin{itemize}
		\item Claim: 
\[
\psi^{\dagger}\gamma^{0}\rightarrow\psi^{\dagger}\gamma^{0}\Lambda_{1/2}^{-1}
\]
\pause
		\item I define		
\[
\overline{\psi}\equiv\psi^{\dagger}\gamma^{0}
\]
so that
\pause
\[
\overline{\psi}\rightarrow\overline{\psi}\Lambda_{1/2}^{-1}
\]
\pause
		\item $\overline{\psi}\psi$ is Lorentz invariant
		\pause
		\item Claim: $\overline{\psi}\gamma^{\mu}\psi$ transforms as a four-vector.
		\pause
		\item 
\[
\mathcal{L}_{\text{Dirac}}=\overline{\psi}(i\gamma^{\mu}\partial_{\mu}-m)\psi
\]
		\pause
		is therefore Lorentz invariant

	\end{itemize}
\end{frame}


\begin{frame}
	\frametitle{The Dirac Lagrangian}
	\begin{itemize}

		\item Recall:
\[
\frac{\partial\mathcal{L}}{\partial(\partial_{\mu}\phi)}-\frac{\partial\mathcal{L}}{\partial\phi}=0
\]
		\pause 
		(treating $\psi$ and $\overline{\psi}$ as independent) for $\overline{\psi}$ yeilds the Dirac equation in $\psi$.

		\item 
\[
m\overline{\psi}\psi=m\left(\begin{array}{cc}
\psi_{L}^{\dagger} & \psi_{R}^{\dagger}\end{array}\right)\left(\begin{array}{cc}
0 & 1\\
1 & 0
\end{array}\right)\left(\begin{array}{c}
\psi_{L}\\
\psi_{R}
\end{array}\right)=m(\psi_{R}^{\dagger}\psi_{L}+\psi_{L}^{\dagger}\psi_{R})
\]
\pause
		\item Mass term mixes the left and right spinors
		\item Explore: mass term that doesn't mix
	\end{itemize}
\end{frame}



\begin{frame}
	\frametitle{Projectors, mixing of mass terms}
	\begin{itemize}
		\item 
		\pause
\[
\left(\begin{array}{cc}
1 & 0\\
0 & 0
\end{array}\right)\left(\begin{array}{c}
\psi_{L}\\
\psi_{R}
\end{array}\right)=\left(\begin{array}{c}
\psi_{L}\\
0
\end{array}\right)
\]
		\item As it turns out, if I define $\gamma^{5}\equiv i\gamma^{0}\gamma^{1}\gamma^{2}\gamma^{3}$, \pause
then in our basis 
\[\gamma^{5}=\left(\begin{array}{cc}
-1 & 0\\
0 & 1
\end{array}\right)\]
\pause
		\item 
		\pause
\[
P_{L}=\frac{1-\gamma^{5}}{2}
\]
Similarly,
\[
P_{R}=\frac{1+\gamma^{5}}{2}
\]

		\item Obvious from matrix multiplication and definitions of
		\pause
$P_{L}$ and $P_{R}$ that 
\pause
\[
P_{L}+P_{R}=1_{4\times4}
\]

	\end{itemize}
\end{frame}






\begin{frame}
	\frametitle{Projectors, mixing of mass terms}
	\begin{itemize}



\item
\pause
and that 
\[
P_{L}P_{R}=0;\,P_{R}P_{L}=0
\]
\pause
		\item Claim: $\{\gamma^{5},\gamma^{\mu}\}=0$
		\pause
		\item It then follows that 
		\pause
\[
\gamma^{\mu}P_{L}=\gamma^{\mu}\frac{\left(1-\gamma^{5}\right)}{2}=\frac{\left(1+\gamma^{5}\right)}{2}\gamma^{\mu}=P_{R}\gamma^{\mu}
\]
		\pause
		\item I define 
\[
\Psi_{L}\equiv P_{L}\psi=\left(\begin{array}{c}
\psi_{L}\\
0
\end{array}\right)
\]
\pause
and 
\[
\Psi_{R}\equiv P_{R}\psi=\left(\begin{array}{c}
0\\
\psi_{R}
\end{array}\right)
\]

	\end{itemize}
\end{frame}






\begin{frame}
	\frametitle{Projectors, mixing of mass terms}
	\begin{itemize}
		\item
This allows me to write 
\pause
\[
\psi=1_{4\times4}\psi=\left(P_{L}+P_{R}\right)\psi=\Psi_{L}+\Psi_{R}
\]
\pause
along with the hermitian conjugate 
\[
\psi^{\dagger}=\psi^{\dagger}1_{4\times4}=\psi^{\dagger}\left(P_{L}+P_{R}\right)=\Psi_{L}^{\dagger}+\Psi_{R}^{\dagger}
\]

		\item Clarity:
		\pause
\[
\overline{\psi}=\psi^{\dagger}\gamma^{0}=\psi^{\dagger}1_{4\times4}\gamma^{0}=\psi^{\dagger}\left(P_{L}+P_{R}\right)\gamma^{0}=\Psi_{L}^{\dagger}\gamma^{0}+\Psi_{R}^{\dagger}\gamma^{0}=\overline{\Psi}_{L}+\overline{\Psi}_{R}
\]
		\item and that in turn allows me to expand the mass term 
		\pause
\[
\begin{aligned}m\overline{\psi}\psi=m\left(\overline{\Psi}_{L}+\overline{\Psi}_{R}\right)\left(\Psi_{L}+\Psi_{R}\right) & = & m\left(\overline{\Psi}_{L}+\overline{\Psi}_{R}\right)\left(\Psi_{L}+\Psi_{R}\right)\\
 & = & m\left(\overline{\Psi}_{L}+\overline{\Psi}_{R}\right)\left(\Psi_{L}+\Psi_{R}\right)\\
 & = & m\left(\overline{\Psi}_{L}\Psi_{R}+\overline{\Psi}_{R}\Psi_{L}\right)
\end{aligned}
\]


	\end{itemize}
\end{frame}






\begin{frame}
	\frametitle{Projectors, mixing of mass terms}
	\begin{itemize}
		\item Recall: 
\[
m\overline{\psi}\psi=m(\psi_{R}^{\dagger}\psi_{L}+\psi_{L}^{\dagger}\psi_{R})
\]
\pause 
both are correct and equivalent
\pause 
		\item Defn: For a four vector $B^{\mu}$ 
		\pause
\[
\cancel{B}\equiv\gamma^{\mu}B_{\mu}
\]
\pause
		\item Result:
		\pause
\[
\begin{aligned}\mathcal{L}_{\text{Dirac}} & = & \overline{\psi}(i\gamma^{\mu}\partial_{\mu}-m)\psi\\
 & = & \left(\overline{\Psi}_{L}+\overline{\Psi}_{R}\right)\left(i\cancel{\partial}-m\right)\left(\Psi_{L}+\Psi_{R}\right)\\
 & = & \overline{\Psi}_{L}i\cancel{\partial}\Psi_{L}+\overline{\Psi}_{R}i\cancel{\partial}\Psi_{R}-m\left(\overline{\Psi}_{L}\Psi_{R}+\overline{\Psi}_{R}\Psi_{L}\right)
\end{aligned}
\]
\pause
		\item In this notationa also, there's mixing




	\end{itemize}
\end{frame}

 










\begin{frame}
	\frametitle{Majorana Mass}
	\begin{itemize}
		\item Defn:
\[
C=i\left(\begin{array}{cc}
\sigma^{2} & 0\\
0 & -\sigma^{2}
\end{array}\right)
\]
\pause
		\item Recall and process: For	
\[
\psi^{T}=\left(\begin{array}{cc}
\psi_{L}^{T} & \psi_{R}^{T}\end{array}\right)
\]
\pause
under a lorentz boost $\mathbf{\beta}$ and a rotation $\mathbf{\theta}$,
\pause
\[
\begin{aligned}\psi_{L}^{T}\rightarrow & \psi_{L}^{T}e^{-i\mathbf{\theta}.\frac{\mathbf{\sigma^{T}}}{2}-\mathbf{\beta}.\frac{\mathbf{\sigma^{T}}}{2}}=\psi_{L}^{T}S_{L}^{T}\\
\psi_{R}^{T}\rightarrow & \psi_{R}^{T}e^{-i\mathbf{\theta}.\frac{\mathbf{\sigma}^{T}}{2}+\mathbf{\beta}.\frac{\mathbf{\sigma}^{T}}{2}}=\psi_{R}^{T}S_{R}^{T}
\end{aligned}
\]
\pause
which in a compact form is
\[
\psi^{T}\rightarrow\psi^{T}\Lambda_{1/2}^{T}
\]

	\end{itemize}
\end{frame}

 









 
\begin{frame}
	\frametitle{Majorana Mass}
		\begin{itemize}
		\item NB: from
\[
\begin{aligned}\sigma^{1} & = & \left(\begin{array}{cc}
0 & 1\\
1 & 0
\end{array}\right)\\
\sigma^{2} & = & \left(\begin{array}{cc}
0 & -i\\
i & 0
\end{array}\right)\\
\sigma^{3} & = & \left(\begin{array}{cc}
1 & 0\\
0 & -1
\end{array}\right)
\end{aligned}
\]
\pause
it's obvious that
\pause
\[
\begin{alignedat}{2}\left(\sigma^{1}\right)^{T} & = & \sigma^{1}\\
\left(\sigma^{2}\right)^{T} & = & -\sigma^{2}\\
\left(\sigma^{3}\right)^{T} & = & \sigma^{3}
\end{alignedat}
\]

	\end{itemize}
\end{frame}

 









 
\begin{frame}
	\frametitle{Majorana Mass}
		\begin{itemize}



		\item consider the object $\psi_{L}^{T}\sigma^{2}$	
		\pause
\[
\psi_{L}^{T}\sigma^{2}\rightarrow\psi_{L}^{T}e^{-i\mathbf{\theta}.\frac{\mathbf{\sigma^{T}}}{2}-\mathbf{\beta}.\frac{\mathbf{\sigma^{T}}}{2}}\sigma^{2}=\psi_{L}^{T}\sigma^{2}e^{-\left(-i\mathbf{\theta}.\frac{\mathbf{\sigma}}{2}-\mathbf{\beta}.\frac{\mathbf{\sigma}}{2}\right)}=\psi_{L}^{T}\sigma^{2}S_{L}^{-1}
\]
		\pause
		similarly
\[
\psi_{R}^{T}\sigma^{2}\rightarrow\psi_{R}^{T}e^{-i\mathbf{\theta}.\frac{\mathbf{\sigma}^{T}}{2}+\mathbf{\beta}.\frac{\mathbf{\sigma}^{T}}{2}}\sigma^{2}=\psi_{R}^{T}\sigma^{2}e^{-\left(-i\mathbf{\theta}.\frac{\mathbf{\sigma}}{2}+\mathbf{\beta}.\frac{\mathbf{\sigma}}{2}\right)}=\psi_{R}^{T}\sigma^{2}S_{R}^{-1}
\]
\pause

		\item We want to make it compact. To that end, we note
		\pause
\[
\psi^{T}C=i\left(\begin{array}{cc}
\psi_{L}^{T} & \psi_{R}^{T}\end{array}\right)\left(\begin{array}{cc}
\sigma^{2} & 0\\
0 & -\sigma^{2}
\end{array}\right)=i\left(\begin{array}{cc}
\psi_{L}^{T}\sigma^{2} & -\psi_{R}^{T}\sigma^{2}\end{array}\right)
\]
\pause
	\end{itemize}
\end{frame}

 









 
\begin{frame}
	\frametitle{Majorana Mass}
		\begin{itemize}

		\item so that the transformation is 
		\pause
\[
\psi^{T}C\rightarrow i\left(\begin{array}{cc}
\psi_{L}^{T}S_{L}^{T}\sigma^{2} & -\psi_{R}^{T}S_{R}^{T}\sigma^{2}\end{array}\right)\]
\[=i\left(\begin{array}{cc}
\psi_{L}^{T}\sigma^{2}S_{L}^{-1} & -\psi_{R}^{T}\sigma^{2}S_{R}^{-1}\end{array}\right)=\psi^{T}C\lambda_{1/2}^{-1}
\]
\pause

		\item And what will all of this do? Well, it means that 
		\pause
\[
\psi^{T}C\psi\rightarrow\psi^{T}C\lambda_{1/2}^{-1}\lambda_{1/2}\psi=\psi^{T}C\psi
\]
\pause
		\item Result: We have arrived at a Lorentz scalar!

	\end{itemize}
\end{frame}

 









 
\begin{frame}
	\frametitle{Majorana Mass}
		\begin{itemize}
		\item Comments:
		\pause
			\begin{itemize}
				\item We could've arrived at the same result by simply noting that $\lambda_{1/2}^{T}C\lambda_{1/2}=C$
				\pause
				\item I can write $C$ as a product of $\gamma$ matrices as 
				\[
				C=i\left(\begin{array}{cc}
				\sigma^{2} & 0\\
				0 & -\sigma^{2}
				\end{array}\right)=-i\gamma^{2}\gamma^{0}
				\]
				which is easy to verify.
			\end{itemize}
		\end{itemize}
\end{frame}

 









 
\begin{frame}
	\frametitle{Majorana Mass}
		\begin{itemize}

		\item Defn: A different mass term
		\pause
\[
\begin{alignedat}{2}\mathcal{L}_{\text{Majorana Mass}} & \sim & m\psi^{T}C\psi\\
 & = & -im\left(\Psi_{L}^{T}+\Psi_{R}^{T}\right)\gamma^{2}\gamma^{0}\left(\Psi_{L}+\Psi_{R}\right)\\
 & = & -im\left(\Psi_{L}^{T}\gamma^{2}\gamma^{0}\Psi_{L}+\Psi_{R}^{T}\gamma^{2}\gamma^{0}\Psi_{R}\right)\\
 & = & m\left(\Psi_{L}^{T}C\Psi_{L}+\Psi_{R}^{T}C\Psi_{R}\right)
\end{alignedat}
\]
\pause
		\item Result: Does \emph{not} mix the left and right spinors!
		\pause
		\item To ensure reality, we add $-m\psi^{\dagger}C\psi^{*}$

	\end{itemize}
\end{frame}

 









 
\begin{frame}
	\frametitle{Majorana Mass}
	\begin{itemize}
		\item Concluding Remarks:
		\begin{itemize}
			\item Majorana fermion is s.t. $\psi$ equals it's own `conjugate'. \pause Reduces dof from 4 to 2
			\pause
			\item $C$ is closely related to the charge conjugation operator
		\end{itemize}



	\end{itemize}
\end{frame}

 









 
\begin{frame}
	\frametitle{Majorana Mass}
	\begin{itemize}
		\item Concluding Results:
		\begin{itemize}
		\pause
		\item The Lagrangian with the `kinetic part' is
		\pause
		\[
		\begin{alignedat}{2}\mathcal{L}_{\text{Majorana}} & = & \psi_{L}^{\dagger}i\overline{\sigma}.\partial\psi_{L}+\frac{im}{2}\left(\psi_{L}^{T}\sigma^{2}\psi_{L}-\psi_{L}^{\dagger}\sigma^{2}\psi_{L}^{*}\right)\\
		 & = & i\Psi_{L}^{\dagger}\cancel{\partial}\Psi_{L}-\frac{m}{2}\left(\Psi_{L}^{T}C\Psi_{L}+\Psi_{L}^{\dagger}C\Psi_{L}^{*}\right)
		\end{alignedat}
		\]
		\pause
		\item Corresponding Euler Lagrange
		\pause
		\[
		\begin{aligned}i\overline{\sigma}.\partial\psi_{L}-im\sigma^{2}\psi_{L}^{*} & =0\end{aligned}
		\]
		\pause
		which implies that the Klien Gordan is satisfied and \pause it is itself is Lorentz invariant.
		\end{itemize}



	\end{itemize}
\end{frame}

 








\section{Physical Relevance}
\begin{frame}	
	\Huge{\centerline{Physical Relevance}}
	% \tiny{\centerline{first demonstrated to us by Prof. Arvind}}
\end{frame}

 
\begin{frame}
	\frametitle{Physical Relevance}
	\begin{itemize}
		\item No `right handed' neutrinos
		\pause
		\item `left handed' nearly massless
		\pause
		\item The great Standard Model has the term 
		\pause
\[
\mathcal{L}=\overline{\nu}_{L}i\cancel{\partial}\nu_{L}
\]
		\item To implement mass, we must handle half a spinor
		\pause
		\item Dirac mass as we know it, necessarily mixes the left and right handed part
		\pause
		\item In this sense, Majorana spinors and the Majorana mass associated can describe neutrinos
		\pause
		\item There're alternatives, such as `see-saw' model

	\end{itemize}
\end{frame}




%------------------------------------------------
\section{Closing Remarks}
	\begin{frame}
	\Huge{\centerline{The End}}
	\end{frame}

	\begin{frame}
	\Huge{\centerline{?}}
	\end{frame}

\begin{frame}
	\frametitle{References}
	\begin{itemize}
		\item An Introduction to Quantum Field Theory 

		\emph{M. E. Peskin, D. V. Schroeder} 

		\textbf{Addison-Wesley Publishing Company}
		\item PHY659 Lectures

		\emph{Prof. C. S. Aulakh}

		\textbf{Spring 2015, IISER Mohali}
	\end{itemize}

\end{frame}
%----------------------------------------------------------------------------------------

\end{document} 