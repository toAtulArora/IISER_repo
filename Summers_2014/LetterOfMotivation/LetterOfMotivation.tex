%%%%%%%%%%%%%%%%%%%%%%%%%%%%%%%%%%%%%%%%%
% Thin Formal Letter
% LaTeX Template
% Version 1.11 (8/12/12)
%
% This template has been downloaded from:
% http://www.LaTeXTemplates.com
%
% Original author:
% WikiBooks (http://en.wikibooks.org/wiki/LaTeX/Letters)
%
% License:
% CC BY-NC-SA 3.0 (http://creativecommons.org/licenses/by-nc-sa/3.0/)
%
%%%%%%%%%%%%%%%%%%%%%%%%%%%%%%%%%%%%%%%%%

%----------------------------------------------------------------------------------------
%	DOCUMENT CONFIGURATIONS
%----------------------------------------------------------------------------------------
\documentclass{letter}

\usepackage{ulem}

% Adjust margins for aesthetics
% \addtolength{\voffset}{-0.5in}
% \addtolength{\hoffset}{-0.3in}
% \addtolength{\textheight}{2cm}

\addtolength{\voffset}{-0.1in}
\addtolength{\hoffset}{-0.1in}
\addtolength{\textheight}{0.1cm}

%\longindentation=0pt % Un-commenting this line will push the closing "Sincerely," to the left of the page

%----------------------------------------------------------------------------------------
%	YOUR NAME & ADDRESS SECTION
%----------------------------------------------------------------------------------------

\signature{Atul Singh Arora} % Your name for the signature at the bottom

\address{4317/3 Ansari Road, \\Darya Ganj, \\New Delhi-110002} % Your address and phone number

%----------------------------------------------------------------------------------------

\begin{document}

%----------------------------------------------------------------------------------------
%	ADDRESSEE SECTION
%----------------------------------------------------------------------------------------

\begin{letter}{The DAAD People,\\German Academic Exchange Service\\2, Nyaya Marg\\Chanakyapuri\\New Delhi-110021} % Name/title of the addressee

%----------------------------------------------------------------------------------------
%	LETTER CONTENT SECTION
%----------------------------------------------------------------------------------------

\opening{Sir/Ma'am,}

My task is to dazzle you. No, my task is to honestly tell you about me. 

\begin{enumerate}
\item I am crazy

\emph{I am over analytic.} Which is to say I try to analyse everything. I try to think about what I should or shouldn't do. I think if religion is required for this purpose or not. What is it to do the right thing. Why be moral. And they all become relevant to this very process of application. If my objective were simply to get through, it would be in my interest to not help my colleagues with the application. Instead I put in my best effort to help (even pursue) every eligible person I knew. Why, for I sincerely believe that the best amongst us should get to go, for that is the right thing to do. I've felt a little lonely with this school of thought. Then I was introduced to the philosopher Immanuel Kant and he made it logically consistent to say that \emph{to be free, is to be moral}. He went on to say that being moral does not guarantee happiness, just guarantees you deserve it.\footnote{Groundwork of the Metaphysics, Immanuel Kant} At this point, I buy it.

\item I am driven

\emph{I am inspired.} I'll write here whatever comes to my mind, as an answer to the question of what is it about science that really interests me, my real passion. I could start by stating some of the things I find extremely fascinating and yes, without too much surprise, we have relativity (special). This is to say that not only is the formalism, as given in Landau's Classical theory of fields, astounding, the thought process, the journey, the tiny radical change that brought the theory into existence to stand up to every experimental test is remarkable. How Einstein was able to realize the inherent bias of our Newtonian experience, and was able to follow it to its root, simultaneity, is proof of the fact that true progress in the field of physics has often hinged on not formalism, but the astute observation of experimental facts and development of a consistent insight, which then is stated with the language of mathematical precision. Why I started with relativity was because with the introduction of quantum mechanics, one can see the phenomenon of entanglement. The apparent instantaneous interaction regardless of the distance between the entangled parts, without violating the postulates of relativity, makes studying nature, one of the most fascinating activity a human could participate in. That said, even in relativity, the concept of existence of world lines themselves, was found to have a classical bias! This is one of the radical change in our understanding, upon which quantum mechanics is built. And this is not special to developments in modern physics. Maxwell's formulation that connected optics experiments, which practically have nothing in common with wires and magnets, was again an example of this.

\emph{I believe.} I thus find it reasonable to believe that at any given point of time, there are always new discoveries, waiting to be unravelled. But let's face it. I am a student and I don't know enough. However I will always have more to learn, thereby will always be a student. Upon this realization I have put my faith in the possibility that I too will innovate. I have a popular final goal, of creating a usable quantum computer, which everybody seems to be working on. So why spend resources to add another name to the list? Faraday upon being asked what use his research was, answered with a question, `of what use is a new born baby?'. Computers when first created commercially, were estimated to have a market of around 7 in the world, in a year. The point, well, we are again in a similar situation with the much to familiar term, Quantum Computing. I believe I can.

\emph{I am focussed.} Quantum Computing, revolutionary as it is, stands at the bleeding edge of physics, mathematics and engineering, the fields closest to my heart. I am interested in looking at the theoretical and practical difficulties of building such a device, specifically with the problems of de-coherence and memory cloning. At this stage I don't know enough, however I am actively learning all the physics necessary, with as much depth as I can, so that my tools are sharp enough to attack the problem when its project time.

% todo: put the science stuff
% Writing this is not easy. Which seems to suggest that reading this, is not either. But while I'm at it, let me tell you how as a child I believed I could build a perpetual motion machine and spent months trying to put the magnets in the write sequence to construct one. I even stuck electrodes in a magnet to extract the current that was producing it, little did I know about bound currents.

\item I am bad

\emph{I have a messy memory.} I wont say I can't remember things, its just that I can't choose what to remember. I forget names, numbers, passwords and even equations, but I tend to remember people, actions (mine and others) and concepts, to enumerate some.
\emph{I can't restrict to syllabus.} Then there are times I am not able to finish the syllabus; get stuck at learning a concept in reasonable depth.
\emph{I'm bad with numbers.} Doesn't imply I am bad at Mathematics; I'm bad at a part of arithmetic.
\emph{I can't score.} With these, I have very rarely been able to score as much as I should. Further, it's mostly because if there's an interesting problem I can't solve, it swallows all my time. 
\emph{I need regular rest.} Working obsessively for too long renders me useless for a short duration. Vacations help. I work on alternate problems in this time, else I get bored.
\emph{I'm irrational.} I have some beliefs I can't rationally justify, eg. Vegetarianism.\footnote{However, I ensure that, no intention thus produced, is morally incorrect.}

\item I am good

\emph{I work hard.} I didn't acquire this, was just capable of it. I have worked hard despite continuous failures at numerous occasions and have eventually succeeded. I find it hard to leave problems unsolved.
\emph{I am sincere.} This helps me do everything simultaneously as close to the way I like, as possible.
\emph{I am critical.} I tend to analyse and accept only when I think there's nothing wrong. Useful in science, not so much socially.
\emph{I am decisive.} I make choices and stick to them.
\emph{I am hyper organized.} It is more of a requirement as I can't choose what to remember.
\emph{I have a strong will.} I am almost always able to do what I want to, not what I feel like. Eg. Eating healthy food.
\emph{I am a geek.} I am good with computers.\footnote{Look at my CV for supporting evidence}


% \item I am crazy

% I don't care about what people say, so long as I am convinced that what I'm doing isn't morally incorrect. Eg. I tried colouring my beard red.

% I find it very hard to leave problems unsolved.

% I behave in much the same way with everyone, friend, junior or even a professor.

% I am blunt; I don't deceive.

\end{enumerate}


To conclude, I would thank you for your time and sincerely hope that the best\footnote{`best', however you define it} candidate is selected.

% That said, I should now atleast tell you why I myself think that even though I don't have a 9 point something CPI (yet), I'll make a for a brilliant researcher. Only yesterday, I had assigned myself the task of reading Landau's Classical Theory of Fields. As it turns out, I got stuck at how it is always true that proper time is the slowest. Don't worry if physics is not your subject. I'm trying to point to a human quality, for which I don't take much credit, I have mostly just been born with it. The quality is that of sustained, directed persistance. I stayed up all night, until it made enough sense that I found myself attempting to interdependently re-deriving a result. I have observed in the past that I am unable to leave problems unsolved which is nearly fatal for being successful at scoring well; thereby suggesting that the CPI in some cases may not accurately reflect a persons ability to research.

\vspace{2\parskip} % Extra whitespace for aesthetics
\closing{Sincerely,}
\vspace{2\parskip} % Extra whitespace for aesthetics

% \ps{P.S. You can find additional information attached to this letter.} % Postscript text, comment this line to remove it

% \encl{Copyright permission form} % Enclosures with the letter, comment this line to remove it

%----------------------------------------------------------------------------------------

\end{letter}
 
\end{document}