%************************************************
\chapter{Introduction to Infra Red and Ultra Violet - Visible Spectroscopy}
%************************************************
\begin{flushright}
August 9, 2012
\end{flushright}

For this session, both IR and UV-visible spectroscopy techniques were demonstrated to us in groups of two.

\section{Theory: Basic Concept}
To find the presence of elements and/or compounds within a given substance, we can use spectroscopy techniques, specially when their concentrations are small and they satisfy certain requirements. The essential idea behind this measurement comes from the fact that elements/compounds absorb lights of certain frequencies to get to a higher energy state. These frequencies are mostly discrete as they correspond to quantized energy levels. This energy could be absorbed for, say, changing the rotational energy (IR Spectroscopy) or for exciting an electron in the substance to a higher energy level (UV-vis Spectroscopy). We note here that these quantized energy levels are properties of individual substances and are, for most practical purposes, unique.\\
 \\
For the analysis to be possible, the first condition is that the substance must \emph{absorb} light incident to it. \marginpar{How much absorption, well, the limit comes from the sensitivity of the experimental setup and concentration of substance given.} Granted this, we can obtain an absorption spectrum for the given substance, which behaves like a fingerprint of the substance. This can thus be used to not only identify the compound, but also to quantify it. For identification, in the simplest case, we simply need to observe the frequency corresponding to the peaks in the absorption spectrum and match it with the known/expected substance(s). Quantification harnesses a rather ``obvious'' law, termed \emph{Beer-Lambert's Law}. In the simplest form, the law quantifies the intuitive notion; higher the concentration of the analyte, higher is the absorption. The relation is given as
\begin{equation}
T=\frac{I}{I_{0}}=10^{-\alpha l}=10^{- \epsilon l c}
\label{beerlambertslaw}
\end{equation}
where $I$ is intensity of incident light, $I_{0}$ is intensity of transmitted light, $\epsilon$ is molar absorbtivity, $l$ is the optical path length, and $c$ is molar concentration.

\section{Experimental Details}
There are a couple of options for \texttt{classicthesis.sty} that
allow for a bit of freedom concerning the layout:
\marginpar{\dots or your supervisor might use the margins for some
    comments of her own while reading.}

Many other customizations in \texttt{classicthesis-config.tex} are
possible, but you should be careful making changes there, since some
changes could cause errors.

Finally, changes can be made in the file \texttt{classicthesis.sty},%
\marginpar{Modifications in \texttt{classicthesis.sty}%
} although this is mostly not designed for user customization. The
main change that might be made here is the text-block size, for example,
to get longer lines of text.


\section{Issues}\label{sec:issues}
This section will list some information about problems using

\subsection*{Compatibility with the \texttt{glossaries} Package}
If you want to use the \texttt{glossaries} package, take care of loading it 
with the following options:
\begin{verbatim}
	\usepackage[style=long,nolist]{glossaries}
\end{verbatim}
Thanks to Sven Staehs for this information. 


\subsection*{Compatibility with the (Spanish) \texttt{babel} Package}
Spanish languages need an extra option in order to work with this template:


\subsection*{Compatibility with the \texttt{pdfsync} Package}
Using the \texttt{pdfsync} package leads to linebreaking problems with the \texttt{graffito} command 

\section{Future Work}
So far, this is a quite stable version that served a couple of people


\section{Beyond a Thesis}
It is easy to use the layout of \texttt{classicthesis.sty} without the


\section{License}
\paragraph{GNU General Public License:} This program is free software;
