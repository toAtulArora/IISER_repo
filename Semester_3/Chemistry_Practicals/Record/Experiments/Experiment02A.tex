%************************************************
\chapter{Experiment 2A: Job's Method of Continuous Variation (UV-Vis)}
%************************************************
\begin{flushright}
August 30, 2012
\end{flushright}

\section{Objective}
	To find the reaction stoichiometry of the $Fe^{3+}$-Salicylic complex using Job's method of continuous variation.

\section{Theory}
	The idea is fairly simple. We analyse a compound which is formed by a combination of two reactant substances. The only requirement is that we should know of some way using which we can quantify the compound, given a mixture consisting of the compound along with the residual reactant substances.
	\par
	Here's what we do...:
	\begin{enumerate}
		\item Decide on a total number of moles you will initially take (of both the reactant substances combined)
		\item Now take the reactant substances in various ratios, such that the total number of moles is as decided
		\item For each ratio, find out the quantity of the compound obtained
	\end{enumerate} 
	...and why:
	\par
	A little thought will make the entire procedure appear elegantly simple. In the various combinations, one of the reactants will always be a limiting reagent. However, the highest yield will be obtained in the case when the reactants are closest to the stoichiometric ratio of the compound. Thus, a plot of the concentration of the compound against that of one of the reactants will attain a maxima. The point at which the maxima is obtained, can be used to find the concentration of both reactants, whose ratio gives us the reaction stoichiometry, with respect to the reactants!
	\par
	And just to state the obvious, as the title suggests, the technique we use for quantifying is UV Visible spectroscopy.
\section{Procedure}
	As always, when practically performing an experiment, we have a few extra details to take care of.
	\begin{enumerate}
		\item Make a $0.001M$, $25mL$ solution of Iron Nitrate, $Fe(NO_{3})_{3}$.
			\begin{enumerate}
				\item The Molar Mass was given to be $404g$.
				\item Thus, for $25mL$ we need $10.0mg$.
			\end{enumerate}
		\item Make a $0.001M$, $25mL$ solution of Salicylic Acid
			\begin{enumerate}
				\item The Molar Mass was given to be $138.12g$
				\item Thus, for $25mL$, we need $3.5mg$.
			\end{enumerate}
		\item \marginpar {\Lisa Since the molarity of both solutions, is the same, we can use volume as a measure of number of moles.} We use $10mL$ as the total volume.
		\item We used the volumes given in \autoref{2A_conc}, of the reactants from the $0.001M$ solutions prepared, for creating $5mL$ solutions and marked the volumetric flasks with the corresponding concentrations.
			\begin{enumerate}
				\item Used an appropriate graduated pipette for measuring the volumes
				\item Used the pipette for measuring one of the reactants only, and filled the volumetric flask with the other reactant using a dropper, to $5mL$ using the mark on the flask. This was done to avoid errors.
				\item Each of the flasks were ensured to be dry.
			\end{enumerate}
			\begin{table}
				\myfloatalign
				\begin{tabularx}{\textwidth}{Xll}
					\tableheadline{$Fe^{3+}$ Solution's Volume ($mL$)} & \tableheadline{Salicylic Acid's Volume ($mL$)}\\
					% \tableheadline{Volume ($mL$)} & \tableheadline{Salicylic Acid Volume ($mL$)}\\
					\hline%
					$0.5$				& 	$4.5$\\
					$2.0$				& 	$3.0$\\
					$2.5$				& 	$2.5$\\
					$3.0$				& 	$2.0$\\
					$4.5$				& 	$0.5$\\
					\hline%
				\end{tabularx}
				\caption{Concentrations for Job's Method}
				\label{2A_conc}
			\end{table}
		\item Recorded the spectrum of one of the samples and found the peak corresponding to the compound we're interested in, viz. $Fe^{3+}-$-Salicylate Complex.
		\item Recorded the absorbance of the frequency determined in the previous step for all concentrations.
	\end{enumerate}

\section{Observations and Analysis}
	

\section{Discussion}
	For this experiment, even though we knew which compounds we're taking, after printing the spectrographs, we didn't label them to see if they could be identified by matching them with their expected peaks. This method does in fact work quite accurately and the fingerprint region was also observed to match, when compared with known spectrographs. I have still not been able to understand the meaning of alpha and beta structures of Naphthalene, which I so far believe to not exist. However, experimentally, in accordance with the literature, the structure seems to be alpha.\\

\section{Acknowledgements}
	I acknowledge the contribution of Mr. Arpit Porwal for the performance of the experiment as a team member. I would also like to thank Mr. Biplob Nandy and Ms. Saumya Gupta for exchanging results to mutually provide scope for wider analysis. \\
	I am grateful to Mr. Arjit Kant Gupta who helped me with the analysis and confirmation of the results by sharing his spectrographs of known compounds.\\
	I also thank L. G. Wade for the book ``Organic Chemistry'' which was used to look up the expected IR peaks in the compounds analysed.