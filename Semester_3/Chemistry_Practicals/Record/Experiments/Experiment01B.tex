%************************************************
\chapter{Experiment 1B: Infra Red Spectroscopy}
%************************************************
\begin{flushright}
August 23, 2012
\end{flushright}

\section{Objective}
	To study the spectroscopic characteristics of atleast three of the following compounds using an FTIR spectroscope.
	\begin{enumerate}
		\item Benzoic Acid
		\item Naphthalene
		\item Urea
		\item Thio-Urea
		\item Salicylic Acid
	\end{enumerate}

\section{Theory}
	This section is the same as and has been covered earlier in \autoref{chapter1}.\\

\section{Procedure}
	\begin{enumerate}
		\item Weighed approximately $200mg$ of $KBr$ and $2-4mg$ of sample and ground it well using mortar and pestle. (Didn't use any sample for the first time, to use for baseline correction.)
		\marginpar{\Lisa The process must be done quickly to avoid accumulation of moisture.}
		\item Transferred the powdered contents into the palette maker and put it inside the hydraulic press
		\item Applied about $6-8$ tons of pressure for $30-60$ seconds. Released the pressure after, and transferred the contents to the palette holder.
		\item Placed the palette holder into the FTIR spectroscope and scanned it (in case of the first palette, background scanned it).
	\end{enumerate}

\section{Analysis}
	\begin{enumerate}
		\item Urea - \autoref{1B_urea}
			\begin{table}
				\myfloatalign
				\begin{tabularx}{\textwidth}{Xlll}
					\hline%
					$N-H$ stretching					& 	$3399.69$	& 	$cm^{-1}$\\
					$C=O$ stretching					& 	$1669.64$	& 	$cm^{-1}$\\
					$N-H$ bending						& 	$1625.39$	&	$cm^{-1}$\\
					\hline%
				\end{tabularx}
				\caption{IR peaks for Urea}
				\label{1B_urea}
			\end{table}
		\item Benzoic Acid - \autoref{1B_benzoicAcid}
			\begin{table}
				\myfloatalign
				\begin{tabularx}{\textwidth}{Xlll}
					\hline%
					$C-H$ stretching					&	$2832.14$				& 	$cm^{-1}$\\
					$O-H$ stretching					&	$2554.86$				& 	$cm^{-1}$\\
					$C=O$ stretching					&	$1694.66$ and $1699.26$	& 	$cm^{-1}$\\
					\hline%
				\end{tabularx}
				\caption{IR peaks for Benzoic Acid}
				\label{1B_benzoicAcid}
			\end{table}

		\item Thio Urea - \autoref{1B_thioUrea}
			\begin{table}
				\myfloatalign
				\begin{tabularx}{\textwidth}{Xlll}
					\hline%
					$NH_{2}$ stretching					&	$3368.14$				& 	$cm^{-1}$\\
					$NH_{2}$ bending					&	$1588.41$				& 	$cm^{-1}$\\
					$C=S$ stretching					&	$1092.87$				& 	$cm^{-1}$\\
					\hline%
				\end{tabularx}
				\caption{IR peaks for Thio-Urea}
				\label{1B_thioUrea}
			\end{table}

		\item Naphthalene - Since there are no functional groups, according to L. G. Wade's \emph{Organic Chemistry}, peaks are expected to be observed in the following ranges
			\begin{enumerate}
				\item 
					For Naphthalene in general:
					\begin{enumerate}
						\item between $3000$ and $3100$ 
						\item between $1570$ and $1650$ \\
					\end{enumerate}			
					Corresponding observations are given in \autoref{1B_naphthalene_expected}
				\item 
					For \emph{alpha} \marginpar{\Bart What does alpha mean structurally in this case?} naphthalene, we have
					\begin{enumerate}
						\item between $1375$ and $1425$
						\item between $750$ and $810$
					\end{enumerate}
					Corresponding observations are given in \autoref{1B_naphthalene_alpha}
				\item
					For \emph{beta} naphthalene, peaks are expected to be in the ranges:
					\begin{enumerate}
						\item less than $700$
						\item between $800$ and $860$
					\end{enumerate}
					These were found to be absent.
			\end{enumerate}

		
			\begin{table}
				\myfloatalign
				\begin{tabularx}{\textwidth}{Xll}
					\hline%
					$3061.97$				& 	$cm^{-1}$\\
					$3048.41$				& 	$cm^{-1}$\\
					$3029.35$				& 	$cm^{-1}$\\
					$1651.94$				& 	$cm^{-1}$\\
					$1634.25$				& 	$cm^{-1}$\\
					$1592.53$				& 	$cm^{-1}$\\
					\hline%
				\end{tabularx}
				\caption{IR peaks for Naphthalene}
				\label{1B_naphthalene_expected}
			\end{table}

			\begin{table}
				\myfloatalign
				\begin{tabularx}{\textwidth}{Xll}
					\hline%
					$1383.98$				& 	$cm^{-1}$\\
					$782.47$				& 	$cm^{-1}$\\
					\hline%
				\end{tabularx}
				\caption{IR peaks for \emph{alpha} Naphthalene}
				\label{1B_naphthalene_alpha}
			\end{table}


	\end{enumerate}

\section{Discussion}
	For this experiment, even though we knew which compounds we're taking, after printing the spectrographs, we didn't label them to see if they could be identified by matching them with their expected peaks. This method does in fact work quite accurately and the fingerprint region was also observed to match, when compared with known spectrographs. I have still not been able to understand the meaning of alpha and beta structures of Naphthalene, which I so far believe to not exist. However, experimentally, in accordance with the literature, the structure seems to be alpha.\\

\section{Acknowledgements}
	I acknowledge the contribution of Mr. Arpit Porwal for the performance of the experiment as a team member. I would also like to thank Mr. Biplob Nandy and Ms. Saumya Gupta for exchanging results to mutually provide scope for wider analysis. \\
	I am grateful to Mr. Arjit Kant Gupta who helped me with the analysis and confirmation of the results by sharing his spectrographs of known compounds.\\
	I also thank L. G. Wade for the book ``Organic Chemistry'' which was used to look up the expected IR peaks in the compounds analysed.