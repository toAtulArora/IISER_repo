% Assignment 2 (MTH201), To be submitted by October 16 2012.

% Change the file name to YourRollNumber.tex (with MS in capitals), for example MS10420.tex

\documentclass[a4paper,10pt]{article}
\usepackage{amssymb,amsmath}
\begin{document}
%\begin{redbox}
% Write a parametric equation of the curve (edit the text within $ $ only)
%\hfill

% Write a popular name of the curve, if any
\begin{center} 
\textsc{Simple Galaxoid} \footnote{This term has been coined by me.}
\end{center}

%\par
\begin{equation}
\gamma(t)=(\gamma_x(t),\gamma_y(t))
\end{equation}

where $\gamma_x$ and $\gamma_y$ are defined after the following discussion.

%\end{redbox}
\begin{itemize}
% Mention if the curve is planar (Planar/Space)
\item Planar or Space?: Planar
% Write the formula of the curvature function (edit the text within $ $ only)
% \item Curvature function: $$\kappa(t) = -\frac{t^n}{a}.$$
% Write the formula of the torsion function (edit the text within $ $ only)
\\
	Since the curve is planar, its torsion = 0 %Since it's a planar curve, its torsion = 0
\item  This curve, despite having a fairly complicated equation, can be understood as a combination of 5 clones of a single spiral curve. I find it interesting because, well of its beauty. Further, I devised a method (algorithm) to combine various curves into a single curve, using what I call a cropping function, which may result in large equations, but it achieves its objective without much difficulty. This was the first curve I created using the said algorithm, thus the interest.

%\begin{itemize}
% Remark 1 about the curve
\item For any given t, only one of the summation terms is non-zero.
% Remark 2 about the curve
\item The curvature is the same as that of the `base' curve, provided it's defined at that point.
% Remark 3 about the curve
\item The cropping function is the one that uses Ceiling and Floor functions, and it essentially `selects' which curve to plot.
\end{itemize}
% Write your Name (Roll No.)
\vspace{30pt}
\hfill {\it Submitted by:} Atul Singh Arora (MS11003)


\begin{equation}
\begin{split}
\gamma_x(t) = &(\lceil m(t - ((0)\alpha)) \rceil)(\lfloor m(-t + (1\alpha)) \rfloor + 1) \\
			&( \cos((0)2\pi/5)\\
			&((t-(0)\alpha))\\
			&(1.2^{(t-(0)\alpha)}))\cos((t-(0)\alpha))))\\ 
			&- \sin((0)2\pi/5)\\
			&((t-(0)\alpha)\\
			&(1.2^{(t-(0)\alpha)})\sin((t-(0)\alpha)))\\
		+	&(\lceil m(t - ((1)\alpha)) \rceil)(\lfloor m(-t + (2\alpha)) \rfloor + 1) \\
			&( \cos((1)2\pi/5)\\
			&((t-(1)\alpha))\\
			&(1.2^{(t-(1)\alpha)}))\cos((t-(1)\alpha))))\\ 
			&- \sin((1)2\pi/5)\\
			&((t-(1)\alpha)\\
			&(1.2^{(t-(1)\alpha)})\sin((t-(1)\alpha)))\\						
		+	&(\lceil m(t - ((2)\alpha)) \rceil)(\lfloor m(-t + (3\alpha)) \rfloor + 1) \\
			&( \cos((2)2\pi/5)\\
			&((t-(2)\alpha))\\
			&(1.2^{(t-(2)\alpha)}))\cos((t-(2)\alpha))))\\ 
			&- \sin((2)2\pi/5)\\
			&((t-(2)\alpha)\\
			&(1.2^{(t-(2)\alpha)})\sin((t-(2)\alpha)))\\			
		+	&(\lceil m(t - ((3)\alpha)) \rceil)(\lfloor m(-t + (4\alpha)) \rfloor + 1) \\
			&( \cos((3)2\pi/5)\\
			&((t-(3)\alpha))\\
			&(1.2^{(t-(3)\alpha)}))\cos((t-(3)\alpha))))\\ 
			&- \sin((3)2\pi/5)\\
			&((t-(3)\alpha)\\
			&(1.2^{(t-(3)\alpha)})\sin((t-(3)\alpha)))\\			
		+	&(\lceil m(t - ((4)\alpha)) \rceil)(\lfloor m(-t + (5\alpha)) \rfloor + 1) \\
			&( \cos((4)2\pi/5)\\
			&((t-(4)\alpha))\\
			&(1.2^{(t-(4)\alpha)}))\cos((t-(4)\alpha))))\\ 
			&- \sin((4)2\pi/5)\\
			&((t-(4)\alpha)\\
			&(1.2^{(t-(4)\alpha)})\sin((t-(4)\alpha)))\\			
\end{split}
\end{equation}

\begin{equation}
\begin{split}
\gamma_y(t) = &(\lceil m(t - ((0)\alpha)) \rceil)(\lfloor m(-t + (1\alpha)) \rfloor + 1) \\
			&( \sin((0)2\pi/5)\\
			&((t-(0)\alpha))\\
			&(1.2^{(t-(0)\alpha)}))\cos((t-(0)\alpha))))\\ 
			&+ \cos((0)2\pi/5)\\
			&((t-(0)\alpha)\\
			&(1.2^{(t-(0)\alpha)})\sin((t-(0)\alpha)))\\
		+	&(\lceil m(t - ((1)\alpha)) \rceil)(\lfloor m(-t + (2\alpha)) \rfloor + 1) \\
			&( \sin((1)2\pi/5)\\
			&((t-(1)\alpha))\\
			&(1.2^{(t-(1)\alpha)}))\cos((t-(1)\alpha))))\\ 
			&+ \cos((1)2\pi/5)\\
			&((t-(1)\alpha)\\
			&(1.2^{(t-(1)\alpha)})\sin((t-(1)\alpha)))\\						
		+	&(\lceil m(t - ((2)\alpha)) \rceil)(\lfloor m(-t + (3\alpha)) \rfloor + 1) \\
			&( \sin((2)2\pi/5)\\
			&((t-(2)\alpha))\\
			&(1.2^{(t-(2)\alpha)}))\cos((t-(2)\alpha))))\\ 
			&+ \cos((2)2\pi/5)\\
			&((t-(2)\alpha)\\
			&(1.2^{(t-(2)\alpha)})\sin((t-(2)\alpha)))\\			
		+	&(\lceil m(t - ((3)\alpha)) \rceil)(\lfloor m(-t + (4\alpha)) \rfloor + 1) \\
			&( \sin((3)2\pi/5)\\
			&((t-(3)\alpha))\\
			&(1.2^{(t-(3)\alpha)}))\cos((t-(3)\alpha))))\\ 
			&+ \cos((3)2\pi/5)\\
			&((t-(3)\alpha)\\
			&(1.2^{(t-(3)\alpha)})\sin((t-(3)\alpha)))\\			
		+	&(\lceil m(t - ((4)\alpha)) \rceil)(\lfloor m(-t + (5\alpha)) \rfloor + 1) \\
			&( \sin((4)2\pi/5)\\
			&((t-(4)\alpha))\\
			&(1.2^{(t-(4)\alpha)}))\cos((t-(4)\alpha))))\\ 
			&+ \cos((4)2\pi/5)\\
			&((t-(4)\alpha)\\
			&(1.2^{(t-(4)\alpha)})\sin((t-(4)\alpha)))\\			
\end{split}
\end{equation}


\end{document}