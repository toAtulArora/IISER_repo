%************************************************
\chapter{Michelson Interferometer}
%************************************************
\begin{flushright}
November 16, 17 2012
\end{flushright}
\section{Aim}
	To
	\begin{enumerate}
		\item observe circular fringes for a monochromatic light source (sodium lamp)
		\item obtain the wavelength of sodium D-lines
		\item obtain the separation of the sodium D-lines
	\end{enumerate}

\section{Apparatus}	
	Michelson interferometer setup, which consists of a sodium lamp source, diffusing glass, beam splitter, moveable and fixed mirror with measurement gauges, compensator

\section{Theory}
	\subsection{Introduction}
		Michelson interferometer is a rather simple and elegant experiment, where a beam is split into two and then made to meet after travelling different distances. 
	\subsection{Determining Wavelength}
		The easiest place to begin the analysis of this apparatus would be to ignore the thickness of the beam splitter and the compensator (which is anyway not needed for this experiment, as is explained later) and assume the observations are made at the centre of a far away screen (although in the experiment we've used a lens to focus at infinity). Now at this point, the path difference in the two beams of light would be essentially because of twice the distance between the fixed mirror and the image of the movable mirror. Let this be given by $2d$, where $d$ is the distance between the said mirrors. Now imagine a point other than than the centre at the screen. The angle this makes with the principal axis, let that be $\theta$. The distance light will travel for this point, which can cause phase difference will be $2d \cos{\theta}$. Keeping in mind the fact that there's an abrupt phase change of $\pi$ because of reflection at the beam splitter, we have for destructive interference
		\begin{equation}
			2d\cos{\theta}=m\lambda
		\end{equation}
		and similarly, for constructive interference we'll have
		\begin{equation}
			2d\cos{\theta}=(m+1/2)\lambda
		\end{equation}
		It is easy to put in a few numbers \footnote{refer to page 15.23 of Ajoy Ghatak, Optics, 4th Edition for details} and conclude that reducing the distance causes fringes to collapse to the centre (while the spacing between fringes increases). Say for a given configuration, the centre's dark;
		\begin{equation}
			2d=m\lambda
		\end{equation}
		and say for a distance $d_o$ the same is achieved, and $N$ fringes have collapsed to the centre in the process. Then we have
		\begin{equation}
			2(d+d_o)=(m-N)/\lambda
		\end{equation}
		Subtracting these, gives us a simple method of finding the `average' wavelength of the sodium source
		\begin{equation}
			\lambda=2d_o/N
			\label{5_eq1}
		\end{equation}

	\subsection{Resolving the D-Lines}
		That was the basic theory behind the experiment. Using a little more naiive Math, we can even find the small difference in the Sodium D-lines. Here's how we go about it. First `$d$' is made 0. Then the mirror is moved away (or towards) through a distance $d$. In general, the fringe patterns will overlap in some fashion. Assume a $d$ is such that,
		\begin{equation}
			2d\cos{\theta}=m\lambda_1 \\
			2d\cos{\theta}=(m+1/2)\lambda_2
		\end{equation}
		For small $\theta$, we can easily obtain
		\begin{equation}
			2d/\lambda_1 - 2d/\lambda_2 = 1/2
		\end{equation}
		Observe here what has happened. The maxima of one pattern falls on the minima of the other and vice versa. This means that what we obtain is all bright! Consequently, the pattern disappears in this situation. Making this general, if
		\begin{equation}
			2d/\lambda_1 - 2d/\lambda_2 = 1/2, 3/2, 5/2 ..
		\end{equation}
		then the fringe pattern will disappear and for 1, 2, 3 .. it will reappear. Now let us move another step further. ay for an experiment, I find the distance between occurrence of two blank patterns, we have
		\begin{equation}
			2d_1 (1/\lambda_1 - 1/\lambda_2) = 1/2
		\end{equation}
		\begin{equation}
			2d_2 (1/\lambda_1 - 1/\lambda_2) = 3/2
		\end{equation}
		\begin{equation}
			2\Delta d (\Delta \lambda / \lambda^2) = 1
		\end{equation}
		\begin{equation}
			\Rightarrow \Delta \lambda = \lambda^2/2\Delta d
		\end{equation}

\section{Procedure}
	\begin{enumerate}
	\item The sodium light source was switched on, allowed to heat up and kept at a distance of about 50 cm from the interferometer. 
	\item Measured the position of fixed mirror w.r.t. beam splitter with a ruler and brought the movable mirror to the same distance from the beam splitter.
	\item The pinhole was brought in front of the sodium light source. Two sets of images of the pinhole were visible through the telescope. The two sets were brought to coincidence with the screws on the fixed and movable mirrors. 
	\item Pinhole was removed and ground glass was placed in front of the light. No fringes were obtained so the positions of movable mirror and compensator were changed. 
	\item The steps 3 and 4 were repeated many times so as to obtain the fringes.
	\item The fringes were observed through the telescope and position of movable mirror was altered with the help of the drum. 
	\item When the mirror was moved the fringes seemed to collapse towards the center of the pattern. Micrometer readings were taken after 10 fringes passed through the cross of the telescope. 
	\item Average wavelength was calculated as $10\lambda = 2(d_0 - d_{10})$.

	\end{enumerate}

\section{Observations and Calculations}	
	The average separation $\Delta D$ was found to be equal to $0.00283 \pm 0.00014 mm$ using the values given in \autoref{5_final}. Using \autoref{5_eq1} we get the average wavelength to be $566 \pm 29 nm$.
	For the separation in D-lines, the experiment couldn't be setup.
	\begin{table}
		\myfloatalign
		\begin{tabularx}{\textwidth}{Xll}
			\hline
			\tableheadline{Distance (mm)} 	&	\tableheadline{Difference (mm)} \\
			\hline
				0.6228	& 	0.0028 \\
				0.6255	& 	0.0027 \\
				0.6284	& 	0.0029 \\
				0.631	& 	0.0026 \\
				0.6339	& 	0.0029 \\
				0.6367	& 	0.0028 \\
				0.6397	& 	0.0030 \\
				0.6428	& 	0.0031 \\
				0.6455	& 	0.0027 \\
				0.6455\\
			\hline
		\end{tabularx}
		\caption{Average Wavelength of Sodium}
		\label{5_final}
	\end{table}


\section{Result}
	The average wavelength of Sodium was found to be $566 \pm 29 nm = 566 \pm 5\%$. The actual wavelength falls within the error range.

\section{Remarks}
	This experiment doesn't quite require the compensator since the source is monochromatic.

\section{Precautions}
	\begin{enumerate}		
		\item Avoid backlash errors by moving the screw in one direction only
		\item The apparatus is very sensitive, so ensure that no part is loose. Tighten the screws that must be tightened
		\item Ensure there aren't any vibration producing devices (viz. laptops, phones etc.) near the setup.
	\end{enumerate}