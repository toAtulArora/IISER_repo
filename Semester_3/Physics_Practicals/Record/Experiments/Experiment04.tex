%************************************************
\chapter{Sonometer}
%************************************************
\begin{flushright}
October 9, 10 2012
\end{flushright}
\section{Aim}
	To 
	\begin{enumerate}
		\item observe standing waves in a streteched string
		\item vary length and linear density of the sonometer wire and observe changes in frequency
		\item find linear density of an unknown wire
		\item find the fundamental modes using an AC source
	\end{enumerate}

	\section{Apparatus}
	Sonometer, Tuning Forks, Weights, Weighing Machine, Screw Gauge, Magnetic Coil and appropriate circuitry

\section{Theory}
	This experiment is based on a single formula which is
	\begin{equation}
		f=n\frac{v}{\lambda}=n\frac{1}{2l}\sqrt{\frac{T}{m}}
	\end{equation}
	where $f$ is the frequency, $n$ represents the mode of vibration, $l$ is length of the wire, $T$ is the tension on the wire and $m$ is mass per unit length of the wire. The only caveat here is to understand that $f_{ac}$ actually excites the string with twice the frequency because it attracts at both times, producing a $2f_{ac}$ frequency in the wire.
	\par
	We keep, for the first experiment, tension and the wire constant, and find the first fundamental mode for various frequencies. We repeat this for the other wire. The slope of the fundamental length against frequency graph will yield
	\begin{equation}
		m=\frac{l}{f}=\sqrt{\frac{m}{T}}
	\end{equation}
	This was used to find the mass per unit length, since the Tension is already known.	
	\par
	For the next part, the frequency was taken from the AC source, and $n$ was plotted against string length. The slope of the graph yields
	\begin{equation}
		m=\frac{n}{l}=2f\sqrt{\frac{m}{T}}
	\end{equation}
	where $f$ and $T$ are again known, and $m$ was determined using the slope.

\section{Procedure}
	TODO: Complete this
	
\section{Observations and Calculations}	
	TODO: Complete this

\section{Result}
	TODO: Complete this once and for all
	
\section{Precautions}
	\begin{enumerate}		
		\item Micrometer should be moved only in one direction to avoid errors due to backlash.
		\item Ensure the table isn't disturbed while taking one particular round of measurements.
		\item Be sure to use the ammeter in the right range for best results.
	\end{enumerate}