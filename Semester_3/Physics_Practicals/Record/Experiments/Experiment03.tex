%************************************************
\chapter{Single Slite Diffraction}
%************************************************
\begin{flushright}
October 9, 16 2012
\end{flushright}
\section{Aim}
	To study diffraction of light due to a thin slit and due to a thin wire, using an optical bench with a laser and a photodiode. 
\section{Apparatus}	
	Optical Bench, Red Laser, Photodiode, multimeter, screen, adjustable slit, thin wire, wire holder and mounts

\section{Theory}
	In optics, the \emph{Fraunhofer diffraction} equation is used to model the diffraction of waves when the diffraction pattern is viewed at a long distance form the diffracting object (and also when it's viewed at the focal plane of an imaging lens) \footnote{Taken from Wikipedia}
	\par
	TODO: Complete this section	

\section{Procedure}
	\begin{enumerate}
		\item The laser was left switched on for 15 minutes so that light intensity from laser did not flicker.
		\item The slit was mounted on the bench so that the diffraction pattern could be seen on the white screen.
		\item The slit size and distance from laser were varied and observations were noted.
		\item The photocell was mounted onto the stand between the slit and the white screen.
		\item A graph between the photo-current and position of the diode was plotted.
		\item The distance D between the slit and photocell was measured.
		\item The distance x from the centre of diffraction pattern to the first minimum was obtained and the slit width was calculated (as explained in the theory)
		\item The expected theoretical values and the observed values were compared.
		\item Lastly, a wire was mounted on the optical bench. The wire and laser sources were moved until the diffraction pattern was visible through the wire. The aperture dimensions were measured by a travelling microscope and the thickness of the wire was found.  
	\end{enumerate}

\section{Observations and Calculations}	
	\subsection{Thin Slit}
	The readings were taken and are appended to the experiment. The first column is relative position of the sensor with respect to the observed maximum peak, given in millimetres. The second column shows the value of current through the sensor, in $10^-3 \mu A$. The last column contains the theoretically expected intensity of light (which should be proportional to the current in the previous column) calculated for each relative position in the first column. The theoretical value was evaluated using
	\begin{equation}
		I(\theta) = I_0(\frac{\sin{\frac{\beta}{2}}}{\frac{\beta}{2}})^2
	\end{equation}
\section{Result}
	Wavelength of Sodium light ($\lambda$) was experimentally found to be ($d\beta /D$) = $617.3 nm \pm 0.85\% = 617.3 \pm 5.2 nm$

	\begin{table}
		\myfloatalign
		\begin{tabularx}{\textwidth}{Xllll}
			\hline
			\tableheadline{Serial} 	&	\tableheadline{Initial Screw Position} & \tableheadline{Final Screw Position} & \tableheadline{Fringe Width}\\
			\hline
				1	&	0.17	&	0.32	&	0.15\\
				2	&	0.32	&	0.32	&	0.15\\
				3	&	0.47	&	0.63	&	0.16\\
				4	&	0.63	&	0.84	&	0.21\\
				5	&	0.84	&	0.97	&	0.13\\
				6	&	0.97	&	1.15	&	0.18\\
				7	&	1.15	&	1.29	&	0.14\\
				8	&	1.29	&	1.42	&	0.13\\
				9	&	1.42	&	1.60	&	0.18\\
				10	&	1.60	&	1.71	&	0.11\\
				11	&	1.71	&	1.80	&	0.09\\
				12	&	1.80	&	2.02	&	0.22\\
				13	&	2.02	&	2.16	&	0.14\\
				14	&	2.16	&	2.31	&	0.15\\
				15	&	2.31	&	2.43	&	0.12\\
				16	&	2.43	&	2.57	&	0.14\\
				17	&	2.57	&	2.71	&	0.14\\
				18	&	2.71	&	2.89	&	0.18\\
				19	&	2.89	&	3.04	&	0.15\\
			\hline
		\end{tabularx}
		\caption{Fringe Width Observations}
		\label{E2_observations}
	\end{table}

\section{Precautions}
	\begin{enumerate}
		\item The surfaces of the optical parts should be wiped properly to obtain clear images with good contrast.
		\item Micrometer should be moved only in one direction to avoid errors due to backlash.
		\item The brightness of the source and width of the mean can be adjusted separately using aperture and slit width
		\item Do not make the distance between the lens and slit too high, else the two positions of sharp images using the lens will not be obtained.
	\end{enumerate}