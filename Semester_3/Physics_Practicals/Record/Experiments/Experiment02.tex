%************************************************
\chapter{Frenel's Biprism}
%************************************************
\begin{flushright}
September 11, 2012
\end{flushright}
\section{Aim}
	To determine the wavelength of sodium light by setting up double slit type fringes using Frenel's Biprism.
\section{Apparatus}	
	Sodium vapour lamp, Aperture, Moveable Adjustable Slit, Moveable Eye Piece with a micrometer, Frenel Biprism

\section{Theory}
	The idea here is to setup interference using a biprims. Light emerges from a near point source and hits the biprism, one of whose angles is approximately $180^{0}$, and others $~30''$. The beam splits into two and these seem to appear from two sources (`seem to', thus virtual sources). Since the light was split from a common source, it's phase locked (coharent). The light is also monochromatic as its coming from the sodium lamp. Thus, interference patterns of dark and light band are obtained. The shape is attributed to the geometry of the prism.
	\par
	The fringe width $\beta = D \lambda / d$, where $D$ is the distance between the slit and the eye-piece, $d$ is the distance between the two virtual sources and $\lambda$ = wavelength of the source. Here $\beta$ and $D$ can be readily observed, directly. For measurement of $d$ we use a method which is not immediately obvious.


\section{Procedure}
	\begin{enumerate}
		\item Sodium lamp was turned on, since it usually takes some time to produce a bright enough light.
		\item The micrometer was calibrated by measuring the lateral displacement in effect with the rotation. \label{E2_calibrated}
		\item The aperture, the biprism, the slit and the eye peice were all vertically aligned, by shifting all of them very close.
		\item The Biprism was rotated using the tangential screws, to make it parallel to the slit.
		\item While looking through the eye peice, the source of light, the distances were suitably increased.
		\item Fringe width was measured for 20 consecutive fringes using the micrometer, which was calibrated in \autoref{E2_calibrated}
		\item Distance between the eye piece and the slit was measured for the configuration, using the marks on the `track'
		\item Without disturbing the configuration, a convex lens was placed between the biprism and the eye-piece. The lens was moved to obtain a sharp image, which ideally should be obtained at two locations. The distance between the beams was noted using the micrometer for both cases and their geometric mean taken.
	\end{enumerate}

\section{Observations and Calculations}	
	One rotation of the screw of the micrometer was found equivalent to $\frac{10}{16}$ $mm$ lateral displacement, in accordance with \label{E2_cal}.\\

	\begin{table}
		\myfloatalign
		\begin{tabularx}{\textwidth}{Xll}
			\hline
			\tableheadline{Rotations of the Micrometer Screw} 	&	\tableheadline{Lateral Displacement ($mm$)}\\
			\hline
				4	&	2.5\\
				8	&	5.0\\
				12	&	7.5\\
				16	&	10.0\\
			\hline
		\end{tabularx}
		\caption{Calibration of the Micrometer}
		\label{E2_cal}
	\end{table}

	Least count of the micrometer $=\frac{2.5}{400} mm$ since 4 rotations are equivalent to $2.5$ $mm$ and each revolution can be resolved into 100 parts.
	\par
	The mean width $=2.87$ rotations $=0.094 mm \pm 0.00625 mm$, using \autoref{E2_observations}\footnote{Numbers given are rotations of the screw of the Micrometer of the eye-piece, where 1 corresponds to $180^{0}$ rotation.}\\
	
	$D=74.0 \pm 0.05 cm$ \\
	$d_{1}=8.13 \pm 0.00625 cm$ (the distance between the virtual sources in the first clear image)\\
	$d_{2}=2.90 \pm 0.00625 cm$ (the distance between the virtual sources in the second clear image)\\
	$d=$ Geometric Mean of $(d_{1},d_{2})$ = $4.86 \pm 0.00625 mm$.
	\par	

\section{Result}
	Wavelength of Sodium light ($\lambda$) was experimentally found to be ($d\beta /D$) = $617.3 nm \pm 0.85\% = 617.3 \pm 5.2 nm$

	\begin{table}
		\myfloatalign
		\begin{tabularx}{\textwidth}{Xllll}
			\hline
			\tableheadline{Serial} 	&	\tableheadline{Initial Screw Position} & \tableheadline{Final Screw Position} & \tableheadline{Fringe Width}\\
			\hline
				1	&	0.17	&	0.32	&	0.15\\
				2	&	0.32	&	0.32	&	0.15\\
				3	&	0.47	&	0.63	&	0.16\\
				4	&	0.63	&	0.84	&	0.21\\
				5	&	0.84	&	0.97	&	0.13\\
				6	&	0.97	&	1.15	&	0.18\\
				7	&	1.15	&	1.29	&	0.14\\
				8	&	1.29	&	1.42	&	0.13\\
				9	&	1.42	&	1.60	&	0.18\\
				10	&	1.60	&	1.71	&	0.11\\
				11	&	1.71	&	1.80	&	0.09\\
				12	&	1.80	&	2.02	&	0.22\\
				13	&	2.02	&	2.16	&	0.14\\
				14	&	2.16	&	2.31	&	0.15\\
				15	&	2.31	&	2.43	&	0.12\\
				16	&	2.43	&	2.57	&	0.14\\
				17	&	2.57	&	2.71	&	0.14\\
				18	&	2.71	&	2.89	&	0.18\\
				19	&	2.89	&	3.04	&	0.15\\
			\hline
		\end{tabularx}
		\caption{Fringe Width Observations}
		\label{E2_observations}
	\end{table}

\section{Precautions}
	\begin{enumerate}
		\item The surfaces of the optical parts should be wiped properly to obtain clear images with good contrast.
		\item Micrometer should be moved only in one direction to avoid errors due to backlash.
		\item The brightness of the source and width of the mean can be adjusted separately using aperture and slit width
		\item Do not make the distance between the lens and slit too high, else the two positions of sharp images using the lens will not be obtained.
	\end{enumerate}