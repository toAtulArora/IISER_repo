%************************************************
\chapter{Introductory Session}\label{ch:introduction}
%************************************************
This bundle for \LaTeX\ has two goals:
\Bart

\section{Organization}
A very important factor for successful thesis writing is the


\section{Style Options}\label{sec:options}
There are a couple of options for \texttt{classicthesis.sty} that
allow for a bit of freedom concerning the layout:
\marginpar{\dots or your supervisor might use the margins for some
    comments of her own while reading.}

Many other customizations in \texttt{classicthesis-config.tex} are
possible, but you should be careful making changes there, since some
changes could cause errors.

Finally, changes can be made in the file \texttt{classicthesis.sty},%
\marginpar{Modifications in \texttt{classicthesis.sty}%
} although this is mostly not designed for user customization. The
main change that might be made here is the text-block size, for example,
to get longer lines of text.


\section{Issues}\label{sec:issues}
This section will list some information about problems using

\subsection*{Compatibility with the \texttt{glossaries} Package}
If you want to use the \texttt{glossaries} package, take care of loading it 
with the following options:
\begin{verbatim}
	\usepackage[style=long,nolist]{glossaries}
\end{verbatim}
Thanks to Sven Staehs for this information. 


\subsection*{Compatibility with the (Spanish) \texttt{babel} Package}
Spanish languages need an extra option in order to work with this template:


\subsection*{Compatibility with the \texttt{pdfsync} Package}
Using the \texttt{pdfsync} package leads to linebreaking problems with the \texttt{graffito} command 

\section{Future Work}
So far, this is a quite stable version that served a couple of people


\section{Beyond a Thesis}
It is easy to use the layout of \texttt{classicthesis.sty} without the


\section{License}
\paragraph{GNU General Public License:} This program is free software;
