%************************************************
\chapter{Chemistry Notes}
%************************************************
\begin{flushright}
October 19, 2012
\end{flushright}
\section{Peak Broadening}
	\subsection{Natural (Lifetime)}
		$\Delta E \tau = h/2\pi$ \\
		$h \Delta \nu \tau = h/2\pi$

	\subsubsection{Collisions}
		Collisions can cause the lifetime to reduce. Some gasses will collide and go away, but in some, as you increase the pressure, the width will increase, and the intensity will decrease simply because a lot of electrons are unable to emit. It's the area under the peak that's constant, not the peak. Not only is the width increase, also there's a loss of intensity for reasons stated earlier.
		\par
		But the average lifetime decreases. So remove other gasses and reduce the pressure, you're suppressing collisional broadening.
		\par
		Every other broadening mechanism when removed, then you're left with the natural width of the peak. Collisions are inelastic, because its not just change of kinetic energy, because there's a change in potential energy (How?)
		\par
		Its not dropping the intensity, there's no physical way you can understand that, but you need to understand FT. Increase in energy, causes increase in temperature also (Biplob Question)
		\par
		Temperature rise is not like you can boil water on it. That class of spectroscopy is called photo thermal spectroscopy. Also called thermal lensing, photo acoustic. What they do is measure the heat (not certain about this..)
	\subsubsection{Doppler}
		Molecular motion that's always happening. By maxwell boltzman distribution, you have the nice bell shaped curve. When you increase the temperature, the same thing just becomes wider. Then there's a greater distribution of speeds. Classic example of the fact that frequencies observed are different (speed in this case) for a moving observer (or source), is a train sound (car sound). If the sodium is moving towards a source of 590 nm, then the sodium atom would not absorb because for the sodium atom, the wavelength is less than 590 nm. So if the source emits various wavelengths, then the sodium atom would absorb at different wavelengths because of doppler effects and this causes peak broadening called doppler broadening.
		\par
		If I make liquid sodium or solid sodium, then that will add to complications. So you must cool it without solidifying or liquefying it. 10 K everything is solid, except helium etc., but you must cool these without them interacting with each other.
		\par
		Need to know mean free path, pressure etc. to calculate things.
		\par 
		Doppler shift you know quantitatively. If you put it in the Maxwell Boltzman distribution, you get the width of the doppler broadening (don't memorize this) is given by $\Delta \nu_D =7x10^{-7}\nu \sqrt{\frac{T}{M}} $. This explains why Raman has a huge doppler, therefore the peaks are so very broad. Plus you have a root of T. Higher the molecular weight, the distribution gets smaller, therefore the doppler width becomes smaller. (not certain)
		\par
		When you're stuck with doppler broadening, you've to cool the guy (because for a given technique your frequency is fixed), without freezing.
		\par
		One method is to put the water in an Argon matrix (Biplob question) and this will be cooled. Also you can use the super cooling thing sir had talked about in one of his talks (figure out the precise name of the technique). Adsorption (Prashansa Question), the doppler's then gone, but the interaction with solid mustn't be strong otherwise pointless. But it could be used, take graphite its not that bad. But others like magnesiume etc. can NOT be used.
	\subsubsection{Others}
		Electric Field, purturbation
		Time of Flight
		Pressure Broadening

%After this basically 
	