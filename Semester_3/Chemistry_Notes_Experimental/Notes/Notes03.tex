	Single Vibronic level: Provided the molecule doesn't have any way to dispose of energy. No collision channels (low pressure) can make this achievable. You go up and you come back to the same place. 

	Irrespective of the wavelength you get the same spectrum. This was because the molecules come to the ground vibrational state in the same electronic state, and the emission is therefore the same for different incident frequencies.

	The wavelength of incident will always be smaller than the emitted (because incident energy is greater, thus smaller wavelength, is used as vibrational relaxation)

	If you take perilene?, it doesn't jump from here to there. In some it does jump. Question was, if there's another electronic level near the higher electronic leve.

	Faster guy will dominate the whole scenario. 

	Collisional timing experiments

	So if you put soemthign in a higher energy level (both electronic and vibrational), then if it can come down faster because of collisions etc., you won't see the emission because before they can emit, they would've come to the ground vibrational level.	

	Stokes found: the emission was always longer than what he put in. Fluorescence is always shift in longer wavelength.

	Kasha mirror image rule: For every one molecule you can show me that follows the mirror image rule, we can find one that doesn't

	Flouroscence: Photon emitted by a molecule going from a state to a lower state.


	In vibrations there's no light coming out, because collisions dominate

	Flouroscence: Any excited level, going to a lower level, by emission of a photon.

	Protein can grab molecules and thus introduce a stokes shift, and then when the protein opens up, the flourescence is lost.

	Ultra Trace Analysis: 2.5 parts per billion regime (2.5 nanomolar), the experiment we performed today

	Biggest service chemistry has done to science, Chemistry does characterisation for you. 

	$UO_2^{2+}$, 100 ppm and higher
	put this in phospheric acid $H_3PO_4$, straight away you go to 100 ppb. The phosphate, the metals can form ligands, it can attach itself to the uranil. Water collides and takes away the possible emission.
	Detection limit:
	With phospheric acid 

	Steady state experiment
	Xenon arc lamp (high pressure): Two electrodes in high pressure, it's a continuum, there's pressure broadening. 200 to 600 nm in a continuum is required, because otherwise, the lines would correspond be xenon's electronic levels. Collisional broadening is different from pressure broadening, therefore perturbations cause alteration of wavefuntions. 

	You'll never get a source with continuous intensity at every wavelength.

	Intensity based measurements (with respect to wavelengths)

	Instrument response function can be used to correct for inaccuracies. (even the grating has wavelength dependent intensity sorting)

	Continuous source lamp and therefore you recorded the system over time

	How could I measure a 10 nano second decay?

	I need a time resolved experiment

	If i give a pulse of light, the light is on only for one nanosecond. Something that keeps it flickering, on for 1 ns and off for the rest of the time.

	The intensity as a function of time will decay as a function of time. Assuming a first order kinetics, you get an exponential with 
	$I=I_0e^{-t/\tau}$.

	Life time becomes shorter, with collisions Potassium iodide in milimolar with flouroscene and collional quenching will kill the flourescence

	Simply because the quenching rate has dominated the flourescence rate, and therefore, the quenching concentration when it increases, the floursences 

	Stern Volmer
	$F_o / F = 1 + K_{sv}[Q]$
	$F_o / F = 1 + K_qT[Q]$

	Quenching efficiency, and \tau, its the normal.

	Quenching your flouroscence can happen and give you compeltely wrong.

	Meauser the life time of flouroscene

	the liftime would be that of a pure flouroscene

	and this drop in lifetime, will result in 

	the ratio of lifetime will tell you the quenching factor, now you can go back and multiply this with the value of concentration you obtain from the original calibration curve.
