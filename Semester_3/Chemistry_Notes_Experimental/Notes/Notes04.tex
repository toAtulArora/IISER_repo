%************************************************
\chapter{NMR and similar}
%************************************************
\begin{flushright}
Novemeber 7, 2012
\end{flushright}
\section{Introduction}
	\subsection{Splitting the degeneracy}
		You can't calculate the $g$ value for nucleus so easily, as compared to for electrons. $h$ you must always know. $g$ and $\beta_n$ and $h$ should be known, $\beta_z$ must be known. In a way spin of the electron and spin of the proton, is very similar. So whatever you learn for a proton, you can apply the same for an electron. Shell structure for electron is very simple, but for a proton is not very simple, but there's a complicated shell structure. $\nu=|g\beta_n\beta_z|$. In vibrations, your energy spacing becomes very high, much greater than $KT$, and thus the population difference is very large. Almost equal population in this case, because the energy difference is very small. The population difference is only in ppm. If you apply a stronger magnetic field, the frequency corresponding to the separation becomes much larger.
		Srijit's doubt: The only way I can have single state, 2S + 1 =1, then S has to be zero.
		There's no bare proton. If I take a hydrogen atom, the electrons would also respond to the spinning charge. They would respond differently than the proton. They would try to oppose the field. So say if I apply 2 Tesla, then the proton feels 1.998, the field is not what you've applied, but somewhat less than that. Total angular momentum is what we're talking about.
		You apply the field, and you get to two levels, then its not the field you applied, its much smaller, I shouldn't say much smaller, its smaller.
		$B_{actual}=B_z-B_{induced}$
		$B_{induced}=\sigma B_z$
		so then $B_actual=B_z(1-\sigma)$, where $\sigma$ is the electron density.
		$CH_3-CBrH-CH_2Cl$, so the most electronegative $Cl$, will suck away the electron density, thus the proton will experience the least $B_{induced}$. The next will be the proton near $Br$, where you will have a slightly higher $B_{induced}$ and the methyl will have the max $B_{induced}$.
		$CH_3-OH$, acidity is manifested in the levels. The one experiencing the least shielding will talk first. This depends on $\sigma$.
		So if there are two guys, with $~2.3T$ and $~4.6T$, with 100 MHz and 200 MHz, the difference would be in the ratio of 1:2. Brooker, or Glassgow, for UV its the same.
		So for NMR also we want a standard we can use. So we use the difference and divide it by the frequency of the machine that you're using. 
		SO you go as $\frac{100Hz}{100MHz}=1x10^{-6}=1ppm$ and for the other machine also you'll have $\frac{200Hz}{200MHz}=1x10^{-6}=1ppm$. The actual frequency difference will be different but their difference normalized will be different. I will give you exercises where you calculate and convince yourself its accurate. 
		In reality, I use a reference compound. So in my sample, I add a reference compound, which will give me a known peak at a given magnetic field. So lets say the difference in frequency is $\Delta \nu=100 Hz/100 MHz=1ppm$. So I report the peak appears 1 ppm away from the beginning. We use TMS, tetra methyl selane. It is nice because, Silicon is electropositive, so it's protons will be highly induced and therefore the peak would be very far away from the usual organic compounds. So this will show up at the highest magnetic field! Lithium Hydride say for instance, the peak will show up even more right on the spectrum. By definition, TMS would be at zero ppm. We had a Tau scale now it's Delta scales. High field machines have a greater split. So high field means high resolution. Therefore in high machines, you have these peaks more spread on the frequency scale.
		Also one peak is given. Volatile also, just heat the mixture and you could recover the analyte. Also has 12 protons, and so it'll have a large peak and would be soluble in it. $CDCl_3$ is the solute. Peak broadening, with increase in intensity, so if you take chloroform, it will swamp all the spectra because it'll sit right in the centre.
		If you had to do an experiment where TMS is not soluble, then you could use a different method.
		Then sir summarized the whole thing. We will harp on a little more and then we'll see the second order effects. You also get information about it's neighbours.
\begin{flushright}
Novemeber 8, 2012
\end{flushright}
		If the lower had $1$, then the upper would have $0.99998$, so the ratio would be $0.99998/1.99998$. Do the assignments twice. But you sit down to doing it, you'll find there're many places where you can make mistakes and therefore should practice instead of losing time in the exam.

		The spacing is so very small for carbon, that its difficult to observe. In the case of carbon, each carbon has 6 protons and 6 neutrons, and in such cases, the spins cancel out. For $C^13$ though, its not NMR inactive. g is intrinsic to the particle, doesn't depend on magnetic fields. You can't quite guess g anyway. 

		TMS concept was reiterated. $B_{actual}=B_0(1-\sigma)$, so the difference depends on the value of $B_0$. Now say in a 100 MHz machine you got a difference of $326x10^{-6}T$.  Now we know that 2.3487 T. Gradients had no concern at the times of $60 MHz$ machines. There're times when you're not clear about weather you're clear or not. Good part of the questions will be from what i teach you in this module.

		Splits will not depend on the magnetic field. The splitting patterns remain the same.
		Hydrogen talks in NMR, and the spin comes in the picture, becaues there's a resultant angular momentum and therefore it has a magnetic field. $^{12}C$ has 6 protons and 6 neutrons. They pair up and there's no angular momentum they posses. Its nice that carbon decided not to talk, therefore it's cool. 
		$^1H$ 1p/0n Odd Z, even N
		$^{13}C$ 6p/7n, Even Z, odd N I=1/2
		$^{12}C$ 6p/7n, Even Z, odd N I=1/2
		$^2H$ 1p/1n, Odd Z, odd N, I=1 (integer)

		I=1, splits into a triplet, it goes 1, 0 and -1. Selection rules (find out)

		Carbon 13 is very small statistically, so organic chemists would start using Carbon 13 as labels.

		Area under the curve would be proportional to the number of protons

\begin{flushright}
November 14, 2012
\end{flushright}
		$CH_3-CH_2-OH$, now for $CH_2$, there are two sets of inequivalent, then couple the strong ones, and then couple the weaker ones. So if $CH_3$ has a stronger spin, you'll first split into four, and then split into two.
		If you have more s character int he bond, then the electrons will be closer to the nucleus, and therefore the natureo f the bond decides the coupling.		
		In 2(broomomethyl)1-Chloro-3iodo propane, you see triplet of a triplet of a triplet.
		\marginpar{\Bart If you want to hit on a geeky girl, try: We're degenerate nuclii; we couple but don't split}
		Double radiation experiment, with carbon being only 1 percent in natural abundance. They're very small in number the abundance is very little. Carbon 12 is silent in NMR. We know how to handle that, because we use Fourier transform NMR, so the sensitivity will be high enough. If you have $R_3CH$, you'll get one peak for the tertiary carbon and that'll be split into two. If the splitting is too much, then it might just get lost in the noise, since the intensity has to get distributed, which is anyway too small. Now to avoid splitting, you give the protons a radiation of 1400 MHz, the proton levels get saturated, basically a dynamic equilibrium is reached between the population of the two states. Since this is very rapid, the carbon sees only an average, therefore splitting doesn't happen. \marginpar{\Lisa Check out Einstein's equation about rate of stimulated emission and that of absorption} Don't have to worry about C 13 when you're doing hydrogen because C 13 is very small in number. Also, apply the broadband pulse because the protons get split at slightly different levels.
		Relaxation mechanisms are rapid, the state is restored because of environmental, lattice etc. So if you put a molecule in a higher state, it comes down to a lower state. The absorptions will depend on the population in the ground state, so if the relaxation is faster, then the peak will be intenser. For a proton, the relaxation time scales are the same. So from the peak, you can predict the number of protons. For carbon though, they're not the same, so you can't quite tell the number of carbons.
\section{Amazing examples}
		$CH_3 ~ 0.9$, $C=-C-H ~ 2$ $C=CHR ~ 4$, $Benzene ~ 6$. With the benzene

\begin{flushright}
November 21, 2012
\end{flushright}
\section{I came late}
	$E=g\beta_NB_zI_z$ for a proton, for an electrons we have $E=gg\beta_lB_zJ_z$. If i want to calculate the $g$ value of an electron we have 
	\begin{equation}
		g=\frac{3}{2} + \frac{S(S+1)-L(L+1)}{2J(J+1)}
	\end{equation}
	For a $^2S_\frac{1}{2}$, we have 
	\begin{equation}
		g=\frac{3}{2} + \frac{1/2(1/2 + 1)}{2(1/2)(1/2 + 1)}
	\end{equation}
	For a $^2P_\frac{1}{2}$, we'll have $L=1$, we plug it in, to get $\frac{2}{3}$.
	Splitting the total Js into it's corresponding components. For atoms, L and S give me J, for molecules, we add the molecular rotation, so with the nuclear spin, we have to add that too.
	We then get,
	\par
		$J+I .... J-I$, so for $^2S_\frac{1}{2}$, we'll have $1/2 + 1/2 = 1$, and $1/2 - 1/2 = 0$, so you get a slit, which is about $21 cm$, which is what revolutionized astronomy. The population difference in normal temperatures is zero, but in inter-galactic spaces, the temperature is so very low, the population difference is significant.

		\marginpar{\Bart I was complaining about it being cold in Delhi and was talking to my colleague, he's like you shouldn't complain about the temperature you do cold molecule spectroscopy and I was like I like my molecules to be cold, not MY molecules to be cold}

	\par
	For the usual states, you use the schrodinger's equation to solve. Now you use approximation methods, with certain perturbations. Thus you can't use very strong fields, because then the quantum mechanics you use to predict things, won't be valid.

\section{ESR}
	Here you end up using 0.3 Tesla and so on. It'll be in the giga hertz region, the frequencies. Simply because the nuclear magneton (magneton is inversely proportional to the mass) is 1000 times less than that of the electron. The electrons are so very mobile, that their lifetimes are very small, thus the peak is very broad compared to NMR, in which the nucleus takes a long time to get restored.
	In NMR you don't want any unpaired electrons, no paramagnetic material. So the electron can come really close to the nucleus and lose energy, thus the time periods are small, thus it's broad, the line width. We normally plot the derivative, since the peaks are very broad. Also, the electron couples with the nucleus, and the peak splits into two. 
	\par
	You keep the frequency constant, (4600 MHz), and sweep the magnetic field, you'll get the spacing between the frequency as 0.05 for 0.32 T.
	For the splitting, the usual ways go. (missed out a bunch of examples, specially naphthalene, with pentant of a pentate). 
	\begin{equation}
		a = R\rho
	\end{equation}
	So for a hydrogen atom, if the value is 0.05 T, then for a methyl ion, the value will be smaller because the electron spends smaller time with each nucleus, as the time is spent on various atoms. For the hydrogen, the split is max because the electron doesn't have anywhere to go. For the methyl radical, it'll be smaller (viz. 2.3 mT)! For benzene it becomes 0.38 mT.
\begin{flushright}
	November 29, 2012
\end{flushright}

	Chemical shifts, g values
	non-zero spin nuclii, can split an electron
	In ESR, if I take a methyl radical, so then there are four Hydrogens, thus a four split would be obtained, because the coupling is with three, then you get four peaks.
	\par
	In an ESR, the spectrum of the methyl radical, will be a four peak. The width will be the coupling constant. Coupling constants in ESR can be much larger, because in ESR, the electron can get much closer to the nuclii compared to in NMR, where the nuclii are at a constant distance at all times. From the spectrum, thus you would know the electron's delocalization. How else can we find this, we can solve the quantum mechanical model, and using the molecular orbital theory, you can find which atomic orbital makes a larger contribution, and thus will have a higher density near those orbitals.
	\par
	Let's take hydrogen radical, spin half, thus you get two peaks (because one hydrogen splits). Electron density is one, since the electron spends all the time with the hydrogen. $A \alpha \rho$, where A is the width, $\rho$ is the density.
	\par
	Thus you have:
	\begin{equation}
		A=R\rho
	\end{equation}
	\par
	For the methyl radical, the spacing is larger, meaning the electron density on the hydrogen is smaller compared to that of a Hydrogen radical. 2.3 mT is for the hydrogen, and 50 mT for the methyl.
	\begin{equation}
		\frac{2.3 mT}{50 mT}=\rho
	\end{equation}

	If the molecular orbital is coming from s, or p is contributing. Which would interact with the nucleus more, the s of course.

	\par
	If I take the benzene radical, then you get 0.38 mT.
	\par
	Naphthalene: Those which you can swap with a symmetry operation are equivalent. Period. Use that to find sets of equivalent and the splits.
