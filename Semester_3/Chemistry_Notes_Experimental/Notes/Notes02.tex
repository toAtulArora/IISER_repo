%************************************************
\chapter{Removing Degenerecy}
%************************************************
\begin{flushright}
Novemeber 2, 2012
\end{flushright}
\section{Introduction}
	\subsection{Nuclear Spin}
		Electron has a spin half. If $s=1/2$, then length of the vector would be $\sqrt{1/2(1/2 + 1)}h/2\pi$. Electron spins, thus it generates a magnetic moment. Thus you can relate the magnetic moment, to its angular momentum. $\mu $ is proportional to $S$.
		So for a proton, everything would be the same, except the charge. Now $\mu$ is related to $I$, the angular momentum of the charged particle. ($\mu$ is the angular momentum). $\mu=g\beta_NI \,\, JT^{-1}$, $g$ depends on the nucleus. There's no need to remember that number, where $\beta_n=\frac{eh}{4\pi m_p}$. It defines the magnetic moment the angular momentum will produce. The $g$ value is different for Carbon and Hydrogen. Very characteristic of the nucleus you're looking at. That's important because it tells you where it'll lie in the spectrum.
		\par
		To make the plus half \\
		$E=B_z.\mu_z$ \\
		$E=-B_z g\beta_nI_z$ \\
		$\Delta E= gB_z\beta_n$ \\
		The population difference will be as small as 10 in 2 million. When would a molecule absorb? Detour: Lets consider a system in which all atoms are in ground state. When you give energy, the first will go up. So say initially there were ten, now there are nine ground, one excited. Now as the population becomes small, the probability of stimulated emission becomes the same as probability of stimulated absorption and a dynamic equilibrium is reached. Now here, if the population of the excited state becomes high enough, then it will relax back due to quenching and thus the experiment can go on. However, the point anyway is the population difference is very small. When you go to higher fields, then you can more sensitively see better. 9 Tesla for a 400 MHz. 
		In solids, the spectra will get extremely broadened (because life time will become extremely small). Therefore we don't use solids, and don't use gasses because they don't have any quenching. In liquids it's ideal and that's what we used. In gasses, first getting the experiment to work will be more difficult since the number density will be very low.
		\par
		You can always express, $\Delta E$ in Herts, so if $B_z$ is 9, then for Hydrogen, $\nu=400$ MHz. You had to wait for superconductors for getting 9 Tesla. Even today, people who sell us 60 MHz machine, had come.
		\par
		Suppose there's a bare proton, this is what I see (400 MHz). But supposing I take $CH_4$, but it's a proton with electrons surrounding it, and they will change the $\Delta E$. The electrons also have their own intrinsic spin (when the magnetic field is applied), then these two spins will start to cancel, then the number of electrons will decide how much of the magnetic field reaches the proton. So if you apply 9 Tesla, then it'll probably see less field. So they won't absorb at 400 MHz, but it'll instead absorb at less than 400 MHz. Now if you take $CH_3OH$, then the oxygen will take away most of the electrons, thus the H in OH will get a field closer to what you've applied. This is the electrons shielding. Thus NMR is so powerful as it can tell you about the environment around it.
		(Srijit: Wouldn't the protons of other atoms in the molecule interfere?)
		You have two absorptions, thus you can tell the molecule in many cases. With infra-red, it's hard even for the expets to tell which molecule it is. If you pin it to the wall and tell him to figure out the exact molecule, he'll say shoot me.
		So even with a molecular formula, you should be able to tell the structure, using NMR. It's almost always unambiguous. Of course then why aren't we just doing NMR? Well it can't tell you how strong the bond is, the forces etc., that's why you need other techniques

		It's always easy to change the magnetic field, so you start with 9 Tesla, and keep dropping. So the first guy that hits, would be $OH$'s H and then the $CH_3$'s H.		

		Today we of course don't do it this way. FT is used today. Nobody today does sweeping magnetic fields.

		Only if $I\neq0$ do you get something. It's like a closed shell system giving you a singlet sigma system. If you take carbon 12, 6 proton, 6 neutron, thus its NMR inactive. If Odd + Even = 1/2 integral spins.

 %After this basically 
	