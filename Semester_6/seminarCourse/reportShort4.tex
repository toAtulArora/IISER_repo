%% LyX 2.0.6 created this file.  For more info, see http://www.lyx.org/.
%% Do not edit unless you really know what you are doing.
\documentclass[british]{article}
\usepackage{ccfonts}
\renewcommand{\sfdefault}{cmbr}
\renewcommand{\ttdefault}{cmtl}
\renewcommand{\familydefault}{\rmdefault}
\usepackage[T1]{fontenc}
\usepackage[latin9]{inputenc}
\usepackage{babel}
\begin{document}

\title{Seminar Course\\
IDC352}


\author{Atul Singh Aurora\\
MS11003}


\date{Instructor: Prof. Arvind\\
Jan-April 2014}

\maketitle
\pagebreak{}
\begin{abstract}
I had the view earlier that attending seminars is not nearly as important
as spending time learning concepts systematically. In the recent past,
I have been made to realize that too much focusing also is not a good
idea. Thus, the topics that I am interested in but do not intend to
study systematically in the perceivable future, I can learn about
through seminars. The topics maybe sub topics in physics or those
of other disciplines. Making connections this way is helpful and accelerates
learning. So far, it has indeed been the case.

The following discussions have been severely shortened and may therefore
not be suitable for understanding the topics. However, the objective
here is to convey the idea behind the talks with very few words. I
hope in the process clarity hasn't been sacrificed.

\pagebreak{}

\tableofcontents{}
\end{abstract}

\section{Why a Quantum Theory of Gravity}


\subsection{The when, where and who | facts}

The talk was given by Prof. G. Date from IMSc, Chennai, in LH-3 of
the Lecture Hall Complex, IISER M, on Wednesday, March 26, 2014.


\subsection{Motivation for attending | feelings}

Our QFT instructor, Dr. Sudipta, informed us and recommended that
we attend it. That's one. Then I remember reading an eminent scientist
say that we are fortunate to have been born at a time when problems
in physics existed and we didn't have to create them. I feel that
Quantum Theory of Gravity is the one for us. It's even more motivating
to see that there was a time when even STR and QM couldn't be combined
consistently.


\subsection{This I could understand}

The speaker opened with `Quantum Gravity is the fiery marriage of
General Relativity and Quantum Mechanics'. The pursuit he said is
not purely hypothetical, nor a problem of unexplained data. Wheeler's
description of the final state of a star is amongst one of the reasons
why such a theory would be necessary.{*}

\[
R_{\mu\nu}(g)-\frac{1}{2}R(g)g_{\mu\nu}=\frac{8\pi G}{c^{4}}T_{\mu\upsilon}
\]


There've been two revolutions in physics; Relativity and Quantum Mechanics.
In General Relativity the notion of causality gets dynamical and metrical
properties can also change. The combination of special relativity
and Quantum Mechanics is Quantum Field Theory. QFT is consistent with
fixed causality defined by minkowski space-time, but assumes that
causal structures can be defined at arbitrarily small lengths.

Next the speaker talked about the basic assumptions of GR.
\begin{enumerate}
\item space-time is defined by a 4d manifold and its metric $g_{\mu\nu}$
(-ve determinant).
\item Stress tensor, $T_{\mu\nu}$, which describes the matter distribution.
\end{enumerate}
The metric imposes a causal structure on time like events in their
local neighbourhood.

Remarks: Given a metric, at any point, you can define a set of tangent
vectors. These can be classified as timelike, spacelike or null. Corresponding
geodesics determine which evens can influence each other causally.

Einstein's equation is only a local PDE and determines solution space-times
only locally. Extended solutions are not guaranteed to be causal without
imposing extra conditions. Then the speaker talked about predictability,
that is events can be divided into cause and effect.

--

Next he talked about time orientability, strong causality and stable
causality. He stated that causality is not automatic, so time-orientable
space times are a subset of solutions of einstein's equation. An example
is given in the next section. 

Strong Causality: We can impose strong causality else its possible
to have future curves that come arbitrarily close to the existing
events. 

Stable Causality: Imposing stable causality on the other hand would
mean that spacelike and timelike do not interchange.

--

Global Hyperbolicity: space-times which admit spacelike hypersurface
whose domain of dependence is the entire space-time are globally hyperbolic.

Thus, global hyperbolicity automatically ensures causal stability.

--

To have a space-time with predictability, we can construct globally
hyperbolic space-time (take a spacelike hyper surface, define domain
of depndence to be the full space-time, it abruptly ends. You extend
it, how far? maximally possible). However, a predictable space-time
should be an inextendable/globally hyperbolic solution of einstein's
equation.

---

The speaker then went on to describe a slightly technical topic, `Property
of Globally Hyperbolic Space Times'. Consider a space $C(\Sigma,q)$
of causal curves, emanating from a hypersurface $\Sigma$ and passing
through a point $q$ to its future. Its a theorem that a curve of
maximum proper time is a geodesic without any `conjugate point' between
$\Sigma$ and $q$.

---

Next the speaker digressed to talk about a `Bundle of Time like Geodesics'.
Imagine a cloud of particles freely `falling' in a given space time.
These will be described by a bungle of time-like geodesics. Their
cross section in general can undergo 1. Shearing, 2. Twist or 3. Expansion.

---

`Raychaudhuri Equations and Conjugate Points' was discussed next.
This I couldn't understand much however what is known that there arises
a contradiction related to the existence of a singularity which is
resolved by concluding that the geodesic must be incomplete, viz.
the observer's world line must end abruptly. The details have been
skipped.

---

Blackholes and Thermodynamics were discussed next. General relativity
contains another class of intriguing solutions - Black holes - space-times
which have event horizons. A stationary black hole can be described
in terms of parameters like mass ($M$), angular momentum ($J$) and
charge ($Q$). The event horizon is similarly described by the ($A$)
area, ($\Omega$) angular velocity and (Potential $\phi$ electrostatic
potential). They have a surface gravity ($K$) which is a constant
over the horizon. 

If these are disturbed by any astrophysical process, they continue
to remain black hole with changed values of their parameters, such
that
\[
\delta M=\frac{K}{8\pi}\delta A+\Omega\delta J+\phi\delta Q,\,\delta A\geq0
\]


This is a striking similarity to classical thermodynamics and there's
no reason why blackholes should obey this. But continuing the analogy
suggests that surface gavity \textasciitilde{} temperature and area
\textasciitilde{} entropy. So by this analogy we expect that being
a hot body, black-holes should radiate but they don't let anything
escape.

Hawking pointed out that this is true only classically. Because of
quantum fluctuations present n the horizon of the black hole, it has
a temperature and entropy given by 
\[
T=\frac{kGh}{2\pi},\,\, S=\frac{A}{4Gh}
\]


The next part was rather interesting. Since we understand the origin
of entropy from statistics of a finite number of entities, it seems
to suggest that the horizons must have some atomic structure, and
they must be purely geometric objects, viz. the`atoms' must be atoms
of geometry. Maybe space-time geometry is discrete. 

This hints us to the conclusion that perhaps continuum geometry is
inadequate near high curvature regions and that geometry may be discrete.
This begins the framework of quantum. So what would be the goal of
such a framework? It should reproduce the classical space-time approximately
at the bare minimum. Further it should atleast resolve the curvature
singularities and provide an explanation of the blackhole entropy.

---

So where's the problem, well GR has a property of general covariance
which makes it a gauge theory and results in loss of determinism.
Further, if space-time is quantized, then `notion of `dynamics'' is
hard to define, since time evolution needs to be interpreted differently.
Besides, the regimes where quantum rescue is sought, involves highly
dynamical, strong curvatures, where perturbative tools are questionable.

---

The speaker ended the talk with discussing the two types of approaches
taken to solve the problem of quantization. The first is based on
discretizing points of geometry. The second involves promoting space
to be an operator that acts on some state.


\subsection{This I'll remember}

It is possible to construct solutions to Einstein's equations such
that locally they follow all the properties we expect but have peculiar
global behaviour. Imagine making a light cone on a piece of paper
and then folding it to make the future and past light cone match.
Locally, everything's consistent, but globally, you're going to the
past from the future!


\subsection{Acknowledgements}

I thank Ms. Prashansa Gupta (MS11021) for sharing her notes so that
atleast our combined notes made sense.

\pagebreak{}


\section{The Dance Language of the Bees}


\subsection{The when, where and who | facts}

The talk was given by Prof. Raghavendra Gadagkar from IISc, Bangalore,
in LH5 of the LHC, IISER M, on Thursday, March 27 2014.


\subsection{Motivation for attending | feelings}

It was compulsory for us. However, it turned out to be rewarding.


\subsection{A Summary}


\subsubsection{Introduction}

\emph{Apis florea} is the species discussed in the talk. It is found
only in Asia, and is open nesting (thus harder to domesticate). They
are interesting as their social behaviour is not as sophisticated
(compared to the other four major species). They communicate by transferring
saliva and dancing, i.e. subject of the talk.


\subsubsection{About these bees}

Within these bees, there are drones (the males), worker (essentially
sterile) and one queen (that lays eggs). 

The said bees dance only on horizontal surfaces unlike the other species.
There are two types of dances known. First is called a round dance,
in which the bee makes a circular motion. The other's called a waggle
dance in which the bee transcribes an 8 shape with vigorous vibrations
half way.


\subsubsection{Communication and Dance (the real deal)}

Earlier it was thought that waggling related to pollen being found
and round to nectar. Later experiments confirmed that the waggling
dance corresponded to the food being farther than roughly 100 m and
the round to less than that.

Experiments have confirmed that the waggle dance conveys information
about the direction and distance of the resource found. They use the
sun's azimuthal as reference (as it is globally stable, unlike the
surroundings) even though it changes its location periodically. The
bees apparently correct for this. Further they even update their map
of the surrounding everyday and use that to locate the sun. 

In the waggle dance, during the middle part of 8, the direction in
which the bee moves contains the direction information. Also during
this part, the tempo of vibration codes for distance.

Although its not known how they find the angle, the distance is measured
not using the energy requirement as was previously thought, but using
optical flow.


\subsubsection{Sophistication}

These bees have evolved to use fallback mechanisms for defining the
angle.
\begin{enumerate}
\item Sun is used when its bright enough
\item Polarization of light on the sky is used if the sky is overcast
\item Local landmarks can also be used. They learn it each day to compensate
for their change with time.
\item Siesta; when the sun is overhead, they take time off
\end{enumerate}

\subsubsection{Experiments}

Various experiments were done to test color vision, co-relation between
sugar/nectar and waggle/round dance, co-relation between wiggle and
flight direction (Fan like experiment), that optical information is
used to find direction, use of Sun as reference (The lamp shift experiment
| miscommunication) and if dancing is sufficient to communicate information
about resources (The robot bee experiment).

\pagebreak{}


\section{The Climate Energy Nexus}


\subsection{The when, where and who | facts}

This talk was delivered by Dr. S K Tandon, IIT Kanpur, on Wednesday,
April 16, 2014, at 6 PM, in LH5, LHC.


\subsection{Motivation for attending | feelings}

Frankly, I don't know and I regret attending it. Except that it helps
in the counting to 6.


\subsection{Summary}

The talk was started by display of the greenhouse gasses data spanning
the last 100 years. Some key estimates of CO\_2 concentrations are
as follows
\begin{itemize}
\item 180 ppm | glacial (cold) period
\item 300 ppm | interglacial warm period
\item 280 ppm | 18th century, Industrial Revolution
\item 380 ppm | transportation revolution leading to urbanization
\end{itemize}
The speaker then discussed some of the techniques used for arriving
at such conclusions
\begin{itemize}
\item Continental Ice sheets (also known as Climate Archive)

\begin{itemize}
\item Cryosphere's annual layars: These are counted to establish the `age'
\item Molten Ice: This is useful for identifying liquids present at the
time
\item Ice Cracks: They are the molten ice for gas
\item Scintering: Scaling of air bubbles is used
\end{itemize}
\item Annual Rings
\item Radioactive
\item Age Modelling
\end{itemize}
Next an attempt was made to relate CO\_2 levels and human CO\_2 emissions.
The speaker claimed that a Nexus of sorts is formed with the following
components:\\
Water, food, population control, health, energy, climate change, environment
sanity, knowledge equity, sustainability and terrorism.

Next the speaker described the earth system as composed of
\begin{itemize}
\item Eco-sphere
\item Human Factors

\begin{itemize}
\item Anthroposphere: Aggregate of human activites
\item Metaphysical: Its a global subject and includes adoption of say international
protocols
\end{itemize}
\end{itemize}
Coming to the most important point of the talk, the speaker justified
how 0.6 degrees of change in temperature is unaccounted for, despite
the removal of all known possible natural causes, including solar
heating (primarily, believed to contribute \textasciitilde{}0.2 degrees),
orbit cooling and millenium warming. Thus by elimination, the only
cause of the observed effects is anthropogenic

The speaker went on to say how the climate models in the past relied
only on factors such as atmosphere, land and ocean while today additionally,
the following are the base standard
\begin{enumerate}
\item Arosols
\item Carbon Cycles
\item Dynamic Vegetation
\item Atmospheric Chemistry
\item Land Ice
\end{enumerate}
He concluded that therefore today we are justified at having more
confidence on these.

In conclusion the speaker pointed out that there's an urgent demand
for affordable, reliable energy (it is he claimed pivotal to all our
problems). He ended on optimistic terms, discussing two examples of
possible solutions.
\begin{itemize}
\item Carbon Capture (involves capturing, liquefying, transporting and releasing
the gas) {[}successful examples are known, although not precisely
for environmental reasons{]}
\item Shail Oil (the extraction technology is still in its infancy but the
reserves are promising)
\end{itemize}

\subsection{This I'll remember}

There's something called shail oil, which will last us a very long
time, but we haven't the technology to extract it yet.


\subsection{Acknowledgements}

I was unable to make proper notes for this seminar and Ms. Prashansa
Gupta (MS11021) was kind enough to share hers.

\pagebreak{}


\section{The Geometry of Physics}


\subsection{The when, where and who | facts}

Mr. Rahul Chajwa, physics major (MS10 batch), spoke on the said topic
on Wednesday, April 16, 2014, at 9 PM in LH3 of the LHC.


\subsection{Motivation for attending | feelings}

It was compulsory to attend atleast 2 in this series, however it turned
out to be rather informative and interesting.


\subsection{What I understood}

The speaker in essence talked about how one would characterize various
physical systems using geometry. He started by asking the question,
how one would characterize surfaces, after showing with a series of
figures. One way of such characterization he said was counting the
number of holes. He then showed how various surfaces like a cup, purse
etc. may deformed to a surface having clearly one or two (or even
no) holes. He went on to explain why such a characterization, though
intuitive is not always easy. He talked about Euler's polyhedron formula
and the Gaus Bounet Theorem. He attempted to show some generalizations
to higher dimensions, including the concept of derivatives. Specifically,
he talked about Liuovilles Theorem and discussed in some detail the
tangent space and its dual, co-tangent space. The fascinating part
of the talk was related to the double pendulum which we know is chaotic.
He explicitly showed an animation proving his point. He then asked
us what geometry we could associate with the co-ordinate phase space
of the setup. The answer was a torus. He then stated a result which
follows from the geometric analysis that the hamilton's equations
result in contraining the movement along geodescics of the torus geometry.
This has a profound implication, the resulting motion will be periodic
since geodesics are closed. This is almost impossible to see without
the geometric considerations since we don't even have an analytic
solution of the motion. Finally he demonstrated Euler's disc and talked
about the infinite frequency before collapsing and discussed why this
singularity is not a consequence of hamiltonian evolution but instead
that of dissipation. He concluded the discussion by attempting to
convince the audience that we can't think of logic without invoking
some sort of object (geometry) in our heads.

The speaker received criticism for not acknowledging the sources referred
to, and for shoddy board work from the instructor.


\subsection{This I'll remember}

A double pendulum's (co-ordinate) phase space can be thought of as
the points on the surface of a torus.


\subsection{Acknowledgements}

I thank Mr. Vivek Sagar (MS11017) for reminding me about the seminar,
else it would've been missed by me.

\pagebreak{}


\section{Magnetic Refrigeration}


\subsection{The when, where and who | facts}

This talk was delivered by Ms. Ayushi Singhania (Int PhD), on Wednesday,
April 23, 2014, at 9 PM in LH3 of the LHC.


\subsection{Motivation for attending}

The topic sounded like fun.


\subsection{What I could follow}

The speaker started with outlining her talk; introduction, history,
magneto caloric effect, thermodynamics, advantages/disadvantages.
She explained that the technology is based in magneto caloric effects
(MCE). This was discovered by P. Weiss and it was suggested by P.
Debey and W. Giaque. The MCE is based in changing the temperature
of specific materials, by exposing it to changing magnetic fields.
The phenomenon hinges on the idea that at a given temperature, the
magnetic field causes alignment and when it is removed, the disorientation
leads to dropping of the temperature (assuming there's no heat exchange).
This has similarities with the Curie temperature which has electric
fields and ferroelectric domains as analogues to their magnetic counterparts.

The speaker motivated the functioning further by explain how thermal
and magnetic entropy together function to drop the temperature. She
broke down the process into four steps. In the first (adiabatic),
magnetic field is increased. Since there's no exchange of heat, the
net entropy is zero; the magnetic entropy drops and thus the temperature
increases. In the next step, a liquid is passed that cools the system,
taking away the heating that resulted. In the third step, the adiabatic
demagnetization, magnetic entropy is increased, while the thermal
entropy is decreased. This step leads to cooling! In the final step,
whatever is intended to be cooled is passed through and the process
is repeated.

The speaker also did some calculations on the board to arrive at an
expression for change in temperature, in terms of the system's parameters.
\[
\partial T=-\frac{\mu_{0}T}{C_{H}}\left(\frac{\partial M}{\partial T}\right)_{H}dH
\]


Candidate and prevalent working material for the refrigerator were
discussed next which included alloys of gadolinium which are known
to produce 3-4 K change, per Tesla of change in magnetic field. 

The speaker concluded by drawing a parallel between magnetic and conventional
refrigeration, pointing out the role of pressure being played by the
magnetic field in the said case. She discussed some advantages, such
as environment friendliness and high efficiency (60-70\%) while also
listing some drawbacks such as its high cost and requirement of rare
earth materials.

The speaker was appreciated for her clarity and neat board work by
the instructor.


\subsection{Acknowledgements}

I referred to the wikipedia article on the same topic for filling
some gaps.

\pagebreak{}


\section{Cooling by Dilution Refrigerator }


\subsection{The when, where and who | facts}

Mr. Subhendu Shekhar delivered the talk on Friday, April 18, 2014,
in LH3, LHC at 6:00 PM.


\subsection{Motivation for attending}

The talk was on a holiday plus I wanted to wind up the seminar work
by that night. But as usual, it turned out to be informative and the
principle discussed was rather interesting.


\subsection{What I understood}

The speaker started with outlining his talk. His first slide was about
how cooling by evaporation of He-4 can only reach about 1.3 K. Below
this temperature, the vapour pressure is very small and thus very
little evaporates to facilitate the cooling. This issue is resolved
by using mixtures of liquid He-3 and He-4. The dilution refrigerator
works on this technique.

At 0.7K, the dilution refrigerator starts working. We start with two
layers of almost pure He-3 and He-4 at 0.1K. He-3 starts diffusing
into He-4. The reason for this was explained later. While diluting
He-3 absorbs heat, thus the cooling action.

This next step can be understood in analogy with the evaporation process,
wherein we need to remove vapour for evaporation to continue. By mixing
into the lower layer, the He-3 layer above is effectively gets diluted.
In order to continue cooling, the He-3 dissolved in dilute phase must
somehow be removed.

Next the speaker talked about cooling power of the system. Vapour
pressure of He falls exponentially with decreasing temperature. It
is possible to increase the limiting concentration above 6.6\% by
increasing pressure, he asserted. He went on to derive an expression
for change in entropy for such a system.

Then the realization of such a system was discussed. The basic principle
may be described as follows. Like in evaporation, cooling takes place
when Helium atoms move from liquid to vapour phase, in dilution, cooling
takes place when He-3 atoms move form the concentrated to dilute phase.

Now coming to the reason for diffusion/dilution; it hinges on the
statistical characteristics of the constituents. He-3 is a fermion
while He-4 is a boson. Consequently, He-3 diffuses into He-4 to minimize
its Fermi energy while He-4 tries to gain more super-fluid condensation
energy.

---

Summary (has been written off of one of the slides):
\begin{enumerate}
\item The helium undergoes phase separation when cooled below 0.87K giving
two phases.
\item The specific heat of He-3 atoms is higher in the dilute phase than
in the concentration phase of the atom goes from the concentrated
to dilute phase, results in the `production of cold'. 
\item Non-zero solubility of He-3 in He-4 even at 0K leads to cooling power
which decreases with $T^{2}$, is much higher than cooling power of
evaporation which falls exponentially
\end{enumerate}

\subsection{What I'll remember}

Even innocent looking phase diagrams can be true gems!
\end{document}
