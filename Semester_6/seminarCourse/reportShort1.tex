%% LyX 2.0.6 created this file.  For more info, see http://www.lyx.org/.
%% Do not edit unless you really know what you are doing.
\documentclass[british]{article}
\usepackage{ccfonts}
\renewcommand{\sfdefault}{cmbr}
\renewcommand{\ttdefault}{cmtl}
\renewcommand{\familydefault}{\rmdefault}
\usepackage[T1]{fontenc}
\usepackage[latin9]{inputenc}
\usepackage{babel}
\begin{document}

\title{Seminar Course\\
IDC352}


\author{Atul Sinh Aurora\\
MS11003}


\date{Instructor: Prof. Arvind\\
Jan-April 2014}

\maketitle
\pagebreak{}
\begin{abstract}
I had the view earlier that attending seminars is not nearly as important
as spending time learning concepts systematically. In the recent past,
I have been made to realize that too much focusing also is not a good
idea. Thus, the topics that I am interested in but do not intend to
study systematically in the perceivable future, I can learn about
through seminars. The topics maybe sub topics in physics or those
of other disciplines. Making connections this way is helpful and accelerates
learning. So far, it has indeed been the case.

The following discussions have been severely shortened and may therefore
not be suitable for understanding the topics. However, the objective
here is to convey the idea behind the talks with very few words. I
hope in the process clarity hasn't been sacrificed.

\pagebreak{}
\end{abstract}

\section{Why a Quantum Theory of Gravity}


\subsection{The when, where and who | facts}

The talk was given by Prof. G. Date from IMSc, Chennai, in LH-3 of
the Lecture Hall Complex, IISER M, on Wednesday, March 26, 2014.


\subsection{Motivation for attending | feelings}

Our QFT instructor, Dr. Sudipta, informed us and recommended that
we attend it. That's one. Then I remember reading an eminent scientist
say that we are fortunate to have been born at a time when problems
in physics existed and we didn't have to create them. I feel that
Quantum Theory of Gravity is the one for us. It's even more motivating
to see that there was a time when even STR and QM couldn't be combined
consistently.


\subsection{This I could understand}

Quantum Gravity is the fiery marriage of General Relativity and Quantum
Mechanics. The pursuit is not purely hypothetical, nor a problem of
unexplained data. Wheeler's description of the final state of a star
is amongst one of the reasons why such a theory would be necessary.{*}

\[
R_{\mu\nu}(g)-\frac{1}{2}R(g)g_{\mu\nu}=\frac{8\pi G}{C^{4}}T_{\mu\upsilon}
\]


There've been two revolutions in physics; Relativity and Quantum Mechanics.
In General Relativity the notion of causality gets dynamical and metrical
properties can also change. The combination of special relativity
and Quantum Mechanics is Quantum Field Theory. QFT is consistent with
fixed causality defined by minkowski spacetime, but assumes that causal
structures can be defined at arbitrarily small lengths.

Next the speaker talked about the basic assumptions of GR.
\begin{enumerate}
\item Spacetime is defined by a 4d manifold and its metric $g_{\mu\nu}$
(-ve determinant)
\item Stress tensor, $T_{\mu\nu}$, which describes the matter distribution
\end{enumerate}
The metric imposes a causal structure on prashansa events in their
local neighbourhood.

Remarks: Given a metric, at any point, you can define a set of tangent
vectors. These can be classified as timelike, spacelike or null (or
prashansa like, viz. going nowhere). Corresponding geodesics determine
which evens can influence each other causally.

Einstein's equation is only a local PDE and determines solution spacetimes
only locally. Extended solutions are not guaranteed to be causal without
imposing extra conditions. Then the speaker talked about predictability,
that is events can be divided into cause and effect.

--

Then the speaker talked about time orientability, strong causlity
and stable causality. He stated that causality is not automatic, so
time-orientable space times are a subset of solutions of einstein's
equation. An example is given in the next section. 

Strong Causality: We can impose strong causality else its possible
to have future curves that come arbitrarily close to the existing
events. 

Stable Causality: Imposing stable causality on the other hand would
mean that spacelike and timelike do not interchange.

--

Global Hyperbolicity: Spacetimes which admit spacelike hypersurface
whose domain of dependence is th entire spacetime are globally hyperbolic.

Thus, global hyperbolicity autmatically ensures causal stability.

--

To have a spacetime with predictability, we can construct globally
hyperbloci spacetime (take a spacelike hyperprashansa, define domain
of depndence to be the full spacetime, it abruptly ends. You extend
it, how far? maximally possible). However, a predictable spaceimte
should be an inextendable/globally hyperbolic solution of einstein's
equation.

----

The speaker then went on to describe a slightly technical topic, `Property
of Globally Hyperbolic Space Times'. Consider a space $C(\Sigma,q)$
of causal curves, emanating from a hypersurface $\Sigma$ and passing
through a point $q$ to its future. Its a theorem that a curve of
maximum proper time is a geodesic without any `conjugate point' between
$\Sigma$ and $q$.

----

Next the speaker digressed to talk about a `Bundle of Time like Geodesics'.
Imagine a cloud of particles freely `falling' in a given space time.
These will be described by a bungle of time-like geodesics. Their
cross section in general can undergo 1. Shearing, 2. Twist or 3. Expansion.

----

'Raychaudhuri Equations and Conjugate Points' was discussed next.
This I couldn't understand much however what is known that there arises
a contradiction related to the existence of a singularity which is
resolved by concluding that the geodesic must be incomplete, viz.
the observer's world line must end abruptly. The details have been
skipped.

----

Blackholes and Thermodynamics were discussed next. 


\subsection{This I'll remember}

It is possible to construct solutions to Einstein's equations such
that locally they follow all the properties we expect but have peculiar
global behaviour. Imagine making a light cone on a piece of paper
and then folding it to make the future and past light cone match.
Locally, everything's consistent, but globally, you're going to the
past from the future!\\
\pagebreak{}


\section{The Dance Language of the Bees}


\subsection{The when, where and who | facts}

The talk was given by Prof. Raghavendra Gadagkar from IISc, Bangalore,
in LH5 of the LHC, IISER M, on Thursday, March 27 2014.


\subsection{Motivation for attending | feelings}

It was compulsory for us. However, it turned out to be rewarding.


\subsection{A Summary}


\subsubsection{Introduction}

\emph{Apis florea} is the species discussed in the talk. It is found
only in Asia, and is open nesting (thus harder to domesticate). They
are interesting as their social behaviour is not as sophisticated
(compared to the other four major species). They communicate by transferring
saliva and dancing, i.e. subject of the talk.


\subsubsection{About these bees}

Within these bees, there are drones (the males), worker (essentially
sterile) and one queen (that lays eggs). 

The said bees dance only on horizontal surfaces unlike the other species.
There are two types of dances known. First is called a round dance,
in which the bee makes a circular motion. The other's called a waggle
dance in which the bee transcribes an 8 shape with vigorous vibrations
half way.


\subsubsection{Communication and Dance (the real deal)}

Earlier it was thought that waggling related to pollen being found
and round to nectar. Later experiments confirmed that the waggling
dance corresponded to the food being farther than roughly 100 m and
the round to less than that.

Experiments have confirmed that the waggle dance conveys information
about the direction and distance of the resource found. They use the
sun's azimuthal as reference (as it is globally stable, unlike the
surroundings) even though it changes its location periodically. The
bees apparently correct for this. Further they even update their map
of the surrounding everyday and use that to locate the sun. 

In the waggle dance, during the middle part of 8, the direction in
which the bee moves contains the direction information. Also during
this part, the tempo of vibration codes for distance.

Although its not known how they find the angle, the distance is measured
not using the energy requirement as was previously thought, but using
optical flow.


\subsubsection{Sophistication}

These bees have evolved to use fallback mechanisms for defining the
angle.
\begin{enumerate}
\item Sun is used when its bright enough
\item Polarization of light on the sky is used if the sky is overcast
\item Local landmarks can also be used. They learn it each day to compensate
for their change with time.
\item Siesta; when the sun is overhead, they take time off
\end{enumerate}

\subsubsection{Experiments}

Various experiments were done to test color vision, co-relation between
sugar/nectar and waggle/round dance, co-relation between wiggle and
flight direction (Fan like experiment), that optical information is
used to find direction, use of Sun as reference (The lamp shift experiment
| miscommunication) and if dancing is sufficient to communicate information
about resources (The robot bee experiment).

\pagebreak{}


\section{Climate Change voo doo}

{[}sit with prashansa and finish this{]}

\pagebreak{}


\section{The Geometry of Physics}


\subsection{The when, where and who | facts}

Mr. Rahul Chajwa, physics major (MS10 batch), spoke on the said topic
on Wednesday, April 16, 2014, at 9 PM in LH3 of the LHC.


\subsection{Motivation for attending | feelings}

It was compulsory to attend atleast 2 in this series, however it turned
out to be rather informative and interesting.


\subsection{What I understood}

The speaker in essence talked about how one would characterize various
physical systems using geometry. He started by asking the question,
how one would characterize surfaces, after showing with a series of
figures. One way of such characterization he said was counting the
number of holes. He then showed how various surfaces like a cup, purse
etc. may deformed to a surface having clearly one or two (or even
no) holes. He went on to explain why such a characterization, though
intuitive is not always easy. He talked about Euler's polyhedron formula
and the Gaus Bounet Theorem. He attempted to show some generalizations
to higher dimensions, including the concept of derivatives. Specifically,
he talked about Liuovilles Theorem and discussed in some detail the
tangent space and its dual, co-tangent space. The fascinating part
of the talk was related to the double pendulum which we know is chaotic.
He explicitly showed an animation proving his point. He then asked
us what geometry we could associate with the co-ordinate phase space
of the setup. The answer was a torus. He then stated a result which
follows from the geometric analysis that the hamilton's equations
result in contraining the movement along geodescics of the torus geometry.
This has a profound implication, the resulting motion will be periodic
since geodesics are closed. This is almost impossible to see without
the geometric considerations since we don't even have an analytic
solution of the motion. Finally he demonstrated Euler's disc and talked
about the infinite frequency before collapsing and discussed why this
singularity is not a consequence of hamiltonian evolution but instead
that of dissipation. He concluded the discussion by attempting to
convince the audience that we can't think of logic without invoking
some sort of object (geometry) in our heads.

The speaker received criticism for not acknowledging the sources referred
to, and for shoddy board work from the instructor.


\subsection{This I'll remember}

A double pendulum's (co-ordinate) phase space can be thought of as
the points on the surface of a torus.

\pagebreak{}


\section{(Himanshu, paradoxes?)}

Definition of paradoxes, (?), Einstein clock, EPR


\section{Dilution Cooling}

{[}to attend today{]}

Introduction

Properties of liquid 

Introduction

1. cooling by evaporation fo He-4 can only reach about 1.3K. Below
this temp, the vapor pressure is very small so that very little would
evaporate

This limitation can be overcome by using imxtures of liquid he3 he4

This way is to 'evaporate' liquid h3 into he4

This is done in dilut

Phase diagram of he3, with temperature

If i start with E, Concentration of 

In the bototomost, the two phases will separate out. Into two layers

Cooling by dilution

The dilution refrigerator typically starts from 0.7K

Let us start with having two layers of pure He-3 and He-4 at 0.1k

Some he3 will diffuse into he4

This causes temprature to fall

--

By mixing into the lower layer, the helium 3 above is effectively
being diluted. Hence the term ``dilution cooling.''

This continues, and the concentration of He3 in the in the bottom
....

--

Cooling power,

Vapor pressure of He falls exponentially with decreasing temp. The
vapor pressure P is directly relted to the rate at which the he atoms
...

This figure shows that is possible to increase the limiting contration
above 6.6\% by increasing pressure

--

Realization of Dilution refirgiagator

The use of the ldiulation process of rcooling is similar in concept
to helium evaporation.

1. In evaporation, cooling takes palce when helium atoms move from
the liquid to the vapouir phase

--

He 3 is afermion, and He 4 is a boson | atoms of helium 3 distribute
in such a way to lower its fermi energy | superfluid concentration|

--

Why its absorbing heat | the chemical potential of 

--

Summary

1. The he (3,4) undergoes phase speration when cooled below 0.87 K
giving 2 phases

2. The specific heat of helium 3 atoms is higher in the dilute phase
than in the conncentrated phase. if the atom goes from the contrated
to the dilute phase, it results in the ``production fo cold.''
\end{document}
