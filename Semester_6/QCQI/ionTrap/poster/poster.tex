\documentclass[final]{beamer} % beamer 3.10: do NOT use option hyperref={pdfpagelabels=false} !
  %\documentclass[final,hyperref={pdfpagelabels=false}]{beamer} % beamer 3.07: get rid of beamer warnings
  \mode<presentation> {  %% check http://www-i6.informatik.rwth-aachen.de/~dreuw/latexbeamerposter.php for examples
    \usetheme{Berlin}    %% you should define your own theme e.g. for big headlines using your own logos 
  }
  \usepackage[english]{babel}
  \usepackage[latin1]{inputenc}
  \usepackage{amsmath,amsthm, amssymb, latexsym}
  % \usepackage{concrete}
  \usepackage{ccfonts}
\usepackage{avant}
\renewcommand{\ttdefault}{cmtt}
\renewcommand{\familydefault}{\rmdefault}
\usepackage{braket}

  %\usepackage{times}\usefonttheme{professionalfonts}  % times is obsolete
  \usefonttheme[onlymath]{serif}
  \boldmath
  \usepackage[orientation=portrait,size=a0,scale=1.5,debug]{beamerposter}                       % e.g. for DIN-A0 poster
  %\usepackage[orientation=portrait,size=a1,scale=1.4,grid,debug]{beamerposter}                  % e.g. for DIN-A1 poster, with optional grid and debug output
  %\usepackage[size=custom,width=200,height=120,scale=2,debug]{beamerposter}                     % e.g. for custom size poster
  %\usepackage[orientation=portrait,size=a0,scale=1.0,printer=rwth-glossy-uv.df]{beamerposter}   % e.g. for DIN-A0 poster with rwth-glossy-uv printer check
  % ...
  %
  \title[The Ion Trap Architecture]{The Ion Trap Quantum Computing Architecture}
  \author[Atul]{Atul Singh Arora}
  \institute[IISER M]{Indian Institute of Science, Education and Research, Mohali}
  \date{May 5, 2014}
  \begin{document}
  \begin{frame}{} 
    % \vfill
    \begin{columns}[t]
    \begin{column} \end{column}
	    \begin{block}{Prerequisites}
	    Can read and take being made fun of occasionally in good spirit. Also
	it'll help if you have some rough idea about what Qubits etc. really
	are and their use in the context of information processing.
	    \end{block}
	% \end{column}
	\end{columns}

    \begin{block}{\large Fontsizes}
      \centering
      {\tiny tiny}\par
      {\scriptsize scriptsize}\par
      {\footnotesize footnotesize}\par
      {\normalsize normalsize}\par
      {\large large}\par
      {\Large Large}\par
      {\LARGE LARGE}\par
      {\veryHuge veryHuge}\par
      {\VeryHuge VeryHuge}\par
      {\VERYHuge VERYHuge}\par
    \end{block}
    \vfill
  \end{frame}
  \end{document}