%% LyX 2.0.6 created this file.  For more info, see http://www.lyx.org/.
%% Do not edit unless you really know what you are doing.
\documentclass[british]{article}
\usepackage[latin9]{inputenc}
\usepackage{babel}
\usepackage[unicode=true]
 {hyperref}
\begin{document}

\title{Power Distribution using Superconductors}


\author{Vinod, Joydeep, Biplob, Anita and Atul}


\date{January 2014}

\maketitle

\section{Motivation}

One of the major concern of power companies in the present time is
the loss on electrical power in the supply process. Present day\textquoteright{}s
electrical connections in cities use aluminium wires. These wires
cause a high power loss of around 740 kW/km due to higher electrical
resistance{[}1{]}. Also the aluminium wires have to be carefully cleaned
to remove the aluminium oxide. Even better conductors like copper
cause a significant power loss. The environmental impact in term of
CO2 emission in the manufacturing process of aluminium wires is 778
ton C/Km/year. It is highly desirable to remove the existing aluminium
cables with some superconductor wires such that power loss due to
high resistance could be reduced. As global warming is one of the
major concern of the world, the alternative for electrical power transmission
should also reduce CO2 emission. In early days, the major difficulty
with super conductor wires was their critical superconducting temperature.
With the discovery of high temperature superconductors in 1986 which
could operate at a temperature higher than 67K (temperature of liquid
nitrogen), it is possible to use superconductor cables for power transmission.
The transmission loss is very less for DC current as compared to AC
and it also required less number of cables. With these high temperature
superconductors, we propose to transmit DC current in the main lines
and which can be further converted to AC depending on the individual
requirement. In this proposal we present the potential of replacing
the existing aluminium power lines with high temperature superconductor
YBCO cables{[}2{]} in the city of Chandigarh. Since the superconductor
cables are much costlier than the traditional copper and aluminium
wires and required to be maintained below a certain temperature, checking
the economic viability of the same should be checked and it is important
know the time period over which the establishment cost is compensated
by the reduced loss of power. 


\section{Blue Print}


\subsection{High Temperature Superconductor Wire}

Superconductivity is a phenomenon where the resistance of some material
is made zero by cooling it. High Temperature Superconductor (HTS)
was discovered in 1986. Superconducting cables have been developed
(TODO: quote paper) where coated conductors of YBCO wires are used.
This results in low loss and low cost for a power cable. YBCO wire,
often called a coated conductor, can be used for carrying AC current
in a superconducting cable. In these circumstances, hysteresis becomes
important in determining the resulting AC loss. This consequently
causes unnecessary load on the cooling system.


\subsubsection{Structure of a single YBCO tape }

On the base substrate, there\textquoteright{}s an intermediate layer
of some insulating material. This is followed by a superconducting
layer of YBCO, which has minimal thickness. In case of current flow
exceeding the critical current, the YBCO wire will burn out due to
heating. To prevent such short high current pulses from causing accidents,
an additional layer of silver and copper is laminated. To prevent
AC losses, multiple such wires are closely wound around a cylinder,
as AC loss results from large magnetic field perpendicular to the
edge.


\subsubsection{Cutting of the YBCO Tape }

For cutting the YBCO tape, laser methods are available. They have
been reportedly efficient enough to cause a minimal degradation in
the critical current (2-5\%). 


\subsubsection{Structure of a usable 3 phase wire }

A current insulating layer is added to a single YBCO wire. Three these
are assembled to create a 3 phase wire. They are enclosed in a larger
cable through which liquid nitrogen flows, to maintain low temperature.
The diameter of the duct cable is 150 mm. The maximum transportable
length of wire is 500 m. Tests have been performed on a 20 m long
wire, with a joint at 10 m. 


\subsection{Small scale experiment}

We plan to set up a small scale test of our experiment on some existing
power line. Basically,we located a 11KV power line and then plan to
implement our model there (near Aiyappa Temple, Dilshad Garden, Delhi),
for which we require permission from BSES Delhi. We chose Delhi because
we had our model town as Delhi from where we got all the data for
the length of 11kv lines as well as other specifications.We can perform
our tests there on the 200 m range where we would place 2 pumps and
can easily see whether we can get the desired results of our transmission
and also we can measure all the costs and see the pros and cons of
our experiment.


\section{Cost Estimate}


\subsection{Estimated Loss in the current distribution}

The cost involved in laying the power transmission lines is estimated
for the city of Delhi.
\begin{enumerate}
\item Rate of laying $a$ kV lines $(=b)\,6.59\times10^{5}$ Rs/Km
\item Length of $a$ kV lines $(=d)\,6520\times10^{6}$ Km
\item Now the laying cost is simply $=db=Rs.\,2.572077\times10^{9}\sim Rs.\,200$
crore rupees 
\end{enumerate}
Running Cost was estimated by finding the current flowing through
the wires and calculating the resulting power loss due to joule heating.
The current flowing through the wires was found indirectly by using
the capacity of the distribution power stations $(=c)$ connected
by these wires. The running cost was calculated as below
\begin{enumerate}
\item Power line voltage $=a=11\times10^{3}$ V
\item Distribution Transformers' total capacity $=c=6520\times10^{6}$ VA
\item Maximum current flowing through all the three wires combined $=i=c/a=592.73\times10^{3}$
A
\item Running Cost$=i^{2}\rho\left(\frac{d}{A}\right)t=4.44\times10^{15}t$
Joules (=Ws) which for $3.90$ Rs./kWh becomes $=4.81\times10^{9}t\sim481t$
crore Rupees, with $t$ in hours (where $A=100mm^{2}$ and $\rho$
has been evaluated from the specification of the $a$ V lines, viz.
`11kV overhead line with `DOG' (100 $mm^{2}$ ACSR on PCC)'.)
\end{enumerate}

\subsection{Estimated Installation Cost of Superconducting distribution}

We are still awaiting quotations for pipes and the cooling system.
However, we have a quote for the superconducting wire, \$250 to \$400/A-m
which we've been told will drop as we get closer to finalizing the
requirements. If we evaluate the price using just the numbers for
the superconducting cable we'll have (with 1\$\textasciitilde{}Rs.
60, cost of wire as \$300/A-m), Rs.$6.96\times10^{19}$for the model
discussed in the previous section. The corresponding time to recover
the cost would then be $1.6509\times10^{6}\sim1.6$ million years.
This although is a little overwhelming, can be reduced significantly
if as is expected, the wire cost is reduced in the near future. Thus,
it is anyway suitable to start perfecting the techniques.


\section{Implementation Phases}
\begin{enumerate}
\item We first of all plan to set up the small scale experiment which was
described earlier.
\item The next phase of our project is to see if the small scale experiment
is properly implemented or not and if implemented we can go directly
to the next phase. If there are some cons then we would try to rectify
them and proceed again to repeating the first step until the defects
are corrected.
\item With the help of BSES Delhi, if we can convince them of the benefits
of such a project, we can implement our plan for replacing the overhead
11kv lines with our high temperature superconductor lines and as a
result we can practically realize our project. 
\end{enumerate}

\section{Reference}
\begin{itemize}
\item \href{http://www.forumofregulators.gov.in/data/study/capital-cost-branchmark.rk}{www.forumofregulators.gov.in/data/study/capital-cost-branchmark.rk}
\item \href{http://www.bsesdelhi.com/HTML/wb-bsesatagiance.html }{www.bsesdelhi.com/HTML/wb-bsesatagiance.html }
\item YBCO\_wire (paper) Author: The Furukawa Electric Co.; Ltd. 
\item {[}1{]} M. HIROSE, T. MASUDA, K. SATO, R. HATA, High-Temperature Superconducting
(HTS) DC Cable. \textbf{SEI TECHNICAL REVIEW 2006, 61, 29}. 
\item {[}2{]} S. Mukoyama, M. Yagi, H. Hirata, M. Suzuki, S. Nagaya, N.
Kashima, Y. Shiohara, Development of YBCO High-Tc Superconducting
Power Cables. \textbf{Furukawa Review 2009, 35}. \end{itemize}

\end{document}
