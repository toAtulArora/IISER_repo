%% LyX 2.0.6 created this file.  For more info, see http://www.lyx.org/.
%% Do not edit unless you really know what you are doing.
\documentclass[english]{article}
\usepackage{ccfonts}
\renewcommand{\familydefault}{\rmdefault}
\usepackage[T1]{fontenc}
\usepackage[latin9]{inputenc}
\usepackage{babel}
\begin{document}

\title{Quantum information with modular variables}


\author{Atul Singh Arora}

\maketitle
I am interested in exploring the foundations of quantum mechanics.
This I find especially interesting because the very postulates of
the theory lead to some striking classically unexpected results which
have been verified experimentally, while the postulates themselves
aren't fully consistent; the measurement postulate and the unitary
time-evolution. Action at a distance like effects, which arise form
quantum correlations slash entanglement therefore are at the heart
of the theory. These effects when carefully studied lead to predictions
that act as tests for a system in states that can't be described classically
{[}arXiv:0811.2803{]}. 

Experimentally these tests have been performed on photonic and atomic
systems. However, performing these tests on massive systems is still
an area of research. A proposed scheme for such tests is the use of
modular variables (which I'll describe shortly) of macroscopic continuous
variable systems {[}Phys. Rev. Lett. 112, 190402 (2014){]}. The objective
of the project would be to use modular variables to understand the
origin of quantum effects, viz. effects peculiar slash characteristic
to quantum mechanical objects. These tests may even be used to quantify
entanglement in such systems and prove to be an interesting route
to studying the foundations of the subject.

Modular variables in simple terms may be understood as variables that
are bounded, which makes them `nice'. In continuous systems, variables
like position and momentum ($x$ and $p$) are unbounded. Use of modular
variables such as $\sin(x)$ and $\cos(p)$, which in fact can be
measured, maybe used in the aforesaid context instead.
\end{document}
