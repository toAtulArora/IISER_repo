\documentstyle[11pt]{article}
\setlength{\oddsidemargin}{-0.4in}
\setlength{\evensidemargin}{0in}
\setlength{\textwidth}{7.0in}
\setlength{\topmargin}{-1.2in}
\setlength{\textheight}{11in}
\pagestyle{empty}

\begin{document}

\begin{center}
{\Large Statement of Purpose} \\[.1in]
{\large Atul Singh Arora}
\end{center}

\vspace*{0.0in}
%Fascinated by the idea that laws of nature are discovered by people, that units like Ohms and Volts weren't God given, but infact created by people with those names, as a child I wanted to become a scientist.
%Fascinated by learning that units like Ohms and Volts were named after scientists, as a child, I wanted to discover new laws of nature. 
Fascinated by the idea that the laws of nature are discovered by people, as a child I wanted to become a scientist. Upon growing up, my interest shifted to building simple robots that can help do everyday chores. The construction involved programming, electronics and assembling mechanical parts. Upon learning physics and doing questions from books like Irodov, I became interested in physics again. It was however only after coming to IISER, my second home, that I took seriously the idea of becoming a scientist. 


Initially we're taught all the basic sciences plus pure math. I developed a taste for abstract mathematics during that time. My first subject for exploration was group theory and symmetry. I also looked at knot theory at the time and was surprised to learn its relation to quantum computation and elementary physics. I learnt eventually that while mathematics was fascinating in its own right, I missed physics, the connection to reality. That \emph{my equations describe nature}, I realised was rather important for me.


I spent the following summer constructing an experiment whose objective was to study the dynamics of spins on a lattice. Having enough experience with robotics, this project wasn't all that challenging in terms of novelty and learning, even though it took a lot of effort. % however; 
By the end of it, I was convinced that while constructing physics experiments, there's not too much focus on physics itself. I learnt that I really \emph{wish to explore theoretical physics} in my future projects.


By this time, I had chosen physics as my major. Physics had never ceased to surprise me, but with solid state physics, fluid mechanics, quantum computation, quantum field theory (QFT) and gauge theories, the standard model \& beyond, the excitement pinnacled. 


In my major years, I spent the first summer exploring the simulation of quantum physics  on a quantum computer. This was fascinating for I had independently discovered a small simulation protocol, that extended the pure state simulation to that of a mixed state. That for me, was the first novel construction of its type. However towards the end of it, I felt that I wasn't doing physics. I wanted to work on \emph{finding new laws of nature}.


In the next summer, I was awarded the DAAD-WISE scholarship to work in Germany. While applying, I was confused between quantum gravity and quantum optics \& foundations. I chose the latter for I felt it is experimentally more accessible, that our results could at least be verified within our lifetime. I was able to make some progress and construct a new extension of the Bell test\footnote{A. S. Arora, A. Asadian. \emph{Proposal for a macroscopic test of local realism with phase-space measurements.} Phys. Rev. A \textbf{92}, 062107.}. 
%\href{http://dx.doi.org/10.1103/PhysRevA.92.062107}{Phys. Rev. A \textbf{92}, 062107}}. 
In addition to this, I learnt about Bohmian Mechanics (BM) which is a deterministic theory that describes the same phenomena that Quantum Mechanics does. While I was not disappointed with my progress and had learnt about exciting research directions such as the No-Signalling principle/PR-box and information causality, I somehow \emph{missed the richness of the remaining physics}.


For my master's thesis, I decided to explore BM, a theory in which observers play no fundamental role. This I felt might eventually make interpretation of `quantum spacetime' more meaningful as a concept. For the thesis though, I'm focusing on a more specific problem, viz. seeing how BM could be consistent with contextuality; more precisely, I want to see how a theory deterministic in position \& momentum (q,p) can be consistent with a quantum mechanics' test that says (q,p) must be contextual, if at all they're deterministic. This would show the relation between non-locality and contextuality in the continuous variable regime, which isn't yet properly understood and is of considerable interest. The larger goal is to see how spin like discrete degrees of freedom are fundamentally different from (q,p). Perhaps this would suggest an appropriate understanding of it's extension to QFTs and quantum gravity (QG).

I haven't had any formal courses in QG but I am confident that I can pick up the essentials in a few months before joining the programme. Perhaps naively so, but I'm more inclined towards the loop quantum gravity (LQG) approach, as opposed to string theory. I have gleaned that the dynamics of LQG is the current active area of research with hamiltonian formulations and the spin foam alternative being among the studied approaches. I don't suppose I can formulate a research problem at the moment for my lack of knowledge about the area, however I hope that my past work supports my application to a PhD in this exciting field. The known applications to cosmology could perhaps be a starting point. Interestingly, recently BM was applied to cosmology as a test to distinguish it from QM.

Friedrich-Alexander Universit\"at Erlangen-N\"urnberg (FAU) has a vibrant group, consisting of various erudite researchers, such as Prof. H. Sahlmann and Prof. K. Giesel who're exploring QLG and its semiclassical limit, a regime that's criticized for not being able to produce flat space time. With mavericks working in important and diverse fields, both within and outside QG, I believe, FAU will be an ideal place, for such a pursuit.


% While I haven't had any formal courses in QG, I know that both popular approaches, Loop Quantum Gravity and String theory, are at the moment, not a complete description of nature, even though the latter is quite matured as a subject and has the more ambitious goal of unifying all forces. The University of Maryland (UM) has various erudite researchers, such as Prof. Ted Jacobson and Prof. Raman Sundrum who're exploring related ideas that include discretization of spacetime and observable implications of extra dimensions. With mavericks working in important and diverse fields, I believe, UM will be an ideal place to pursue my interests; additionally, the course work in the initial years would be especially conducive to learning the background for the research. A possible direction could be to attempt applying these principles to cosmology, however, for my lack of knowledge about the area, a definite research problem can't be formulated at the moment. I hope however that my past work sufficiently supports my application to a PhD in this exciting field. 


% I haven't had any formal courses in QG but I am confident that I can pick up the essentials in a few months before joining the programme. Perhaps naively so, but I'm more inclined towards the loop quantum gravity (LQG) approach, as opposed to string theory. I have gleaned that the dynamics of LQG is the current active area of research with hamiltonian formulations and the spin foam alternative being among the studied approaches. I don't suppose I can formulate a research problem at the moment for my lack of knowledge about the area, however I hope that my past work supports my application to a PhD in this exciting field. The known applications to cosmology could perhaps be a starting point. Interestingly, recently BM was applied to cosmology as a test to distinguish it from QM.


% An opportunity to work with one of the founders of LQG will be a very fortunate one. I needn't describe how conducive being with the group at CPT Marseille would be to research, especially since their interests, viz. quantum gravity and foundations, are almost identical to mine.% I must mention that my interests almost exactly overlap with those of the goup, viz. quantum gravity and foundations of quantum mechanics.

\end{document}
