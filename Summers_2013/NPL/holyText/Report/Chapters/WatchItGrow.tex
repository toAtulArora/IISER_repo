%************************************************
\chapter{Watch It Grow}
%************************************************

\section{Sentimental Introduction\protect\footnote{This section can be skipped, without any loss of continuity.}}
	Science often seems like a blackbox that relates observables. Even more often, it is rather convenient to lose touch of observables altogether, and wander in the blackbox. Performing an experiment, gets one closer to nature, to the roots of the subject.

\section{The Journey}
	\subsection{Look it has begun}
		This experiment wasn't started from scratch. My guide, \myProf, had already worked with a team and created the Dipoles as described earlier. The team had also worked on the image detection algorithms, but their work wasn't usable.
		\par
		There were three tasks at hand, of which one had been significantly simplified by the prior work.
		\begin{enumerate}
			\item The Dipole
					\par
					This had one apparent problem; the dipoles had to be made virtually frictionless (which is not to say they had excessive friction, infact they would oscillate atleast about 8 times before stopping aligned with earth's magnetic field)
			\item The Image Analysis
					\par
					This part I had to start from the beginning with two basic objectives, as stated earlier; measuring the angle of the dipoles and evaluating the current to be pumped based on the temperature selected.\\
					What was known soon, was that C++ will be used for programming and linux would be the operating system, to facilitate USB interface with the AVR (next step)
			\item The Current Control Hardware
					\par
					This is simply for providing a current pulse proportional to the intensity calculated by the lattice analyser. Some schematics for this were available, but were found to be inaccurate and incomplete.
		\end{enumerate}
	
	\subsection{Time Line}
		Listed below is the event log, which has the progress as and when it was made.
		\lstinputlisting[firstline=22,title=Time Line]{../../../../README.md}
	
	\subsection{Construction of the Dipole}
		To remove the friction, there were various ideas, including use of a super conductor. However, eventually three methods were considered and experimentally tested.
		\begin{enumerate}
			\item Ferro-Fluid: 
					\par
					As it turns out, there are substances that have a ferro magnetic properties but in the liquid form. Consequently, a strong enough magnet would glide if coated with this substance.
					\par
					Experimentally, it was found that the friction was higher than the `needle on glass' setup. 
			\item Magnetic Levitation:
					\par
					A magnet can easily suspend another magnet, granted it doesn't flip. This idea was used and a magnetic cylinder was placed co-axial to the needle, using a cylindrical eraser and glue. Beneath the glass slide, an identical magnet was placed with the face that repels upwards.
					\par
					Experimentally, again it was found that the motion was more damped than the `needle on glass' setup. The reason for this case was obvious after a little analysis and closer observation. The dipole would align to the field of the magnet, viz. the magnetic field was interfering with the dipole.
			\item Air Levitation:
					\par
					To test this, the very first requirement was a source of stream of air. For this, we started small. We arranged for a small USB fan from a colleague. The next task was to channel the flow of air. This was accomplished by attaching the front part of a Pepsi Bottle such that the larger diameter was closer to the fan and the mouth of the bottle had the chord stuck to it (could still be moved if required), as diagrammatically given in \autoref{fanSetup}. The final setup has been given in \autoref{fanSetupActual}.
					\begin{figure}[bth]
						\begin{center}
							\includegraphics[width=0.7\linewidth]{gfx/fanSetup.jpg}
						\end{center}
					\caption[Fan Setup]{Fan Setup}
					\label{fanSetup}
					\end{figure}

					\begin{figure}[bth]
						\begin{center}
							\includegraphics[width=0.7\linewidth]{gfx/fanSetupActual.jpg}
						\end{center}
					\caption[Final Fan Setup]{Final Fan Setup}
					\label{fanSetupActual}
					\end{figure}
					This failed miserably for the air pressure would fall the moment the assembly covered the fan. Introducing slits to allow air to flow in created no appreciable difference. This idea had to be dropped in favour of the vacuum cleaner setup as shown in TODO: add the figure. 
					\par
					The vacuum cleaner was used as a blower and connected to a box using a pipe. On one of the faces of the box, four holes were drilled (which were later enlarged). These were covered to increase the pressure when required. On the open hole, one dipole was placed (which had to be re built with a minor modification, refer to TODO: Add image) with a disc at the bottom. This is when an apparently bizarre observation was made. At a given pressure, it was found that the dipole could remain suspended in air beyond a certain height (and obviously there was an upper bound for the same). However, for heights lower than that, the dipole would fall. The explanation which seemed to resolve this was that the air could spread while rising sufficiently only beyond that height to apply pressure at a large enough surface area, thus create enough force. Albeit the experiment was not performed under precisely the same conditions, when it was repeated with a larger disc, the same problem was encountered, suggesting that there may be more to the explanation that deduced so far. Other geometries at the base (other than the disc) were also tried, such as a cone and a thermecol sphere, neither of which worked at low pressures which were enough to suspend the disc based setups. Further, at high pressures, disturbances in the form of torque in the dipole's axis of rotation begin to appear, which are fatal for the experiment.
					\par
					Other problems with air-suspension included damping in the vertical direction. Once the dipole reaches the vertical equilibrium point, it overshoots just as an oscillator. Since the friction at the plates holding the needle vertical is negligible, the oscillations don't get damped quick enough. This was experimentally observed also. The energy ratio between the rotational part and the translation part, in the earth's field, was calculated and found to be approximately one for the given setup.
					\par
					The most stable we could achieve with this setup was using a cylinderical projection from which the high pressure air escapes and a disc of comparable size attached to the dipole. This did get suspended satisfactorily, unlike the other methods where the suspension could not be maintained at the desired height, however the needle became rather wobbly and unstable resulting in increased friction.
					\par
					Online research indicated that air-bearings do function well enough and so do the boards of air hockey. These will be explored in the coming weeks to improve upon the methods to achieve the desired results.
		\end{enumerate}
	\subsection{Construction of the Lattice Analyser}
		The lattice analyser has come a long way. Image detection trials were initiated with \autoref{sampleImage}. 
		\begin{figure}[bth]
			\begin{center}
				\includegraphics[width=0.7\linewidth]{../../latticeAnalyser/picture002.jpg}
			\end{center}
		\caption[Sample Image]{Sample Image}
		\label{sampleImage}
		\end{figure}

		The idea was that once the ellipses have been detected, and they are different in colour, one can evaluate from their centroids, the position and the angle of the dipole. It must be stated that earlier it was attempted to use the greyscale image as was provided. However soon the shadow interference led to using coloured patterns instead. These patterns were not printed but displayed on a screen and the camera aimed appropriately.
		\par
		So first, the algorithm for detection of relevant part of the image had to be frozen. There were two candidates for this
		\begin{enumerate}
			\item Hough Transform Method
				\par
				Either one could use the already available in OpenCV, line detection or circle detection, both would've required changing the pattern on the dipole
				\par
				Or one could use an ellipse modification for the same, which would require programming the algorithm.
			\item Contour Detection and Ellipse Fitting
				\par
				This method detects contours in a given image, and the OpenCV example also shows ellipse fitting for the same. This seemed promising too, but it seemed more expensive (computationally) than looking for predetermined shapes.
		\end{enumerate}
		This work had been done within the first few days. 
		\par

		\begin{figure}[bth]
			\begin{center}
				\includegraphics[width=1.1\linewidth]{../../latticeAnalyser/snapshot1.png}
			\end{center}
		\caption[Hough Transform]{Hough Transform}
		\label{snapshot1}
		\end{figure}

		\begin{figure}[bth]
			\begin{center}
				\includegraphics[width=1.1\linewidth]{../../latticeAnalyser/snapshot2.png}
			\end{center}
		\caption[Contour Detection]{Contour Detection}
		\label{snapshot2}
		\end{figure}

		Next, a colour filter was to setup to improve the accuracy. When the algorithms were implemented, it was found that the Hough Transform method often misses detection of circles, refer to \autoref{snapshot1} (this is ofcourse after attaching a video stream instead of images to the code) as compared to contour detection \autoref{snapshot2}.
		\par

		\begin{figure}[bth]
			\begin{center}
				\includegraphics[width=0.3\linewidth]{../../latticeAnalyser/singleDipole.png}
			\end{center}
		\caption[Final Pattern]{Final Pattern}
		\label{singleDipole}
		\end{figure}

		\begin{figure}[bth]
			\begin{center}
				\includegraphics[width=1.1\linewidth]{../../latticeAnalyser/snapshot5.png}
			\end{center}
		\caption[Multi Shape, Single Colour]{Multi Shape, Single Colour}
		\label{snapshot5}
		\end{figure}

		After the detection, according the plan, two colours were to be used for the ellipses. However, running the hough transform twice would've dropped the detection speed to half, which wasn't worth it. It was then decided that the shapes should be made different instead of relying on two colours for the same information. After looking at various combinations, \autoref{singleDipole} was finalized, with an ellipse at the centre, and a circle along the minor axis for breaking the symmetry. This method did infact work as shown in \autoref{snapshot5}.
		\par

		\begin{figure}[bth]
			\begin{center}
				\includegraphics[width=1.1\linewidth]{../../latticeAnalyser/testGraphs}
			\end{center}
		\caption[First Observation]{First Observation}
		\label{testGraphs}
		\end{figure}

		The next challenge was to realize that a dipole detection can be missed and therefore mess up the counting, if that is the only way of uniquely identifying them. Unique identification is obviously required, as the external hardware must fire the coils of the right dipole. Thus a reference frame was used to uniquely identify the dipoles initially. This is expected to happen when they are stationary to get a good reading. In each frame, whenever a dipole is detected, it is associated with the dipole in the reference frame, by matching its location. If a dipole is not detected in a given frame, the software knows it was unable to record it and doesn't mess up neither the numbering nor the observations.
		\par

		\begin{figure}[bth]
			\begin{center}
				\includegraphics[width=0.7\linewidth]{../../latticeAnalyser/ellipseCircleDipoles4x4_black.jpg}
			\end{center}
		\caption[Final Test Pattern]{Final Test Pattern}
		\label{testPattern}
		\end{figure}

		After implementation of the last part, an animation sequence was created in Power Point, with the dipoles rotating with a constant speed and the camera was aimed at the screen. A still from the same is given in \autoref{testPattern}. \autoref{testGraphs}, shows the angular position versus time plot, for the first dipole and yes, it is linear, just as expected. Standard deviation tests are still to be done.
		\par
		Further tests, relating to it's timing were done and it was found that the processing itself was taking longer than 
		Following is the source code of the same, which has been made available online.
		\lstinputlisting[language=C++,title=latticeAnalyser.cpp]{../../latticeAnalyser/latticeAnalyser.cpp}
		
