%*******************************************************
% Acknowledgments
%*******************************************************
\pdfbookmark[1]{Acknowledgments}{acknowledgments}

\begin{flushright}{\slshape    
Every honest researcher I know admits he's just a professional amateur.
He's doing whatever he's doing for the first time. That makes him an amateur.
He has sense enough to know that he's going to have a lot of trouble,
so that makes him a professional.\\ \medskip
--- Charles F. Kettering (1876-1958) (Holder of 186 patents)}
\end{flushright}



\bigskip

\begingroup
\let\clearpage\relax
\let\cleardoublepage\relax
\let\cleardoublepage\relax
\chapter*{Acknowledgements}
\section*{Instructor}
	Here I've taken the liberty to write more than conventionally accepted. Biology is amongst the subjects I was certain I can never enjoy till class XII. Yet today, I can claim that it is this practical (1 credit course) for which I have spent maximum time this semester, second only to the Physics course on Thermodynamics (3 credits). I mention this since there exists only one reason for such a significant direction of effort into a direction formerly known to be un-interesting and `voodoo'; \myProf.
	\par
	Saying, I express my sincere gratitude to \myProf, for bringing the subject to life and helping us discover, in depth, the science behind the procedures, would be an astounding understatement. I hope you can however imagine the emotion I am trying to communicate, without me attempting to explicitly state it.
	\par
	I had for a semester, started to believe that optimizing for marks is a good idea. However I now realize that I can \emph{only learn} if I learn because I \emph{want to learn}. I am unable to restrict my learning in accordance with the syllabus or the time constraints. I thus had a conflict this semester about having to put in a disproportionate amount of time in a one credit course because of which I even cursed the design of the laboratory course. And today I am glad to tell you, its all worth it, for at the end, you need to aim at excellence, not success, especially if you \emph{want to be} successful. So I credit this realization to \myProf, for this is perhaps amongst the most important stepping stone in my learning process. Which is not to say he is the only one, but he is certainly amongst the finest, to whom I owe a lot of my \emph{true} education.\\
	\par
\section*{Team Members}
	Our team consisted of Ritu, Prashansa, Evelyn, Biplob and Me. I am very thankful for every single member for their work, unique insight, team spirit and most importantly, curiosity.
	\par
	I would like to specifically thank Ritu and Prashansa, who have gone out-of the way to do things right, to keep the standard of working high and consequently kept me motivated, even at low moments. They have worked extremely hard\footnote{We'd monitored everyone's contribution in terms of time and they were the amongst the highest scorers. This was done for a period of about two weeks when the activity was rather high}, have shown exceptional patience, for I now know that for certain that evolution/ecology experience can require that in excess, and have helped keep the team together, even during hostile disagreements.
\bigskip


\endgroup



