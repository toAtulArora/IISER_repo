%************************************************
\chapter{Quincke's Method}
%************************************************
\begin{flushright}
March 19 and 26, 2013
\end{flushright}
\section{Aim}
	To determine the magnetic susceptibility of an $Mn^{2+}$.
\section{Apparatus}
	U-tube, $Mn^{2+}$ solution (to make the solution: Volumetric Flasks, foil paper, weighing balance), Magnetic Field Sensor, Magnetic Field Producer

\section{Theory}
	\subsection{The Rationale}
		A spatially varied magnetic field can apply force on a magnetic moment. In this experiment, such a field is applied to one arm of a u-tube containing a paramagnetic liquid. The force is balanced by a height difference between levels of the liquid in the two arms.
		\par
		Simple calculations reveal that the force depends only on the value of the magnetic field at the beginning and end of the spatial region (of presence of the magnetic field). These are readily measurable. Further, the said height difference can be measured easily using the setup provided. As will be derived, there's a relation between these quantities that may be used to yield the `magnetic susceptibility' of the paramagnetic material, which is what we wish to investigate here. This quantifies how strongly the substance magnetizes in response to an external field.
	\subsection{Derivations}
		The experimental setup must be clear before continuing with this section. The conventions used will be defined as and when required.
		\subsubsection{Magnetization of bulk}
			The magnetization $\pmb M$ of a bulk material is defined as the magnetic dipole moment per unit volume. For a paramagnetic material, $\pmb M$ is parallel to $\pmb B$, the applied field and they're related as
			\begin{equation} {\label{eqn_magnetization}}
				\pmb M = \frac{\chi \pmb B} {\mu_0}
			\end{equation}
		\subsubsection{Force due to a Magnetic Field}
			The following will not be derived, but used directly. Here $\pmb F$ is the force, $\pmb m$ is the magnetic moment, $\pmb B$ is the magnetic field.
			\begin{equation}{\label{eqn_magneticforce}}
				\pmb F = \pmb \nabla (\pmb m.\pmb B)
			\end{equation}
			Before proceeding, we specify here the direction conventions. The u-tube is parallel to the Y-X plane. The two open arms are along the X axis. Within the tube, the z-axis corresponds to the direction of the magnetic field (whose magnitude is a function of position). 
			\par
			Now, in our case, we'll use $\pmb M$, instead of a single magnetic moment $\pmb m$, which is the magnetic moment for a unit volume. However, since $\pmb M$ is not expected to be constant over a given volume in general, we need to be a little more careful before substituting. Let the magnetic moment at some arbitrary height be given by $A \pmb M dx$, where $A$ is the cross sectional area of the u-tube and $dx$ an infinitesimal height. \footnote {we are ignoring in the analysis the bottom most part of the tube} This is justified as the magnetic field for any given height is constant. So using \autoref{eqn_magnetization} we have
			\begin{equation}
				d\pmb m = A \pmb M dx = A \frac{\chi \pmb B} {\mu_0} dx
			\end{equation}
			Thus we can write \autoref{eqn_magneticforce} as
			\begin{equation}
			\begin{split}
				d\pmb F &= \frac {A \chi} {\mu_0} &\pmb \nabla (B^2 dx) \\
						&= \frac {A \chi} {\mu_0} &[\pmb x \frac{\partial}{\partial x} (B_x^2 + B_y^2 + B_z^2)dx \\
						& & \pmb y \frac{\partial}{\partial y} (B_x^2 + B_y^2 + B_z^2)dx \\
						& & \pmb z \frac{\partial}{\partial z} (B_x^2 + B_y^2 + B_z^2)dx]\\
			\end{split}
			\end{equation}
			Now in accordance with the experimental setup, only $B_z \neq 0$ and $B_z$ is invariant with respect to Y and Z co-ordinates (restricted to within the u-tube), we get
			\begin{equation}
				d\pmb F = \pmb x \frac {A \chi} {\mu_0} \frac{\partial B_z^2}{\partial x} dx
			\end{equation}
			We now have an expression which just needs to be integrated to give the final result. However, there're two caveats. One is that for the same $x$, the term $dF$ can have at most two values depending on which arm we are looking at. \footnote{Observe that this still doesn't contradict the assumption that $B^2$ is independent of Y and Z coordinates \emph{within} the u-tube} Second is that we must apply the limits very carefully. This upward force is exerted only at all points within the tube where $\displaystyle \frac{\partial B_z^2}{\partial x} \neq 0$. So if we integrate from the top of the first arm to the top of the second, say from $x_t,y_t$ to $x_b,y_b$, we'll have
			\begin{equation}
				\pmb F = \pmb x \frac {A \chi} {\mu_0} (B^2_{z(x_t,y_t)} - B^2_{z(x_b,y_b)})
			\end{equation}
			From this, we can readily calculate the pressure exerted by the magnetic field as
			\begin{equation}
				P = \frac {\chi} {\mu_0} (B^2_{z(x_t,y_t)} - B^2_{z(x_b,y_b)})
			\end{equation}
			Since at equilibrium, this must be balanced by some other force, the liquid in the arm subjected to the magnetic field rises to create a height difference, say $2h$, to balance the pressure, which by elementary analysis we know would be
			\begin{equation}
				P = \rho g 2h
			\end{equation}
			This yields the final relation.
			\begin{equation} {\label{eqn_final}}
				\chi (B^2_{z(x_t,y_t)} - B^2_{z(x_b,y_b)}) = 2 \mu_0 \rho g h
			\end{equation}
		\subsubsection{Paramagnetic Susceptibility}
			Water also has it's own susceptibility which contributes to $\chi$. So to evaluate $\chi_{Mn^{2+}}$, we have
			\begin{equation}
				\chi = \chi_{Mn^{2+}} + \chi_{\text{water}} 
			\end{equation}
			where $\chi_{\text{water}} = -0.9 \times 10^{-5} \frac{m^3}{kg}$. Further, $\chi$ here represents volume susceptibility. To get mass susceptibility, we, as is dimensionally obvious, have
			\begin{equation}
				\chi_m=\frac \chi \rho
			\end{equation}
\section{Observations and Calculations}
	Observations have been appended at the end of this experiment.
	\par
	Slope of the graph equals $\chi / 4\mu_0 \rho g$ in $mm/G^2$. 
	\par
	For a 2M solution, the slope was $6\times 10 ^{-9} mm/G^2$ with $R^2=0.992$ and for the 3M solution of was $6\times 10 ^{-8} mm/G^2$ with $R^2=0.971$. 
	\par
	Now $\chi_{2M + \text{water}}=4\mu_0 \rho g (6\times 10^{-4})$ in SI units, where $\rho=$ Molarity times molar mass + mass of water in one unit volume $=(338+1000) g/L = 1338 kg/m^3$. On subsitution we get
	$\chi_{2M}=(1.054 \pm 0.008) \times10^{-5} - 0.9 \times 10^{-5} = (1.54 \pm 0.08) \times 10^{-6}$.
	\par
	Similarly, we have $\chi_{3M}$ (with $\rho=1507 kg/m^3$) = $(12.0 \pm 0.036) \times 10^{-5} - 0.9 \times 10^{-5} = (11.1 \pm 0.036) \times 10^{-5}$

\section{Error Analysis}
	\begin{align}
		\sigma (\text{slope}_{\text{2M + Water}})			&=0.048\times10^{-4}\\
		\text{Thus error in $\chi_{\text{2M + Water}}$} 	&=4\mu_0 \rho g (0.048\times10^{-4})\\
															&=0.008\times10^{-5}\\
		\text{Similarly, } \sigma(\text{slope}_{\text{3M + Water}}) &=4\mu_0 \rho g (0.174\times10^{-4})\\
																	&=0.036\times10^{-5}
	\end{align}

	% \begin{figure}[bth]
	% 	\begin{center}
	% 		\includegraphics[width=1.0\linewidth]{gfx/e2_circuit}
	% 	\end{center}
	% \caption[Simplified schematic of the Setup]{Simplified schematic of the Setup}
	% \label{e2_circuit}
	% \end{figure}

\section{Procedure}
	This has been heavily influenced by the Physics Lab Manual provided to us at IISER M, for the year 2013.
		\begin{enumerate}
			\item Calibrated the magnetic field produced by the electromagnet for various values of current in the Hall probe sensor. Fit the field vs. current data into a straight line. The probe was placed steady on a stand for all measurements.
			\par
			\emph{This calibration may be unnecessary if one uses only the hall sensor reading for measuring the magnetic field}
			\item The u-tube was cleaned with distilled water. Dried with compressed air and wiped clean from the outside as well.
			\item Two solutions of different concentrations of Manganese sulphate were prepared and one of Ferric Chloride.
			\item Placed one arm of the u-tube between the pole pieces so that the meniscus of the liquid is in the centre.
			\item Noted the initial reading of the meniscus with a travelling microscope.
			\item Measured the displacement $2h$ of the column of the liquid as a function of the applied field $B$.
			\item Measured the field near the surface of the liquid in both arms.
			\item Calculated the susceptibility using \autoref{eqn_final} and plotted a graph to test the linear relation between the difference of square of magnetic fields with $h$.
		\end{enumerate}

\section{Result}
	The susceptibility of $Mn^{2+}$ was found to be
	\par
	$\chi_{2M}=(1.54 \pm 0.08) \times 10^{-6}$ and $\chi_{3M} = (11.1 \pm 0.036) \times 10^{-5}$