%% LyX 2.0.6 created this file.  For more info, see http://www.lyx.org/.
%% Do not edit unless you really know what you are doing.
\documentclass[english]{article}
\usepackage[latin9]{inputenc}
\usepackage{amssymb}
\usepackage{babel}
\begin{document}

\section{Zeeman effect }

It is estabilished now that an atom in a magnetic field can be modelled
as a simple harmonic oscillator. Let $\omega_{o}$ be the frequency
of motion in the absence of magnetic field; the electron has the same
resonant frequency in motion along $x$, $y$ and $z$ directions.
In the presence of magnetic field B, the equation of motion, in the
rest frame of electron is given as 
\[
m_{o}\frac{dv}{dt}=-m_{o}\omega_{o}^{2}r-ev\times B
\]
where the $r$ is the position vector and $v=\dot{r}$ is the velocity
vector. If the direction of the field is along the $z$ axis, $B=B\hat{e_{z}}$

\[
\ddot{r}+2\Omega_{L}\dot{r\times\hat{e}_{z}+\omega_{o}^{2}}r=0
\]
 where $\Omega_{L}$ is the Larmor frequency defined as
\[
\Omega_{L}=\frac{eB}{2m_{o}}
\]
 Applying the matrix method to solve the above equation, we look for
a solution in the form of a vector oscillating at $\omega$ such as
\[
r=Re\left\{ \left(\begin{array}{c}
x\\
y\\
z
\end{array}\right)exp(-i\omega t\right\} 
\]
 and obtain the following matrix
\[
\left(\begin{array}{ccc}
\omega_{o}^{2} & -2i\omega\Omega_{L} & 0\\
2i\omega\Omega_{L} & \omega_{o}^{2} & 0\\
0 & 0 & \omega_{o}^{2}
\end{array}\right)\left(\begin{array}{c}
x\\
y\\
z
\end{array}\right)=\omega^{2}\left(\begin{array}{c}
x\\
y\\
z
\end{array}\right)
\]
 we then find the eigen values by building the following determinant
\[
\left|\begin{array}{ccc}
\omega_{o}^{2}-\omega^{2} & -2i\omega\Omega_{L} & 0\\
2i\omega\Omega_{L} & \omega_{o}^{2}-\omega^{2} & 0\\
0 & 0 & \omega_{o}^{2}-\omega^{2}
\end{array}\right|=0
\]
 which then gives 
\[
\left\{ \omega^{4}-\left(2\omega_{o}^{2}+4\Omega_{L}^{2}\right)\omega^{2}+\omega_{o}^{2}\right\} \left(\omega^{2}-\omega_{o}^{2}\right)=0
\]
By inspection, $\omega=\omega_{o}$is one solution. For $\Omega_{L}\ll\omega_{o}$
, the approximate frequencies are $\omega\backsimeq\omega_{o}\pm\Omega_{L}$.
The eigen vectors corresponding to these
\[
r=\left(\begin{array}{c}
cos(\omega_{o}-\Omega_{L})t\\
-sin(\omega_{o}-\Omega_{L})t\\
0
\end{array}\right),\left(\begin{array}{c}
0\\
0\\
cos(\omega_{o}t)
\end{array}\right),\left(\begin{array}{c}
cos(\omega_{o}+\Omega_{L})t\\
sin(\omega_{o}+\Omega_{L})t\\
0
\end{array}\right)
\]
 the motion along $z$ axis is not affected by the magnetic field
and the angular frequency remains $\omega_{o}$. The off diagonal
terms $\pm2i\omega\Omega_{L}$of the matrix depict the coupling of
the motions in the $x$ and $y$ directions resulting in circular
motion confined to the $xy$ plane, with frequencies shifted from
the Larmor frequency $\omega_{o}$. Evidently the applied external
field separates the initially equal frequencies.

The direction of radiation is given by the $a\times r$, where the
$a$ is the acceleration and $r$ is the position vector of the electron. 

The electron oscillating parallel to $B$ radiates with angular frequency
$\omega_{o}$, in the $y$ direction when seen in the $xy$ plane,
and no radiation is observed in the direction of the magnetic field.

The circular motion of the electron produces radiation with angular
frequencies $\omega_{o}+\Omega_{L}$ and $\omega_{o}-\Omega_{L}$.
Looking edge-on, this circular motion is seen as a linear sinusoidal
motion. Seeing along the $y$ axis, the linear motion is along the
$x$ axis, and the radiation is polarised perpendicular to the magnetic
field. Looking along the $z$ direction, the motion is circular and
the radiation is circularly polarised, associated with the respective
frequencies.
\end{document}
