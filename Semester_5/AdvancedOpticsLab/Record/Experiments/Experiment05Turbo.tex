%% LyX 2.0.6 created this file.  For more info, see http://www.lyx.org/.
%% Do not edit unless you really know what you are doing.
\documentclass[oneside,english]{amsbook}
\usepackage[latin9]{inputenc}
\usepackage{amsthm}

\makeatletter
%%%%%%%%%%%%%%%%%%%%%%%%%%%%%% Textclass specific LaTeX commands.
\numberwithin{section}{chapter}
\numberwithin{equation}{section}
\numberwithin{figure}{section}

\makeatother

\usepackage{babel}
\begin{document}

\section{Calibrating the Fabry Perot}

Let us start with the assuming that you have a bright fringe at the
centre. Let the distance initially between the two mirrors of the
etalon be $d_{1}$. So in this situation we have

\[
2d_{1}=n\lambda
\]


Where $n$ is some arbitrary number, and $\lambda$ is wavelength
of prashansa. Now suppose you move one of the mirrors such that the
new distance is $d_{2}=d_{1}+\Delta d'$, where $\Delta d'$ is what
you measure using the screw guage. Also, let us assume that the centre
is bright again, and a total of $m$ fringes collapsed to the centre
in the process of changing the distance. Then, ideally we must have

\[
2\Delta d'=m\lambda
\]


We know all the quantities in the equation. However, in practice,
there'll be some error, which can be removed by calibration. Thus
we define the expected distance by

\[
\Delta d\equiv m\lambda/2
\]
And we define our calibration constant

\[
C=\frac{\Delta d}{\Delta d'}
\]



\section{Determining the Space Between the Mirrors of the Etalon}

The analysis for this part rests on the assumption that effect displayed
in the following figure can be neglected. We'll show later by simple
calculations the range of the radius for which this can be neglected.

We start simple. From trigonometric arguments, we have the following
relation between the angle $\theta_{m}$ of a ray from the source
to the screen%
\footnote{with respect to the line joining the centre of the pattern and the
source%
}, corresponding to a bright fringe, where $m$ is a number (which
will be defined soon), and $r_{m}$, the radius of the ring on the
screen.

\[
\frac{r_{m}}{D}=\tan\theta_{m}
\]


For small angles, we have

\[
\tan\theta_{m}=\theta_{m}=\frac{r_{m}}{D}
\]


Now we look at the $n$th large bright ring from the aforesaid ring.
For this also, we correspondingly have

\[
\tan\theta_{m+n}=\theta_{m+n}=\frac{r_{m+n}}{D}
\]


We now give meaning to number $m$. For the first ring, the condition
for being bright is given by 

\[
2d\cos\theta_{m}=m\lambda
\]


where $\lambda$ is the wavelength of the incident laser light. Then
it follows that

\[
2d\cos\theta_{m+n}=(m-n)\lambda
\]


where the minus sign is owing to the fact that we've assumed the \emph{larger
}ring to correspond to $\theta_{m+n}$, which implies $\theta_{m+n}>\theta_{m}$.

We eliminate $m$ from these equations to obtain

\[
2d(\cos\theta_{m+n}-\cos\theta_{m})=-n\lambda
\]


If we use small angle approximation for $\cos\theta=1-\theta^{2}/2$,
we get

\[
d(\theta_{m+n}^{2}-\theta_{m}^{2})=n\lambda
\]


which in terms of observables, is

\[
\frac{r_{m+n}^{2}-r_{m}^{2}}{D^{2}}=\frac{n\lambda}{d}
\]


So finally we have

\[
d=\frac{n\lambda D^{2}}{r_{m+n}^{2}-r_{m}^{2}}
\]



\subsection{Small Angle Approximation}

We have assumed that the analysis given above works, so long as the
distance $2d\tan\theta<1mm$, the least count of the radius measurement
apparatus. We then find the range of $\theta$ for which this is valid
and thereby determine the radius of fringes we can use, without the
theory failing. The $d\approx2cm$ in our setup, so we get $\theta<0.025$,
and using $\theta\approx\tan\theta=d/D$, with $D=70$cm, thus the
radius must have the following contraints

\[
r<1.75cm
\]
for confident results.


\section{The setup}

The setup has very curious similarities with the michelson interferometer.
First note that in the michelson interferometer, there are three cases
to be considered
\begin{enumerate}
\item Point source (the black paper with hole dosn't function as a very
good point source)
\item Non coherent extended source (the diffuser case)
\end{enumerate}
In the first case, we would obtain a pattern on a screen, but won't
be able to see it. In the second case we won't see it on a screen.
However we (experimentally known) can see it directly.

One could verify these results in the Fabry Perot setup as well. With
a point source (the diode laser, with or without an additional lense),
when viewed through the fabry perot, doesn't yield a pattern, although
you get it on a screen. Conversely, if you converted the point source
into an extended source, the pattern on the screen dissapears, however
you can now observe it directly through the etalon. In complete agreement
with the predictions.
\end{document}


%%%%%%%%%%%%%%%%%%%%%%%%%%%%%%%%%%%%%%%%%%%%%%%%%%%%%%%%%%%%%%%%%%%%%%%%

