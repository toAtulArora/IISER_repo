%************************************************
\chapter{Michelson Interferometer}
%************************************************
\begin{flushright}
January 8 and 19, 2012 \\
% [$3 \times 6 hours spent$]
\end{flushright}

\section{Aim}
	To determine
	\begin{enumerate}
		\item the average wavelength of a monochromatic source
		\item the separation in wavelength of a dichromatic source
	\end{enumerate}

\section{Theory}
	The theory given in the manual is roughly sufficient. Finer points have been discussed here. We start with posing some questions and then discuss them to develop a better understanding of the subject.
	\begin{enumerate}
		\item What is the wavefront of the waves after they go through a diffuser?
		\item Can the interference pattern be obtained without the diffuser? How does the diffuser help?		
		\item Why are the fringes circular? Explain using Huygen's principle (the ray optic method is rather simple).
		\item Are the fringes real or virtual?
		\item If the light source was in fact a point source, what type of an interference pattern will you obtain? (courtesy sir)
		\item In a typical YDSE setup, if we don't use a screen, are we expected to observe fringes?
		\item How can we span all angles using just two knobs in the mirror?
		\item Can you apply the idea of beats to explain the increase decrease of contrast?
		\item What is the expected pattern for a plain wavefront?
		\item For a plain wavefront, when a `dark' pattern is obtained (you'll know once you solve the previous question), where does the energy of the electromagnetic wave disappear? (sir asked this)
	\end{enumerate}
	
	\subsection{`Practical' Theory}
		Some more questions whose answers become clear after playing with the apparatus for sufficiently long
		\begin{enumerate}
			\item What is the primary source of the backlash error in the fine rotation knob?
			\item When the average distance of the mirrors from the partly reflecting mirror is higher, why is it harder to get circular fringes?
			\item What can cause a uniform rotation of the knob to screw (threaded cylinder) to not cause a corresponding change in the fringe pattern? (basically identify the main cause of this error)
		\end{enumerate}
\section{Procedure}	
	\subsection{Obtaining the ring}
		It is assumed that you've setup the michelson interferometer in accordance with the diagram in Jenkins White.
		\begin{enumerate}
			\item Move the coarsely moveable mirror to (roughly) the smallest distance from the beam splitter.
			\item Now move the other mirror to a slightly larger (you can use smaller also, but then the steps would change correspondingly) distance from the beam splitter, than that of the coarsely movable mirror.
			\item Ensure that the pin hole disk is being used.
			\item Align the mirror using the three screws provided such that all the four spots coincide (you can choose to look directly without the telescope; in fact that works better usually).
			\item Now remove the pin hole disk from the view and put the diffuser (if not already present).
			\item Move the moveable mirror at most four times using the coarse movement drum (ensure the movement knob is unlocked) until the fringe pattern is observed. If the pattern is too dense (more than roughly 15 fringes), then follow from the mirror alignment step.
			\item Assuming you have roughly 10 fringes at this stage, you now need to bring the centre of the rings into view (if it is already, you're running on beginner's luck).
			\begin{enumerate}
				\item There are two screws on the mirror and they can roughly be thought of as adjusting the X and Y offset.
				\item You'll know you're on the right track if the fringes magnify as you adjust
				\item Note that you must leave the knob to know where it really is. Simply holding it also causes the position of the knob to change.
			\end{enumerate}
			\item If you want further magnification, you can continue rotating and aligning the centre as you go if the need be.
		\end{enumerate}
		CAVEAT: For certain path differences, the contrast will become very low (as is clear from theory); don't panic.
	\subsection{Finding $\lambda_\text{average}$}
		Assuming you have obtained the ring already;
		\begin{enumerate}
			\item Set the movement to fine using the lock knob on the apparatus.
			\item Place the telescope if you haven't done that already, such that one of the rings is just at the cross-wire.
			\item Move the fine rotation knob until the fringes just start to move.
			\item Now keep track of the rotations and count 20 fringes as they cross the cross-wire.
			\item Repeat this to get sufficient observations			
		\end{enumerate}
	\subsection{Finding $\lambda_\text{separation}$}
		Assuming you've obtained the ring;
		\begin{enumerate}
			\item Lock the movement to fine using the lock knob
			\item Move the fine rotation until all the fringes disappear and just begin to appear
			\item Note the position at this point
			\item Now continue rotating the knob until the fringes disappear again and just begin to appear
			\item Note this position again
			\item Repeat this to get sufficient observations
		\end{enumerate}