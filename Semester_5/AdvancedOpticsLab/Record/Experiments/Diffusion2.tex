%% LyX 2.0.6 created this file.  For more info, see http://www.lyx.org/.
%% Do not edit unless you really know what you are doing.
\documentclass[english]{article}
\usepackage[latin9]{inputenc}
\usepackage{textcomp}
\usepackage{amstext}
\usepackage{amssymb}
\usepackage{esint}

\makeatletter

%%%%%%%%%%%%%%%%%%%%%%%%%%%%%% LyX specific LaTeX commands.
\newcommand{\lyxmathsym}[1]{\ifmmode\begingroup\def\b@ld{bold}
  \text{\ifx\math@version\b@ld\bfseries\fi#1}\endgroup\else#1\fi}


\makeatother

\usepackage{babel}
\begin{document}
There are two important things to understand in this experiment.
\begin{itemize}
\item The phenomenon of diffusion and Einstein-Stokes' equation.
\item Pulse Field Gradient - Spin Echo (or PFGSE)
\end{itemize}

\section{Diffusion}

Diffusion is the random translational (or Brownian) motion of molecules
or ions across a system driven by its internal thermal energy. Translational
diffusion is the basic mechanism by which molecules are distributed
in space and plays an important role in any chemical reaction. The
systems for which there is an initial concentration gradient, Fick's
laws quantitatively explains the relation between the (instantanous
and local) concentration and the flux of the molecules/ions.

Fick's first law postulates that the flux of a material across a given
plane is proportional to the concentration gradient across the plane.
In other words,

\begin{equation}
J\ =-D\frac{\partial C(x,t)}{\partial x}
\end{equation}


Where \emph{J} is the flux (Number/meter$^{2}$ sec) and \emph{$\frac{\partial C(x,t)}{\partial x}$}
is the concentration gradient. The constant of proportionality is
called the diffusion constant and has the dimensions of $length^{2}/time$.
There is a negative sign to show that molecules/ions are flowing in
the direction of lower concentration. The above postulate is sufficient
to understand the behavior of the systems for which the concentration
gradient doesn't vary with time. To analyze the variations in concentration
due to time, Fick further postulated that the magntiude of change
in the local concentration over time is equal to that in local diffusion
flux, implying,

\begin{equation}
\frac{\partial C(x,t)}{\partial t}=-\frac{\partial J(x,t)}{\partial x}
\end{equation}


The above two equations can be combined to give the one dimensional
version of the famous diffusion (or heat) equation:

\[
\frac{\partial C(x,t)}{\partial t}=\frac{\partial^{2}C(x,t)}{\partial x^{2}}
\]


It is assumed that the diffusion coefficient is independent of the
position. There are three more things to be proven before one can
move on to the NMR concepts. Firstly, The mean square value of the
displacment of the particles can be evaluated as $6Dt.$ To prove
this, one first needs to find the fundamental solution of the diffusion
equation. Let us skip the long procedure of deriving the fundamental
solution of the diffusion equation and take it as:

\begin{equation}
P(x,t)=\frac{1}{(\sqrt{4\pi D})^{3}}e^{\frac{-(x-x_{0})^{2}}{4Dt}}
\end{equation}


Average of any function $L(t)$ can now be calculated as:

\begin{equation}
<L(t)>\equiv\int_{\mathbb{R}^{3}}L(x,t)P(x,t)dx
\end{equation}


Secondly, we need Stoke's law that the force needed to move a small
sphere of radius \emph{R} through a continuous medium of viscosity
$\eta$ with a velocity \emph{V} is given by the following relation:

\begin{equation}
F=6\pi\eta RV
\end{equation}


Lastly, Einstein-Stoke's equation describes the way that diffusion
increases in proportion to temperature, and is inversely proportional
to the frictional force experienced by a molecule $f,$ where $f=6\pi\eta R.$
Thus, 
\begin{equation}
D=\frac{k_{B}T}{6\pi\eta R}
\end{equation}



\section{Pulse Field Gradient - Spin Echo}

Let's take a detour and first understand what spin echo sequence means.
Imagine that one has a sample where initially the magnetization of
the molecules is parallel to the external field. Let's ignore the
diffusion effects for a while. 

A $90\lyxmathsym{�}$ pulse is applied along the x direction so that
the net magnetization now lies in the xy plane, in the y axis. During
the period of time following the removal the RF pulse, each spin experiencing
a slight variation in magnetic field begin to fan out slowly or \textquotedblleft{}dephase\textquotedblright{}.
The variations in the magnetic field come from both transverse relaxation
due to T$_{2}$ (spinspin interaction) and inhomogeneities in the
external field. The importance of the spin echo experiment is that
the effects of the inhomogeneities are made reversible. At a time
$\tau$ a $180$� pulse is applied along the y direction. The spins
are therefore rotated by $180\lyxmathsym{�}$ around the y axis thereby
remaining in the xy plane. As a result of the inverted relative positions,
and because each spin continues to precess with its former frequency,
all spins will be perfectly reclustered at time $2\tau$ forming what
is called an echo.%
\footnote{Diffusion measurements by NMR, http://www.uni-muenster.de/%
}

How are this dephasing and echoing related to diffusion coefficient
measurements? Well, imagine now that I have the sample enclosed in
a thin tube. If I apply an external magnetic field that varies linearly
with height then I should expect a linear variation in the larmor
freqencies. Heterogenous magnetic field is often supplied by applying
two pulses of duration $\delta$ and constant gradient $g$ between
the $90$� and $180\text{�}$ pulse and after the $180$� pulse with
an interval of time $\Delta$ between them, in the background of a
homogenous field. If the molecules are now in translational motion
then the re-phasing of the spins won't be perfect as the frequency
of the spin vector while dephasing is different from the frequency
while rephasing. This decreases the intensity of the spin echo which
can be quantified by the following expression:

\begin{equation}
I(t)=e^{-\frac{2\tau}{T_{2}}}e^{-Dg^{2}\gamma^{2}\delta^{2}z(\Delta-\delta/3)}
\end{equation}


(INSERT PROOF)

We can vary $I$ by keeping any $3$quantities out of $g,\ \gamma,\ \delta,\ \Delta$
constant and varying one and do a best fit with equation $7$ to obtain
$D.$ In our experiment we varied $g$ and held all the other parameters
constant. 
\end{document}
