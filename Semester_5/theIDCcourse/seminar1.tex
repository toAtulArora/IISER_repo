%% LyX 2.0.6 created this file.  For more info, see http://www.lyx.org/.
%% Do not edit unless you really know what you are doing.
\documentclass[oneside,english]{amsbook}
\usepackage{ae,aecompl}
\usepackage{berasans}
\renewcommand{\ttdefault}{cmtl}
\renewcommand{\familydefault}{\rmdefault}
\usepackage[T1]{fontenc}
\usepackage[latin9]{inputenc}
\usepackage{amsthm}
\usepackage{amstext}
\usepackage{amssymb}
\usepackage{esint}

\makeatletter

%%%%%%%%%%%%%%%%%%%%%%%%%%%%%% LyX specific LaTeX commands.
\providecommand{\LyX}{L\kern-.1667em\lower.25em\hbox{Y}\kern-.125emX\@}

%%%%%%%%%%%%%%%%%%%%%%%%%%%%%% Textclass specific LaTeX commands.
\numberwithin{section}{chapter}
\numberwithin{equation}{section}
\numberwithin{figure}{section}

\makeatother

\usepackage{babel}
\begin{document}

\title{This I understood%
\thanks{or merely noted (depends on the seminar)%
} in the Seminars}


\email{to.AtulArora@gmail.com}


\urladdr{http://www.github.com/toAtulArora/IISER\_repo/}


\author{Atul Sinh Aurora}
\begin{abstract}
Seminars have been very challenging for me as the protocol of asking
questions there is not very clear. Consequently, what follows, is
not even close to being rigerous; the ideas understood, however have
been attempted to be conveyed with enough clarity, although this may
not represent my best work. %
\thanks{I am experimenting with 3 new concepts; an alternative keyboard layout,
the Dvorak English, then there's \protect\LyX{} and finally the document's
class here.%
} The template used here is that of a book, however the chapters are
merely corresponding to different seminars.
\end{abstract}
\maketitle
\tableofcontents{}


\chapter{Adventures of the interface of physics, chemistry, biology and engineering}


\section{The facts}


\subsection*{When, where and who}

The seminar was held on October 23, 2013, 4:00 PM, in LH 3 (LHC).
The speaker was Prof. Deepak Mathur (TIFR, Mumbai).


\subsection*{The Abstract}

Access to ultrashort laser pulses that last long enough to accommodate
only a few optical cycles, is begginning to allow time dependent nuclear
and electron dynamics to be probed within atoms and molecules, thereby
enhancing our ability to gain proper insights into how quantum systems
react to strong external fields and how they might be subjected to
optical control. In this talk I shall present an overview of how we
have utilized intense laser pulses of duration as short as 4 femto
seconds to explore dynamical effects of relevance to divere areas
of the natural and engineering sciences, including applicationl purtaining
to DNA damage, laser-induced materials modification and green photonics.


\section{What I could follow}

Prof. Mathur started with talking about how very short duration of
light pulses (order of femto seconds) allows only one or two wavelengths
of light to fit in. He then talked about the intensity of his high
energy laser.

To get an understanding of the strength of this laser, consider the
hydrogen atom for instance. Its coulombic attraction force, is a result
of the electric field which is given by

\[
I=2\frac{\epsilon_{0}}{\mu_{0}}^{\frac{1}{2}}|E|
\]


With $E=5\times10^{9}V/cm$, we get $I=3.5\times10^{16}W/cm^{2}$.

He then talked about small disturbances in the hydrogen atom that
can be approximated as a spring. $F=-kz$, where as for large enough
distrubances you get terms of higher order as well, which add overtones
of the oscillation frequency to the mix. He said theoretically because
of this and the time dependence, large disturbances (produced by the
laser in this case), this is a difficult problem to solve.

He considered the potential energy curve for a hydrogen atom given
by 
\[
V=-\frac{q}{x}\pm E.x
\]


which has roughly an energy $10^{13}-10^{14}W/cm^{2}$. When this
potential is dostorted by the disturbance, there can be a situation
where tunelling occurs (roughly $10^{14}-10^{15}W/cm^{2}$) and then
there can be a situation where it's over the barrier. 

There was also then, a discussion on Pondermotive forces, which are
given by 

\[
F_{p}=-\frac{e^{2}}{4m_{e}\omega^{2}}\nabla\langle E^{2}(x)\rangle
\]


(which he mentioned is also there in landau's book) and the potential
for the same is then given by

\[
U_{p}=\frac{e^{2}I\lambda^{2}}{16c^{2}\pi^{2}m_{e}}
\]


where $I=10^{16}W/cm^{2}\,\,,\,\lambda=806\, nm\,\,,\, U_{p}\sim100\, eV$
so then we have $U_{p}>\text{Interaction Energy }>h\nu$

Thereafter, he started discussing the effect of the laser on Ethanol.
The experiments were performed on a single molecule, using techniques
he said he'll mention later. The objective was to be able to break
some specific bond, and not the one with the least energy first, and
so on, as was thermodynamically predicted. To partially achieve this,
he'd experimented with various polarizations of light, specifically
linear and circularly polarized light. He was able to demonstrate
that for the same intensity, the polarized light was able to break
bonds much more easily. Further, he also showed that the thermodynamically
expected behaviour of weaker bonds breaking first and the stronger
linearly later, is not observed.

Next he talked studies related to how for a given carrier gaussian
envelope, changing the phase affects the action of light on the molecule.
He went on to talk about how the duration of the pulse also plays
an interesting role. He showed results of an experiment where a $10\, fs$
laser beam produces mostly molecular entities, suggesting only the
electronic degrees of freedom are affected by the pulse, as the nuclear
degree of freedom operates in the time scales of roughly $30\: fs$.
In an experiment with a $30\: fs$ pulse, mixed entities were found
(both atomic and molecular). However another experiment proved that
in shorter time scales, these degree of freedoms actually combine.
The experiment is related to the \emph{Jahn-Teller Effect}. The consequence
of this is that symmetric molecules like $TMS$ can't quite be converted
into $TMS^{+}$. However, as it turns out, using the laser method,
one can produce these species which live for $\mu s$, by reducing
the duration of the pulse to about $5fs$. 

Due to shortage of time, Prof. Mathur screamed past numerous slides
to reach the one that read Blood, Saliva and DNA. Their relevance
here may seem to be absent, but he connected this by first talking
about rivers and how dead bodies tend to end up on the river bank,
even when carefully thrown at the centre of the banks. The point was
that since the velocity of the central part of the flow is the highest,
this is bound to happen. Now it gets interesting. This flow may be
thought to dipict the flow of blood in our vessels, and yet cells
like the RBCs for instance, don't end up on the wall. The cause for
this may be understood by one of the experiments performed by Prof.
Mathur's group, which involved optically trapping (also known as optical
tweezing in Biology) a single cell by virtue of creation of a potential
minima using lasers, and then observe a flourescent spot inside the
RBC. The results of this experiment seemed to suggest that there's
an uplift, like that produced by helicopters. This to a certain extend
explains why the RBCs don't end up at the edges of the vessels. However,
there seemed to be a lack of clear explanation to the claim, which
was pointed out during the question answer session which followed.


\chapter{A theoretical study of formation of clusters at nanoscale using reaction-diffusion
models in one and two dimensions}


\section{The Facts}


\subsection*{When, where and who}

The seminar was held on October 24, 2013, 3:00 PM, in LH 3 (LHC).
The speaker was Dr. Trilochan Bagarti (Harishchandra Research Institute,
Allahabad).


\subsection*{The Abstract}

Nano-patterning on surfaces can be achieved by self-organization or
artificially by direct atom manipulation. Surface defects plays a
very important role in the formation of patterns at nanoscale. It
has been observed in experiments that preferential nucleation of self-organized
nano-structures takes place on surfaces along step edges, dislocations
and domain boundaries. In the deposition of Ge atoms on Si(111)-(7x7)
surface, we have found interesting patterns induced by domain boundaries
and step edges. To understand the formation of these defects induced
nano-patterns we have proposed a coupled reaction-diffusion equation
in the presence of surface defects. I will discuss a linear model
which qualitatively explains the experimentally observed pattern.
Then I will discuss nonlinear model which incorporates the effect
of exclusion on these reaction diffusion processes. The effect of
exclusion on simple trapping problem has been found to affect the
stretched exponential behavior of the asymptotic survival probability.
This has been recently verified by our numerical simulation.


\section{What I could note}

The talk given by Dr. Trilochan Bagarti was rather innaccessible to
our feeble minds. However, what I could understand to the most extend,
was the contents of the first slide, the \emph{outline}.
\begin{enumerate}
\item Motivation
\item Patterns in nature
\item Pattern formation in reaction diffusion systems (turing mechanism)
\item Reaction-diffusion systems in disordered media
\item Models

\begin{enumerate}
\item Reaction diffusion models for formation of clusters on a surface with
defects
\end{enumerate}
\end{enumerate}
He said that the motivation was lack of clear enough explanation of
the formation of patterns on diffusion of Ge on Si(111)-(7x7). Patterns
at a nanoscale seem to self organize. This can also be done arificially
by atomic manipulation. He showed an STM image of a clean Si(111)-7x7
surface, where bilayer steps and terraces with domain boundaries could
be seen. The key ingredient necessary for the emergence of pattern
from a uniform structureless state, he said, was instability that
results in spontaneous braking of symmetry.

There was a mechanism discovered by Alan Turing that talked about
the origin of diffusion driven instabilities that can give rise to
such patterns.

To solve this problem, he started with Linear Stability Analysis.%
\footnote{I must mention that the mathematical formulations weren't very accessible
to me, for lack of understanding of certain mathematical tools. However,
I was able to note down some of the equations, which may be useful
in some way.%
} A homogenous stead state solution to this may b egiven by

\[
\dot{u}(t)=\rho(u(t)),\, u(0)=u_{0}
\]


So we linearize $u=u_{s}+\partial u;\,\partial u\sim e^{\omega t+ikx}$and
solve the characteristic equation

\[
det(\omega I+|k|^{2}D-J\}=0
\]


where $J$ is the jacobian.

Then he turned to the turing activator; inhibition model. For this
he defined 
\[
n=2,\ d=2,\ u=(u_{1},u_{2})^{T},\ \rho(u(t))=(f,g)^{T}
\]


\begin{align*}
D_{ij}=\delta_{ij}D_{i}\ i,j=1,2\ D_{1},D_{2}>0\text{ and const}\\
\omega_{1}\omega_{2}<0\ \text{using the characteristic equation}\\
K^{4}-\alpha^{2}\left(\frac{\partial u_{1}f}{D_{1}}+\frac{\partial u_{2}g}{D_{2}}\right)+\left(\frac{\partial u_{1}f\partial u_{2}g-\partial u_{2}f\partial u_{1}g}{D_{1}D_{2}}\right)<0\\
\text{Turing space}\\
\frac{\partial u_{1}f}{D_{1}}+\frac{\partial u_{2}g}{D_{2}}>0\\
\left(\frac{\partial u_{1}f}{D_{1}}+\frac{\partial u_{2}g}{D_{2}}\right)^{2}-4\left(\frac{\partial u_{1}f\partial u_{2}g-\partial u_{2}f\partial u_{1}g}{D_{1}D_{2}}\right)>0
\end{align*}


Next he discussed models of diffusion on disordered media and also
in regular lattice. He further showed simulations of Serpinski Gasket
and Percolation Cluster. He talked about anomolour diffusion in some
detail.

\[
\left\langle R_{N}^{2}\right\rangle =N^{2/d\omega}
\]


where

\[
d\omega=2-d-d_{f}=2.32\pm0.01
\]


He then talked about fractal dimensions, stating $p=\text{fraction of occupied states}$
and at $p=p_{c}=0.592745$(transition). $p<p_{c}$would imlpy only
finite clusturs and $p>p_{c}$implied only large infinite clusters. 

Next he talked about a random walker that walks after waiting a time
$\tau$, continous time random walk. 

$\psi(\tau)\sim\tau_{0}^{4}t^{-1-\mu}$

This was followed by a discussion which had a few more pictures; \emph{Trapping
reaction-diffusion problem}. Diffusing particles arriving at the traps
get annhilated. This is given by 
\[
A+T\rightarrow(1-\epsilon)A+T
\]


So the probability in the case where the traps are distributed uniformly,
and are static, we obtain

\[
p(t)\sim
\]


And for the case of mobile traps we have

\[
p(t)\sim\begin{cases}
e & d=1\end{cases}
\]


This work was done by Balagurov \& vaks, Donsker and Vardhan \& Bramson,
Lebowitz, Bray and others.

If you take a distribution of traps, there'll be large regions where
there're no traps. Consequently there'll be particles that take a
very long time to get trapped. Further, there's no recent work, in
which they have fully modelled moving traps.

He next talked about solving for such a 1 D system, using the \emph{Green's
Function Method}. Assuming the trapping reaction to be in 1D, we have
$\delta=\{x_{1},x_{2},..x_{n,}...\}$ uniformly distributed in $\mathbb{R}$,
we have

\[
\partial_{t}U(x,t)=\nabla\partial_{x}^{2}U(x,t)-k\sum\partial(x-x_{i})\mu(x,t)\ \text{where}\ x\in\mathbb{R},\ t>0
\]


Then, he talked about \emph{perfect traps and the law of stretched
explonential}. The mathematical parts of this have been skipped as
the notes weren't consistent. The idea however was that every particle
that arrives at the trap vanishes. In this case, the x-axis becomes
a collection fo disjoint intervals. He then showed that here, the
survival probability, the lengths of intervals, becomes a poisson
random variable.

\[
p(a,t)=\frac{1}{2a}\intop u(x,t)dx
\]


He went on to talk about the \emph{Geometry of the System}.%
\footnote{At this point I hope it's clear that I couldn't follow most of what
I noted. Only people familiar with the topic have the remote possibility
of gaining something from this. This seems useful other than that,
only for evaluation purposes, although what it is supposed to be evaluating,
is beyond my comprehension at this stage.%
}Real surfaces have domains which have domain boundaries and step edges
which form a bounded regions on the surface. Preferential growth takes
place at these boundaries at a very high rate.

Next, the \emph{Linear Model}, where reaction diffusion happens in
the presence of reaction centres, which are located at radii $r_{1},r_{2},...r_{N}$.
There's a single at the boundary $r=R$. Particle flux is perpendicular
to surface. We have some formulations here that has been skipped.

Reaction diffusion equations in the Laplace domain are written and
then we can write the Green's Function for it. The solution is a linear
set of equations, an $N\times N$ matrix. He discussed in particular
the ring model, and the point model (point defects). He showed a simulation
and its context remarked that the wall like structure exists since
$k_{r}$ is very large.

He then talked about \emph{Stochastic Simulations Algorithm} based
Smoluchowski equation, which went something like this
\begin{enumerate}
\item Drop a particle of type S at a random r, uniform in $\sigma$
\item Choose a particle from the surface
\item Compute $r_{i,t+\delta t}$(there was a formula, which couldn't be
noted)
\item Generate a random number $\rho$. $\rho<k_{i}\delta t$ and $r_{i,t+\delta t}\in\omega_{j}\forall1<j<N$,
perform the reaction step.
\item Goto 1
\end{enumerate}
Then there are some results
\begin{itemize}
\item Mean first passage time 
\[
\left\langle t_{f}\right\rangle \sim\frac{C_{1}}{k_{b}+C_{2}}
\]
where $C_{1}$and $C_{2}$are constants (positive) dependent on $D_{p}$(which
was defined in an earlier part, which couldn't be noted)
\item First passage time probability density
\[
p(t_{f})=A^{2}t_{f}e^{-At_{f}}
\]

\end{itemize}
He supplemented this with various diagrams and then moved on to describe
a particularly simple example. Suppose we start with a constant concentration
and after somet ime this is what it looks like, there'll be formation
of clusters at the origin and decay of particles. If you were to take
the gradient and analyze further, you'll find that more the number
of atoms that come there (the clusters he meant) for the atoms, there's
a repulsion which reduces the incoming rate.%
\footnote{Again, I'm very sorry, I myself am lost and not certain in what context
this is to be placed.%
}

Towards the end, I could only list the topic titles he covered, which
go from \emph{Derivation from the master equation}, \emph{Linearized
Equations} to \emph{Drift Diffusion Equations} and \emph{Trapping
reaction with self exclusion}. About the last topic, he said you could
simulate for trapping with exclusion and numerically take the slope,
however that is not a very simple tast as the slope deals with a log
term.

In conclusion, his second last slide (last being thank you) read
\begin{enumerate}
\item Presented a minimal model for cluster formation at nanoscales in the
presence of surface defects
\item The linear model agrees qualitatively with the eimernets
\item Effects of exclusion and non-linearity were studied
\end{enumerate}
In the question answer session, there was a discussion on the exponent
power $\frac{1}{3}$ which is manifested if you wait for long enough
in a constant rate reaction, say $e^{-kT}$, then it becomes $e^{-\alpha kT}$;
and this had been predicted earlier. Certain theoretical parts were
clarified to not have been experimentally verified. Further, it was
further clarified that no major work on quantum formulation of the
same has been done. Most approaches were classical.


\chapter{Coherences, Photosynthesis and Quantum Biology}


\section{The Facts}


\subsection*{When, where and who}

The seminar was held on November 6, 2013, 3:00 PM, in LH 3 (LHC).
The speaker was Prof. K. L. Sebastian (Department of Inorganic and
Physical Chemistry, Indian Institute of Science).


\subsection*{The Abstract}

It was believed that energy transfer within the photosystem involved
exciton hopping from one site to the next. 2D electronic spectroscopic
experiments carried out by Fleming et al., needed a modification of
this view. The experiments suggested that at least in organisms living
in extreme conditions, there is a wave like energy transfer and that
evolution has optimized the process using quantum coherence. Since
then the energy flow has been studied a lot, both theoretically and
experimentally. Also it has been suggested that quantum mechanics
is important in the most important biological process happening on
earth, leading to the genesis of (according to some scientists) the
subject of quantum biology. An overview of these studies and our theoretical
work in the area will be presented.


\section{What I could follow}

Prof. Sebastian started with talking about claims about biological
systems that perform quantum computation. The outline of the talk
has been listed:
\begin{enumerate}
\item Problem
\item Coherences
\item Environment - Decoherence
\item The FMO photosystem
\item Coherences Experimental Observations
\item The Hamiltonian
\item Approach based on Adiabatic Basis
\item Decoherence and Population Relaxation
\item Summary
\end{enumerate}
So let us start with the problem itself. The FMO complex (named after
Scientists Fenna-Matthews-Olson, workin in the field) consists of
three chlorophyl monomers. Chlorophyl itself has 7 centres. You excite
one of them, and the excitation can travel. There's one centre that
transfers the energy to the reaction centre (number 3). The most prone
to excitation however are two centres remote from number 3, namely
number 1 and 7. This was clear from the diagram which hasn't been
included here. It was earlier an accepted view that these excitations
can jump from one to the next level, traditionally `hopping'. However,
more recent experimental evidence seems to suggest that there exists
a wave like mechanism as opposed to the popular Forster resonance
mechanism. The experimental paper by Fleming et al (2007) showed that
for such a picture, coherences are very important.

The speaker then went on to give a short introduction to coherence.
He began by discussing a superposition state of two wave functions.

\[
\psi=\psi_{1}+\psi_{2}
\]


The non-square terms in the square of $\psi$will constitute what
is called coherence. For clarity, he asked us to imagine superposition
of two electronic states, specifically the s and p states. The question
then is what happens if we let this wave function evolve with time.
We simply operate it with the time evolution operator in the position
basis. When the square of this time evolved state is calculated, it
turns out that the square terms are time independent but the coherent
term picks up time dependence. This results in the oscillation of
the wave function with time. He even showed a simple animation to
get the point through. TODO: Add the math.

The next important question was then how the environment effects the
system. For a physical picture, one can imagine an electron surrounded
by water. One way to proceed would be to start with considering the
effect of the fields produced by the S orbital and P orbital separately.
For the S orbital obviously the field would be radially symmetric.
However for the P orbital it won't be. The environment (the water
molecules) will be affected by this. The water molecules can be approximated
as dipoles being acted upon by the field and they will re-orient,
changing the environment itself, in response to the electronic field.
This interference of the environment must be accounted for in the
wave function of our system and as it turns out, this causes what
is termed as 'decoherence'.

The speaker insisted on noting the distinction between relaxation
and decoherence. This may be expressed mathematically as 

\[
\left(\frac{t_{D}}{t_{g}}\right)^{2}=\frac{6k_{b}T}{\lambda}
\]


where $\lambda$is the stokes shift and the other terms have obvious
association. 

The speaker then talked about an experiment in which 2D electronic
spectroscopy was used to observe an interesting result. The speaker
first described the result and then explained quickly how the experiment
was performed. The result essentially was the visibility of oscillations
of peaks in a biological system, as was expected from the aforesaid
theory, lasting for about 600 fs. Typically they are expected to die
out within 60 fs or so. There've been claims of oscillations lasting
for over 1.2 to 1.5 ps also.

The explanation of the experiment was rather simplistic and not too
deep. The idea was an application of a pulse sequence, which consisted
of three pulses, each of 35 fs duration. The first pulse puts the
ground state into a linear combination of ground and excited state.
The second coverts it to a predominantly excited state and the final
pulse gets them back to the ground and excited state. The time between
the first and second \& the second and third can be changed experimentally.
Thus the signal we get is a function $S(\tau,T,t)$ which is fourier
transformed to the frequency domain $S(\omega_{0},T,\omega_{1})$.
In the 2d spectrum then, with change in T, you can see the cross peaks
and their amplitude oscillating.

Prof. Sebastian then went on to talk about the photosystem. This essentially
consists of one monomer of the FMO complex. The energy difference
between the Highest Occupied Molecular Orbital and the Lowest Unoccupied
Molecular Orbital was stated to be about 12,000 $cm^{-1}$. If we
now look at the energy levels of each of these, then it is found that
the one corresponding to centre number 3 (described earlier) is the
least, as expected. The surroundings are essentially tuned so that
the flow is towards the reaction centre. One can construct a Hamiltonian
which as it turns out, goes as distance power minus three. This may
be understood as some sort of dipole interaction for in two dipole
interactions, we get the form $1/R^{3}$.

Prof. Sebastian then talked about donor and acceptor molecules. Suppose
there are four orbitals, one ground and its corresponding excited
and similarly another set. Now suppose a molecule gets excited form
the lower to higher level. When this molecule comes back to the ground
state, it interacts with the adjacent molecule and puts it in the
excited state.

So the question now is how to describe such a system including the
surroundings. Imagine now that you have 7 molecules. I take their
Hamiltonian without worrying about the surroundings. I can consider
the surroundings to be phonon like sites, with a Hamiltonian approximated
by that of a harmonic oscilator. For each site I can write this. The
coupling then will essentially cause fluctuations which results in
the energy gaps going up and down, which as diagramatically shown
by Prof. Sebastian. Thus one can write the coupling as 

\[
H_{\text{{el-ph} }}=\sum Q_{j}|j\rangle\langle j|
\]


So to solve the problem, the initial conditions are taken to be the
exciton being at the $j^{th}$centre and the temperature bath at $T$.
The hamiltonian for this system however has one problem. The decoherence
and population relxation are caused byt he same term. There're no
small parameters to do purturbation.

An alternative method is to numerically integrate, which a research
group did infact do. The other method is just as complex. 

However, Prof. Sebastian's group took yet another approach and started
with what is popularly known as the Born Openheimer approximation.
For a fixed position, a hamiltonian is written and then the kinetic
energy of the oscillators is added. The eigenfunction then for the
Hamiltonian will correspond to the 7 centres. Reaching the solution
to the Hamiltonian he explained wasn't straight forward for some terms
do not comute. However, the final result is that moving energy levels
are obtained. The method used involved a Markovian Master Equation.

The final result of all the hard work, was displayed on a very interesting
slide. This had results from two different papers, numerically calculated
points, on a population vs time (in femto seconds), together with
Prof. Sebastian's curve, in complete agreement, for various centres
of chlorophyl.

The speaker then went on to talk about how transfer between the eigenstates
of the Hamiltonian considered had been ignored. If the surrounding's
not ignored, then the oscillation doesn't last for long enough. Which
is to say that without the population relaxation, the surroundings
kill the oscillations very quidkly. However, with population relxation,
you can see a transfer to site 3.

Some of the lurking questions were essentially to do with why the
oscillations lasted for so very long and as it turns out, we can answer
it in atleast the following ways.
\begin{enumerate}
\item Protected by the protein
\item Not many water molecules
\item Change in dipole moment is small
\end{enumerate}
Experimentally however, it has been found that there are various water
molecules and they even change orientation which is enough to disturb
the Hamiltonian. So in more realistic calculations, oscillations can't
be observed.

The speaker concluded by stating that an almost analytic approach
to the coherence problem was explored and presented. He also made
use of adiabatic basis with decoherence and relaxation separated.
And ofcourse, the excellent agreement with the results.

The final remark was an answer to the question `Are Bactera Quantum
Computing?'. According to the speaker, such quantum phenomena in biology,
in his opinion, he doubts exist.


\chapter{Mechanics of Information Processing and Computation in Cells}


\section{The Facts}


\subsection*{When, where and who}

The seminar was held on November 6, 2013, 4:00 PM, in LH 3 (LHC).
The speaker was Prof. Madan Rao (RRI, NCBS).


\subsection*{The Abstract}

TODO: Add the right abstract
\end{document}
