%************************************************
\chapter{Brief Outline}
%************************************************
% \begin{flushright}
% August 18, 2012
% \end{flushright}

\section{Introduction}
My summer project consisted of two parts. The first was about reading Linear Algebra in greater depth than was covered in the course, and the second was about exploring Knot theory. For the former, I used the second edition of Michael Artin's Linear Algebra, and for the latter various books were referred to. The most interesting part of the project was the very beauty of mathematical abstraction. It was exceedingly fascinating for me to follow the author's flow and to apply abstractions at various levels to achieve unusual, striking results. Apart from the mainstream mathematics, during the exploration period, I read a little about mathematical history, the development of sigma delta definitions, quarternions and spent significant time on understand mathematical developments on the understanding of Knots. Overall, it was a very enriching experience, the mathematical part of which has been documented in detail using LaTeX, learning which, for typecasting the report, turned out to be another very rewarding skill.

\section{Summary}
Following is a date-wise summary of my work. All numbers that seem arbitrary are sections from the aforesaid Artin's book.
	\begin{enumerate}

		\item May 28
		\begin{enumerate}
			\item Product Groups
			\item Correspondence Theorem
			\item Prime fields
		\end{enumerate}

		\item May 29
		\begin{enumerate}
			\item 3.3 3.5
			\item Computing with bases
			\item Direct Sum
			\item Infinite Dimensional Spaces
			\item Python mod p
		\end{enumerate}
		
		\item May 30
		\begin{enumerate}
		\item 4.2
		\item Linear Operators
		\item Eigen Vectors
		\item Characteristic Polynomials
		\item Triangular and Diagonal Forms
		\item Jordan Form (not completed)
		\end{enumerate}
		
		\item May 31
		\begin{enumerate}
		\item Jordan Form (continued)
		\item Applications of Linear Operators
		\end{enumerate}
		
		\item June 1
		\begin{enumerate}
		\item Proof of Euler's Theorem
		\item Isometries
		\item Change of coordinates
		\item Isometries of the plane
		\item 6.3.5
		\item 6.3.8
		\item Difference between Points and Vectors
		\item Finite groups of Orthogonal Operators on the plane
		\item Fixed Point theorem
		\item Lemma 6.4.9 Corollary 6.4.1
		\item 6.5 intiated
		\end{enumerate}
		
		\item June 2
		\begin{enumerate}
		\item Translation Groups
		\item Theorem 6.5.5
		\item The Point Group
		\item Crystallographic Restriction
		\item Theorem 6.5.12, discrete H
		\end{enumerate}
		
		\item June 3
		\begin{enumerate}
		\item [Sunday]
		\end{enumerate}
		
		\item June 4
		\begin{enumerate}
		\item Revision
		\item Read about Encryption from Michael J Jacobson and Hugh C Williams
		\item Revised change of coordiantes + application in isometries
		\item revising proof of theorem 6.4.1
		\item 6.5 Re analysis of Discrete Groups of Isometries
		\item Crystallographic Restriction
		\end{enumerate}
		
		\item June 5
		\begin{enumerate}
		\item Stuck at Claim of Prop 6.6.4
		\item Understood 6.6.4
		\item 6.7 Abstract Symmetry
		\item 6.8 Operations on Cosets
		\end{enumerate}
		
		\item June 6
		\begin{enumerate}
		\item Built a Telescope for transition of Venus
		\item 
		\item June 7
		\item Revisited the Correspondence Theorem
		\item 6.8 again
		\item 6.9 The counting formula
		\item stuck, trouble with partitioning with orbit using subgroups
		\item 6.9 Completed
		\item 6.10 Operation on subsets
		\item 6.11 Permutation Representation
		\item 6.11.3 Done
		\item 6.12 Finite Subgroups of the Rotation Group
		\item completed till 6.12.6
		\end{enumerate}
		
		\item June 8
		\begin{enumerate}
		\item Revisited Example 6.12.3
		\item LaTeX installation and setup research initiated
		\item 6.12 with latex
		\item Poles with latex
		\end{enumerate}
		
		\item June 9
		\begin{enumerate}
		\item Application of the Famous Formula, Latex
		\end{enumerate}
		
		\item June 10
		\begin{enumerate}
		\item [Sunday]
		\end{enumerate}
		
		\item June 11
		\begin{enumerate}
		\item 6.12, Finite subgroups of the rotation group
		\item Documenting the k-gon orbit of poles
		\item Case 2 with 3 more subcases initated
		\item Thought about the second case
		\end{enumerate}
		
		\item June 12
		\begin{enumerate}
		\item Symmetries of an Octahedron
		\item Documenting that
		\item Icosahedron
		\item *Finished the Symmetry Chapter*
		\end{enumerate}
		
		\item June 13
		\begin{enumerate}
		\item Worked on an interesting physics problem | Motion of water pipe
		\item Looked for books related to Braid groups
		\item Got a book by Christian Kassel issued
		\item Corresponding with sir
		\end{enumerate}
		
		\item June 14, 15
		\begin{enumerate}
		\item Reading Knots: Mathematics with a Twist
		\item Till Chapter 5
		\end{enumerate}
		
		\item June 17
		\begin{enumerate}
		\item [Sunday]
		\end{enumerate}
		
		\item June 26-30, 2012 (Tuesday, the day I returned till Saturday) {done on paper}
		\begin{enumerate}
		\item Revised chapter 5 from the book Knots
		\item Completed reading chapter 6, 7 and 8.
		\item Sent a very brief report
		\item Initiated and completed the following Symmetry problems
		\item 3.6 (a,b)
		\item 4.1
		\item 4.2 (a)
		\item 4.2 (b)
		\item 4.2 (c)
		\item 4.3 (a), (b), (c)
		\item 5.1
		\item 5.5
		\item 5.6 
		\item 5.7 
		\item 5.8 (a)
		\item 5.9
		\item 5.11 (a)
		\item 5.11 (b)
		\item 5.11 (c)
		\item Worked on understanding the precise difference between a vector and a point.
		\item PRoved independently Vivek's theorem about the number of elements necessary and sufficient to generate a symmetric group of order n.
		\item Attempted classification of wallpaper patterns and their symmetry
		\item Getting Certification for KVPY
		\end{enumerate}
		
		\item July 2-7, 2012 (Monday to Friday, Saturday omitted) {mostly on paper, and reading}
		\begin{enumerate}
		\item Redid section 6.8, operation on coset, to gain further familiarity
		\item Met with Prof. Paranjape to discuss the project.
		\item Attempted Problem 7.3 (non-transitive action of S3 on a set of 3 elements)
		\item Initiated chapter 7 and completed till section 7.3
		\end{enumerate}
		
		\item July 8-13, 2012 (Sunday to Friday, Saturday omitted) {reading and documenting in LaTeX}
		\begin{enumerate}
		\item Studied Chapter 7 till section 7.9 (still left to complete)
		\end{enumerate}
		
		\item July 16-22, 2012 (Monday to Sunday) {reading | working on a parallel project}
		\begin{enumerate}
		\item Studied about the History of the delta sigma definition of limits and development of rigorous calculus.
		\item Studied about the development of Quaternions by Hamilton and inspiration because this seemed very unnatural to me at the first glance.
		\end{enumerate}
		
		\item July 23-28, 2012 (Monday to Friday)
		\begin{enumerate}
		\item Read and Typesetted till section 7.10
		\item Read section 7.11
		\item Completed Todd-Coxter's Algorithm
		\end{enumerate}

	\end{enumerate}
