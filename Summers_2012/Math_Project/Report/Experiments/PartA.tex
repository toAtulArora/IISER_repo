\chapter{Subgroups of the Rotation Group}
\begin{flushright} {\small June 8, 2012} \end{flushright}
\section{\textsc {Pole of Group Element: }}
 Let $G$ be a finite subgroup of $SO_{3}$, of order $N > 1$. For now, consider only the rotation elements (of $\mathbb{R}^3) \in G$. Poles for such an element $h(\neq 1) \in SO_{3}$ are defined as the intersection points of the axis of rotation with the unit sphere $\mathbb{S}^2$. $h$ can't be identity since the definition of pole requires existence of an axis. Clearly, for each such $h$, there are 2 poles.
\par
\textsc {Pole of a Group: } Similarly in general, pole is defined for an element $g(\neq 1) \in G$. This pole can also be referred to as pole of the group.
\par
Thus, a pole of $G$, is a point, fixed by a group element $g \neq 1$.\\
\par
\section{\textsc {G-Orbits and Poles: }}
 According to Artin, and I quote ``The set $\mathcal P$ of poles of $G$, is a union of $G$-orbits. So $G$ operates on $\mathcal P$.''
\par
Here's the analysis of the second part of this statement. The set $\mathcal P$ of poles is basically a set of points. Each of these points can be ``operated'' upon by an element of $G$. In accordance with the definition of the word ``operate'' in terms of group action, we must verify that the point obtained after multiplication with an element $g(\neq) 1 \in G$ is also a pole. The other properties of group action are easy to verify.
\par
Let us prove it before continuing.\\ Consider a pole $p \in \mathcal P$.\\
Now $\exists\;h(\neq 1)$ s.t. $hp=p$ \hfill [from the definition of pole]\\
Let $g$ be an element of $G$. We have to show that $gp=q (say)$ is also a pole,\\
that is, $\exists\;k(\neq 1)$ s.t. $kq=q$\\
Lets replace $q$ with $gp$ in both sides, and $p$ with $hp$ in the RHS in the equation above.\\
So we need to solve for $k$ in the equation\\
$\Rightarrow kgp=ghp$\\
$\Rightarrow k=ghg^{-1}$\\
Since $G$ is a group, $ghg^{-1}$ exists and thus k exists, proving $gp$ to also be a pole of the group (specifically of $k$). Hence, we have shown $G$ operates on $\mathcal P$.
\par
Now for the first part of the statement. First, orbits are defined for a particular element of a set. However, if the set is the same as the orbit, then it needn't be specified.\\
{\bf Immediate Context before the doubt | }
In this case, let's fix a pole. The orbit of this pole under the group action would be the set of all poles obtained by operation of all $g \in G$. We just proved that each element of the group, when operates on a pole, creates another pole (of an element present in the group). Union of all the orbits must therefore be equal to $\mathcal P$, as $\mathcal P$ was defined to be the set of all poles of $G$.\\
{\bf Doubt | }
However, Artin has presented this in the other way as can be seen from the quote. What am I missing here?\\

\par
\section{\textsc {Spin: }}
 For a given $g \in G$, spin is the number of unordered pairs $(g,p)$, where $p$ is a pole of the group $G$.
\par
Since each rotation (excluding identity) has two poles, there are two unordered pairs associated, and thus the spin of each such rotation is two.\\
\par
\section{\textsc {Deriving a Famous Formula: }}
 The first objective here is to find a relation between the order of stabilizers for different $p$, and the order of the group, $G$.
\par
Now, for a given pole $p$, all elements $g \in G$, that leave $p$ unchanged, form a set $G_{H}$. $G_{H}$ is a cyclic subgroup and also a stabilizer by definition. The fact that its a subgroup can be verified easily as was done for the kernel. It is cyclic because the group $G$ is finite. It then follows that $G_{H}$ is generated by rotation by the smallest angle $\theta (>0)$ present in the group. If the order of the group $G_{H}$ is given by $r_{p}$, then 
$\theta = 2 \pi/r_{p}$.
\par
{\bf Doubt | } \emph{(Supplement explanation)} Is it correct to note here that the stabilizers $G_{H}$ for different poles $p$, will either form the same subgroup, or be disjoint (can be readily verified). The case would be former if and only if the poles are a result of intersection of the same axis. The latter would happen for all other cases. Also, if such subgroups were created for every pole $p$ of $G$, their union would exhaust the group $G$.\\
Now if all such stabilizers, minus their identity element, are made into a union, they would consequently contain twice as many elements as there are in (${G}$ minus the Identity element).  i.e.\\
\begin{equation}
\sum\limits_{p \in \mathcal{P}} {r_{p} - 1}= 2 \times ({|G|} - 1) 
\label{eqn.orderof_stab_group}
\end{equation}\\
It took me a while to get here. This relation is exactly the same as that given in Artin, but I would want to confirm if the reasoning is correct.
\par
\emph{(Textbook method explained)}
Now since $p$ is a pole, the stabilizer $G_{H}$ will contain at least one element other than identity, and thus $r_{p}>1$. Consequently $r_{p}-1$ elements (since we're excluding identity), stabilize $p$. Each of these elements thus has a spin two.\\
So every group element except the identity has two poles, implying its spin is 2. Taking $|G|=N$, there are $2(N-1)$ spins. Recalling that spin means the number of unordered pairs $(g,p)$ for a given $p$, total spins would be the total number of unordered pairs $(g,p)$. Now if we sum over all $r_{p} - 1$ (for excluding identity), then also we are counting all such pairs.\\
So the same relation (equation \ref{eqn.orderof_stab_group}) follows from this school of thought.\\
\par
So lets move forward. To simplify the LHS of equation \ref{eqn.orderof_stab_group}, we will use the counting formula.
\par
We already know from the counting formula (Size of a Group $G$= Order of coset of $H \times$ Number of Cosets) that
\begin{equation}
|G| \, = \, |G_{H}| \times | \text{Orbit of } G_{H}|
\label{eqn.B}
\end{equation}
since there is a bijective map between the orbit of $G_{H}$ and the cosets of $G_{H}$.
\par
Let $n_{p}$ denote $|$Orbit of $G_{H}|$. Rewriting both equations in terms of $N$, $r_{p}$ and $n_{p}$, we get\\
\begin{equation}
N=r_{p} \times n_{p}
\label{eqn.C}
\end{equation}
\par
Now we can see that if two poles $p$ and $p'$ are in the same orbit, then the order of their orbits is the same, i.e. $n_{p} =n_{p'}$. Equation \ref{eqn.C} demands the order of their Stabilizers to also be the same, i.e. $r_{p}=r_{p'}$.\\
Let's us arbitrarily denote different orbits by $O_{1}, O_{2}, ... O_{k}$. Now we note that if $n_{i}=n_{p}$ (note that the p is the same as it was in the previous context, a particular pole in the summing, although this result is not dependant on it) so by equation \ref{eqn.C} we have $r_{i}=r_{p}$.
\par
It is right here that I got stuck, which caused me to initiate writing like this in the first place. Now the trick here is to realize this very essential fact which is as follows.\\
We are summing over all poles $p$ in equation \ref{eqn.orderof_stab_group}. Now if $\exists\,p \text{   s.t. } p \in O_{i}$, then $\exists\,n_{i}$ poles in the orbit, each with the same number of stabilizers, i.e. $(r_{p} - 1) = (r_{i} -1)$ since $r_{p}=r_{i}$.\\ 
Now we can, from the left side of the equation, take out the contribution of all such poles to the total spin, and express it as $n_{i} \times (r_{i}-1)$. Also, each pole must belong to some orbit, therfore the entire sum (the LHS) may be written as
\begin{equation}
\sum\limits_{i=1}^{k} n_{i} \times (r_{i}-1) = 2 \times (N - 1)
\label{eqn.D}
\end{equation}
Take $r_{i}$ common from LHS and divide by N on both sides, to obtain
\begin{equation}
\sum\limits_{i=1}^{k} (1 - \frac{1}{r_{i}}) = 2 \times (1 - \frac{1}{N})
\label{eqn.E}
\end{equation}
\par
And that was the `famous' formula I'd never even seen before! As the book says, this formula might look small, but its a very strong tool.\\
\hrule
\par
The power of this function will be explored tomorrow.
\par


%%%%%%%%%%%%%%%%%%%%%%%%%%%%%%%%%%%%%%%%%%%%%%%%%%%%%%%%%%%%%%%%%%%%%%%%%%%%%%%%%%%%%%%%%%%%%%%%%%%%%%%%%%%%%%%%%%%%%%%%%%%%%%%%%%%%%%%%%%

\begin{flushright} {\small June 9, 11 \& 12, 2012} \end{flushright}

\section{\textsc {Application of the Famous Formula: }}
The famous formula is:
\begin{equation}
\sum\limits_{i=1}^{k} (1 - \frac{1}{r_{i}}) = 2 \times (1 - \frac{1}{N})
\label{eqn.E}
\end{equation}
Recalling that $N$ is the order of the group which is not trivial, hence $N>1$. Also, $N$ is an whole number, and therefore the smallest value it can have is 2. So the RHS $\geq 1$. Also, as $N \to\infty$, the RHS $\to 2$, but remember N is finite. So effectively the $1\leq$ RHS $< 2$. Also, each term in the LHS $\geq \frac{1}{2}$, since $r_{i} \geq 1$.
\par
Now since the LHS must equal the RHS, there can't be more than 3 terms of LHS, else the sum would become $\geq$ 2, which the RHS can't reach for any value of $N$.
\par
Dividing this into 3 and classifying, we get\\
\emph{One orbit: }\\
So for a single orbit, $k=1$. So, the LHS becomes
\begin{equation}
1-\frac{1}{r} < 1
\end{equation}
while the RHS
\begin{equation}
2 \times (1 - \frac{1}{N}) \geq 1
\end{equation}
So this case is impossible.\\
\emph{Two orbits: }\\
For two orbits, we would have
\begin{equation}
(1 - \frac{1}{r_{1}}) + (1 - \frac{1}{r_{2}}) = 2 - \frac{2}{N}
\end{equation}
which is  the same as
\begin{equation}
\frac{1}{r_{1}} + \frac{1}{r_{2}} = \frac{2}{N}
\end{equation}
{\bf Doubt | } From this itself, Artin concludes that since $r_{i}$ divides $N$, the equation will hold only when $r_{1}=r_{2}=N$. I was unable to see why this was so. However, a little manipulation got me to the same result, but it still doesn't seem obvious to me. What am I missing?\\
Here's what I'd done.\\
Replaced $N$ once with $r_{1}n_{1}$ and one with $r_{2}n_{2}$ to get
\begin{equation}
\frac{1}{r_{1}} + \frac{1}{r_{2}} = \frac{1}{r_{1}n_{1}} + \frac{1}{r_{2}n_{2}}
\end{equation}
rearranged
\begin{equation}
\frac{1}{r_{1}}(1 - \frac{1}{n_{1}}) = \frac{1}{r_{2}}(\frac{1}{n_{2}} - 1)
\end{equation}
simplified
\begin{equation}
\frac{1}{r_{1}n_{1}}(n_{1} - 1) = \frac{1}{r_{2}n_{2}}(1- n_{2})
\end{equation}
since ${r_{1}n_{1} = r_{2}n_{2}}$
\begin{equation}
n_{1} + n_{2} = 2
\end{equation}
And since each orbit must contain atleast one element, $n_{i} \geq 1$. So the only possible solution is\\
$n_{1}=n_{2}=1$\\
$\Rightarrow r_{1}=r_{2}=1$.\\
So since there are only two poles, both fixed by all elements in $G$ hence, the only possibility (of the type of elements in the group) is rotation about a single axis, passing through both these poles (read points!).
\par
{\bf Doubt Context | }
Now as Artin says, is the most interesting case.\\
\emph{Three orbits: } What the text says till Case 1: $r_{1}=r_{2}=2$ and $r_{3}=k$ s.t. $N=2k$, is clear. For further clarity its given as\\
$r_{i}=2,2,k; \,\,\,\, n_{i}=k,k,2; \,\,\,\, N=2K$\\
It goes on to then say that there's one pair of poles ${p,p'}$ making the orbit $O_{3}$. So far so good as it readily follows from the value of $n_{3}$.\\ {\bf Doubt |} This is where I'm stuck.\\
It asserts, \emph{Half} of the elements of $G$ fix $p$, and the other \emph{half} interchange $p$ and $p'$.\\
Elephant in the room is, why Half?\\
This is what I had in mind, but I'm not sure.
\par
My Analysis:\\
Now we know that $O_{3}$, contains 2 elements since $n_{3}$ is 2. For a pole in this orbit, say $p$ as used above, $r_{p}=r_{3}$ [terms have the meaning as per their prior definition]. This means that the stabilizer of the pole, has order k and these are rotations about the axis passing through the origin and the pole (read point) $p$. Since there are only two poles, the other pole $p'$ must lie on this very axis. Thus, the same $K$ stabilizers, stabilize it. However the group has $2K$ elements. The other elements are NOT stabilizers and hence MUST interchange $p \text{ and } p'$. So they are `reflections' which in $\mathbb R^{3}$ become rotations by $\pi$ about a line perpendicular to the line containing the poles. So half of them are fix $p$, other half interchange $p \text{ and } p'$.
\par
So effectively, there are K rotations about the axis $p\,p'$. The rest of the rotations have their axis contained in a plane perpendicular to the $p\,p'$ axis and passing through its mid point. This is so that each such rotation does infact swap the poles $p$ and $p'$. {\bf Doubt | } This is where the story gets even more interesting. I initially thought like so. I imagined a point in the said plane. Then I pictured it getting rotated by an angle $\theta=2\pi/K$ (why this, the rigorous proof is given in the text, basically its because the group [since they're stabilizers] of rotations is finite) along the $p\,p'$ axis. The orbit of the point makes the vertices of a regular K-gon. Then I went on to imagined ``reflecting'' the K-gon, for the orbit is obtained by operating the point with all group elements. However, here's the mistake I made. I ended up reflecting the pentagon (that's what I'd imagined for simplicity) along an axis which doesn't exist in the group! This resulted in expansion of the orbit and that kept me startled for a while, until I realized that the pentagon must be ``reflected'', and by that I mean, rotate by $\pi$ along one of the axis contained in the plane. The result in that case of the ``reflection'' is again the same pentagon. So the orbit obtained in general will be K-gon.
\par
\begin{center}\includegraphics[width=0.7\linewidth]{Chapter_6_images/pentagon_wrong.jpg}
\par
\includegraphics[width=0.7\linewidth]{Chapter_6_images/pentagon_right.jpg}\\
\end{center}
The trouble however is that there are supposed to be two distinct orbits, each with K distinct elements, with 2 stabilizers. Now one stabilizer is identity, and so the other must be rotation by \emph{some} angle (the angle must actually be $\pi$, since $p$ and $p'$ are the only points in their orbit) along an axis containing the element.\\
As has been shown, one of the orbits consists of the poles corresponding to the vertices of the k-gon. (I've come back to poles since poles are what we derived the ``famous formula'' using.)
\par
The text says that the vertices \& the centres of the faces of the K-gon \emph{correspond} to the remaining poles.
\par
So essentially, the centre of faces, which in this case may even be taken to be the centre of the edges, also form an orbit such that each element has 2 stabilizers (one is identity, the other rotation by $\pi$ [effectively a reflection along the chosen axis]).\\
Now two interesting cases arise. When K is odd, say 5 for a pentagon, the poles made by opposite face/edge centre and vertex, correspond to the same axis. Yet, when the pentagon is rotated, the vertices go to the vertices, and the face/edge centres go to themselves, preserving 2 different orbits. Take a moment to realize that there aren't any other poles such that their stabilizer is of order two and they restrict the orbit of $p$ and $p'$ to $p,p'$.\\
In this case, the reflections as was shown earlier, map the pentagon to a pentagon so the orbits are preserved.\\
\begin{center}
\includegraphics[width=0.7\linewidth]{Chapter_6_images/pentagon_rotations.jpg}\\
\end{center}
\par
And when K is even, say 4, the poles made by opposite faces/edge centres form an axis, and those made by opposite vertices, form a different axis. And again, the poles made by the faces goto faces (under rotation by $2\pi/K$) \& correspondingly the poles made by the vertices, goto vertices. So the orbits continue to be separate!
\begin{center}
\includegraphics[width=0.5\linewidth]{Chapter_6_images/square_rotations.jpg}\\
\end{center}
\par
NOTE: I did not include reflection in the analysis above since again, the reflections along both kind of axis, map the square to a square, and the orbits are thereby preserved.\\
\begin{center}
\includegraphics[width=0.4\linewidth]{Chapter_6_images/square_reflections.jpg}
\end{center}
\vspace{40pt}
With that done, lets move on to the next case.\\
The arguments in the book are easy to follow. The conclusion is given as follows:
\begin{aenumerate}
\item $r_{i}=2,3,4;\,\,n_{i}=6,4,4;\,\,N=12$
\par
The poles in the orbit $O_{3}$ are the vertices of a regular tetrahedron, and G is the tetrahedral group T of its 12 rotational symmetries.
\item $r_{i}=2,3,4;\,\,n_{i}=12,8,6;\,\,N=24$
\par
The poles in the orbit $O_{3}$ are the vertices of a regular octahedron, and G is the octahedral group $O$ of its 24 rotational symmetries.
\item $r_{i}=2,3,5;\,\,n_{i}=30,20,12;\,N=60$
\par
The poles in the orbit $O_{3}$ are the vertices of a regular icosahedron, and G is the icosahedral group I of its 60 rotational symmetries.\\
\par
Take a note and verify the fact, that $r_{i}$ represents the number of edges, the number of faces and the number of vertices respectively. Why that's happening (aside from the ordering) is partially answered if one thinks of them as caused by the rotation of different stabilizers (of varied order) to form different orbits.\\
\end{aenumerate}
\par

In Artin the explanation for the number of symmetries for the first two shapes is probably left as an exercise.\\
However an attempt to understand the concept of ``truncated polyhedron'' seems futile without first deriving these simpler, more intuitive results.\\
\begin{aenumerate}
\item
So for a tetrahedron, we can begin the analysis by considering the rotational symmetry about its vertices. Note that the symmetrical axis about a vertex, passes through the centre of the face on the opposite side. So we will not count the rotational symmetry for the faces. So we have a $3$ fold symmetry, of which one element in the group will be identity, which will be common, so there are $2$ distinct group elements corresponding to each vertex. Also, there are $4$ vertices. So there are $2 \times 4 = 8$ non-identity elements in the group, because of symmetry of vertices (and faces). The remaining elements come from the rotation along an axis through the centre of edges, but we must be careful here as well, for each axis of symmetry passes through the centre of 2 edges. Also, the symmetry is $2$ fold, and only $1$ element is therefore non-identity. There are $6$ edges, $3$ of which share an axis, so we have $1 \times 3 = 3$ more non-identity elements in the group.\\
The total then becomes $8 + 3 + 1 \text{ (the identity element) } = 12$ and that's precisely what was expected.
\par
\item
Now the next shape is an octahedron. So first let's resolve the simplest rotational symmetries. Consider rotations about the axis joining opposite vertices, which have a $4$ fold symmetry. There are $6$ vertices and therefore $3$ such axis. So total contribution from this rotational symmetry will be $3 \times (4-1) = 9$. Next consider axis created by joining opposite edge centres. There are $12$ edges \& thus $6$ edge centres. The rotational symmetry is $2$ fold. Thus their contribution is $6 \times (2-1) = 6$. The last symmetry is easier to visualize if we picture a cube, and its centres of faces forming the octahedron. Now consider the body diagonal of the cube. There are $4$ body diagonals and around each there is a $3$ fold symmetry. So the final contribution will be $4 \times (3-1) = 8$. So the total number of symmetries becomes $9 + 6 + 8 + 1 \text { (identity) } = 24$, as was expected and is given in the text.
\par
{\bf Doubt | }Interesting Observation: As was shown in the discussion for the k-gon, poles corresponding to vertices (or edge centres) do NOT span the entire set of poles. Similarly in a cube, the edge centres don't span the poles of a cube, but the figure they span is called a \emph{truncated polyhedron}, as claimed by the text.
\item
To justify this assertion, Artin uses an entire page. So this won't be quite as straight forward. So let's start. Let $V$ be the orbit $O_{3}$ of order $12$. Thus, the order of stabilizer of each pole in the orbit will be $5$. Now we choose any pole $p$, present in $V$ and ``declare it'' to be the north pole of the unit sphere. Now that is sufficient to derive an equator from it (a unit circle with centre at origin, that lies on a plane perpendicular the line containing $p$ and the origin) and also a south pole (diametrically opposite point to $p$). Now we let $H$ be the stabilizer of $p$ which as stated earlier, must have order $5$. Thus $H$ is a cyclic group, generated by a rotation (say $x$) about $p$ with an angle $2\pi/5$. {\bf Doubt | } At this stage Artin asserts there must be 2 H-orbits of order 1 and then goes on to identifying them as north \& south poles.\\
Continuing with my analysis, since $H$ consists of orthogonal operations, thus if $p$ is stabilized, so will its diametrically opposite point be, i.e. the south pole. So $H$ has $1$ orbit with $2$ elements.\\
Now since $H$ has order $5$, the remaining poles (i.e. poles that have a stabilizer of order $5$), form two $H$-orbits of order $5$.
{\bf Doubt | }Now on what rigorous mathematical grounds should I back that argument? I have to somehow show that all poles that have a stabilizer of order $5$, under the action of $H$, form an orbit of order $5$.\\
Back to Artin, by symmetry, either, one of the $H$-orbits is in the northern hemisphere and one is in the southern hemisphere, or else, both are on the equator. Let us name the orbits as $\{q_{0},..,q_{4}\}$ and $\{q'_{0},..,q'_{4}\}$, where $q_{i}=x^{i}q_{0} \text{ and } q'_{i}=x^{i}q'_{0}$. (recall $x$ is the generator of $H$)\\Let $|x,y|$ denote the spherical distance between the points $x$ and $y$ on the unit sphere. We note that $d=|p,q_{i}|$ is independent of $i=0,...,4$.\\
{\bf Doubt | } Now the explanation that Artin provides here, is that $\exists\,\,h \in H \text { s.t. } hq_{0}=q_{i}$. I initially couldn't see how the result follows from this, but since $H$ is just rotations along the polar axis, the spherical distance of a point from the pole remains unchanged, under the action of $H$.\\
The same argument can be used to prove $d' = |p,q'_{i}|$ will also be independent of $i$. Now the only two values $d=|p,p_{i}|$ can take are: $0, \text {(say) }d$. Similarly the only two values $d'=|p,q'_{i}|$ will take would be: $\text{(say) } d', \pi$. So the values $|p,p'|$ where $p' \in V$ (the orbit), are $0,d,d',\pi$.\\
By the definition of $d$ (that is the fact that the elements $q_{i}$ were defined to be the ones between the equator and the north pole, or on the equator), $d\leq \pi/2$. Similarly we can say $d'\geq \pi/2$.\\
Now let's show that the orbit with $5$ elements does NOT lie on the equator. For this, let's first note that the operation of $G$ on $V$ is transitive (simply because $V$ was one of the orbits, so the conclusion follows from the definition of transitivity). This essentially means that we could've picked any $p$ as the north, and $|p,p'|$ would've had the same $4$ possible values. Now this implies, if we choose $q_{i}$ as our north pole (but following the old notation), then $|q_{i},q_{i+1}| = d$ (try visualizing to help your intuition). However, there are $5$ poles in the orbit $q_{i}$ so their angular separation can NOT be $=\pi/2$, for the sum of angles between them MUST add up to $2\pi$ (and not $5\pi/2$!). So this implies their angular separation, which we just showed equals $d$, must be $<\pi/2$ and hence, the poles do NOT lie on the equator.\\
Since $|p,q_{i}|=d=|q_{i},q_{i+1}|$, the north pole $p$ and points of the orbit $q_{i},q_{i+1}$ form equilateral triangles! Since all the triangles share a point $p$, their are five congruent triangles with a common vertex, forming the face of an Icosahedron.\\
{\bf Doubt | } Artin concludes from the above that poles of the group, correspond to the vertices of an Icosahedron. However to me it seems incomplete since the relation between ${q}$ and ${q'}$ hasn't explicitly been derived, although calculation of $d$ and $d'$ (which can be easily computed) should be sufficient to show that the rest of the faces are also built off of the same equilateral triangles.
\end{aenumerate}
\vspace{330pt}
\hrule
The images of the polygons used in this document were specifically created for this purpose.
